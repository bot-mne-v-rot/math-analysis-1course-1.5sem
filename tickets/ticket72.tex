\section{Частные производные высших порядков. Теорема о перестановке частных производных в $\R^2$}

\begin{conj}
    Пусть $f : E \to \R$, $E \subset \R^n$, $a \in \Int E$.
    Если существует $\frac{df}{d x_i}$ в окрестности $a$
    и у неё есть частная производная $\frac{d}{d x_j}$,
    то она называется \textbf{частной производной второго порядка}.
    \textit{Т.е. это частная производная по частной производной}

    Обозначение (длинное):
    $$\frac{d^2 f}{d x_j d x_i} = \frac{d \frac{d f}{d x_i}}{d x_j}$$
    \textit{Храбров считает, что такую запись надо читать так:
    сначала дифф. по $i$-й координате, потом по $j$-й. В некоторых
    книжках считают иначе.}

    Альтернативное обозначение:
    $$f''_{x_i x_j} = (f'_{x_i})_{x_j}$$
\end{conj}

\begin{conj}
    Если существует в окр. точки $a$ частная производная порядка $r$:
    $$ \frac{d^r f}{d x_{i_r} d x_{i_{r-1}} \dots d x_{i_1}} $$
    и у этой производной есть частная производная $\frac{d}{d x_{r+1}}$,
    то результат:
    $$ \frac{d^{r+1} f}{d x_{i_{r+1}} d x_{i_{r}} \dots d x_{i_1}} $$
    называют \textbf{частной производной порядка $r+1$}.
\end{conj}

\textbf{Пример:}
$$f(x, y) = x^y \quad\quad \frac{df}{dx} = y x^{y - 1} 
\quad\quad \frac{df}{dy} = x^y \cdot \ln x$$
\begin{align*}
    \frac{d^2 f}{d x^2} &= \frac{d}{dx} \left( \frac{df}{dx} \right)
    = \frac{d}{dx} \left( y x^{y - 1} \right) = y(y-1) x^{y-2} \\
    \frac{d^2 f}{d y^2} &= \frac{d}{dy} \left( \frac{df}{dy} \right)
    = \frac{d}{dy} \left( x^y \cdot \ln x \right) = x^y \cdot \ln^2 x \\
    \frac{d^2 f}{d y d x} &= \frac{d}{dy} \left( \frac{df}{dx} \right)
    = \frac{d}{dy} \left( y \cdot x^{y - 1} \right) = x^{y-1} + 
    y \cdot x^{y-1} \cdot \ln x \\
    \frac{d^2 f}{d x d y} &= \frac{d}{dx} \left( \frac{df}{dy} \right)
    = \frac{d}{dx} \left( x^y \cdot \ln x \right) =
    y \cdot x^{y - 1} \cdot \ln x + x^y \cdot \frac{1}{x} =
    \frac{d^2 f}{d y d x}
\end{align*}
\textit{Но частные производные по $x,y$ и по $y,x$ совпадают не всегда.}

\textbf{Пример:}
$$
f(x, y) = \begin{cases}
    xy \cdot \frac{x^2 - y^2}{x^2 + y^2} \text{ при } x^2 + y^2 \neq 0,\\
    0, \text{ иначе.}
\end{cases}
$$
Воспользуемся тем, что:
$$\frac{x^2 - y^2}{x^2 + y^2} = 1 - \frac{2y^2}{x^2 + y^2}$$

Пусть $x^2 + y^2 \neq 0$. Тогда:
$$\frac{df}{dx} = y \cdot \frac{x^2 - y^2}{x^2 + y^2} + 
xy(-2y^2) \frac{-2x}{(x^2 + y^2)^2} =
y\frac{x^4 - y^4 + 4x^2y^2}{x^2 + y^2}$$

$$\frac{df}{dx}(0, 0) = \lim_{h \to 0} \frac{f(h, 0) - f(0, 0)}{h} = 0$$
$$\frac{d^2f}{dy dx} (0, 0) = \lim_{h \to 0} 
\frac{\frac{df}{dx}(0, h) - \frac{df}{dx}(0, 0)}{h} =
\lim_{h \to 0} \frac{1}{h} \cdot h \cdot (-1) = -1
$$
Легко заметить, взглянув на формулу, что производная по $x, y$ ---
это производная по $y, x$, но с другим знаком, поэтому:
$$\frac{d^2f}{dx dy} (0, 0) = 1 \neq \frac{d^2f}{dy dx} (0, 0)$$

\begin{theorem}
    Пусть $f : E \to \R$, $E \subset \R^2$, $(x_0, y_0) \in \Int E$,
    частные производные $f'_x$, $f'_y$, $f''_{xy}$ существуют в
    окрестности точки $(x_0, y_0)$ и $f''_{xy}$ непрерывна в 
    $(x_0, y_0)$. Тогда существует $f''_{yx}(x_0, y_0)$ и 
    $f''_{xy}(x_0, y_0) = f''_{yx}(x_0, y_0)$.
\end{theorem}
\begin{proof}
    Введём:
    $$\varphi(s) := f(s, y_0 + k) - f(s, y_0)$$
    Эта функция дифферецируема в окрестности $x_0$, 
    т.к. производная $\varphi(s)$
    --- это частная производная $f$ по $x$ в окрестности $(x_0, y_0 + k)$
    минус частная производная $f$ по $x$ в окрестности $(x_0, y_0)$.

    Т.к. эта функция дифференцируема, мы можем написать для неё теорему
    Лагранжа:
    $$ \varphi(x_0 + h) - \varphi(x_0) = 
    \underset{\text{где } \Theta_1 \in (0, 1)}
    { h \varphi'(x_0 + \Theta_1 h)} =
    h (f_x'(x_0 + \Theta_1 h, y_0 + k) - 
    f_x'(x_0 + \Theta_1 h, y_0)) = \oast $$
    Введём ещё:
    $$ \psi(t) := f_x'(x_0 + \Theta_1 h, t)$$
    Эта функция тоже дифференцируема в окрестности $y_0$,
    т.к. производная этой функции -- это просто частная производная
    $f$ сначала по $x$, потом по $y$. А она существует по условию.
    
    Т.к. эта ф-я дифф., то про неё тоже можно написать теорему
    Лагранжа. Тогда:
    $$ \oast = h(\psi(y_0 + k) - \psi(y_0)) =
    \underset{\text{где } \Theta_2 \in (0, 1)}
    {hk \cdot \psi'(y_0 + \Theta_2 k)} =
    hk \cdot f''_{xy}(x_0 + \Theta_1 h, y_0 + \Theta_2 k) $$

    Получили в итоге:
    $$ f(x_0 + h, y_0 + k) - f(x_0 + h, y_0) - f(x_0, y_0 + k)
    + f(x_0, y_0) = hk \cdot 
    f''_{xy}(x_0 + \Theta_1 h, y_0 + \Theta_2 k) = \circledcirc $$

    Воспользуемся теперь непрерывностью $f''_{xy}$ в точке $(x_0, y_0)$:
    $$ \circledcirc = hk \cdot (f''_{xy}(x_0, y_0) + \alpha(h, k)) 
    \text{, где $\alpha(h, k) \to 0$ при $(h, k) \to 0$} $$
    Поделим выражение на $hk$. Тогда:
    $$ \abs{\alpha(h, k)} = \abs{\frac{1}{h} \left( 
        \underbrace{\frac{f(x_0 + h, y_0 + k) - f(x_0 + h, y_0)}{k}}
        _{\longrightarrow f'_y(x_0 + h, y_0) \text{ при $k \to 0$}} - 
        \underbrace{\frac{f(x_0, y_0 + k) - f(x_0, y_0)}{k}}
        _{\longrightarrow f'_y(x_0, y_0) \text{ при $k \to 0$}}
    \right) - f''_{xy}(x_0, y_0)} $$
    При этом для любого $\varepsilon > 0$ верно, что
    $ \abs{\alpha(h, k)} < \varepsilon$ при малых $h, k$.

    Сделаем предельный переход:
    $$ \abs{\frac{f'_y(x_0 + h, y_0) - f'_y(x_0, y_0)}{h}
    - f''_{xy}(x_0, y_0)} 
    \leqslant \varepsilon \text{ при малых $h$} $$

    Получаем:
    $$ f''_{yx}(x_0, y_0) = \lim_{h \to 0} 
    \frac{f'_y(x_0 + h, y_0) - f'_y(x_0, y_0)}{h}
    = f''_{xy}(x_0, y_0) $$

\end{proof}

\begin{theorem}
    Если $f'_x$ и $f'_y$ определены в окрестности точки $(x_0, y_0)$
    и дифф. в этой точке, то $f''_{xy}(x_0, y_0) = f''_{yx}(x_0, y_0)$.
\end{theorem}
\begin{proof}
    \textit{Упражнение.}
\end{proof}
