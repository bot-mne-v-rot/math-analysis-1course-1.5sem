\section{Теорема Римана}
\begin{theorem}
    (Теорема Римана о рядах)

    $\sum a_n$ условно сходится. Тогда $\forall s \in \overline \R$ существует перестановка $\varphi$,
     такая что $\sum a_{\varphi(n)} = s$. Также существует перестановка, для которой ряд не имеет суммы.
\end{theorem}
\begin{proof}
    Возьмем $\sum(a_n)_+$ и выкинем из него все нули. $\sum b_n$ - ряд, который остался (равносильно тому, что из $\sum a_n$ выкинуть все отрицательные и все нули).

    Возьмем $\sum(a_n)_-$ и выкинем из него все нули, за исключением тех, которые соответствуют нулям в $\sum a_n$, то есть выкидываем положительные слагаемые из $a_n$. Назовем это $\sum c_n$

    Тогда каждое $a$ попало либо в $\sum b_n$ с плюсом, либо в $\sum c_n$ с минусом.
    Что мы понимаем? $\sum a_n$ сходится, значит его члены стремятся к нулю: $a_n \to 0$. А тогда $b_n \to 0$ и $ c_n \to 0$, так как это всё какие-то подпоследовательности. 
    Кроме того, мы знаем, что $\sum b_n = \sum c_n = +\infty$, так как расходились ряды $\sum(a_n)_+$ и $\sum(a_n)_-$.

    Будем брать $b$-шки до тех пор, пока их сумма не станет больше $s$. Это когда-нибудь случится, так как их сумма бесконечность. То есть:
    \begin{gather*}
        b_1 + \ldots + b_{n_1 - 1} \leqslant s < b_1 + \ldots + b_{n_1} =: S_1
    \end{gather*}
    Теперь будем вычитать $c$-шки, пока сумма не станет снова меньше $s$:
    \begin{gather*}
        b_1 + \ldots + b_{n_1} - c_1 - \ldots - c_{m_1 - 1} \geqslant s > b_1 + \ldots + b_{n_1} - c_1 - \ldots - c_{m_1} =: S_2
    \end{gather*}
    Будем повторять процесс.
    \begin{gather*}
        b_1 + \ldots + b_{n_1} - c_1 - \ldots -  c_{m_1} + b_{n_1 + 1} + \ldots + b_{n_2-1} \leqslant s < \\
        < b_1 + \ldots + b_{n_1} - c_1 - \ldots -  c_{m_1} + b_{n_1 + 1} + \ldots + b_{n_2} =: S_3
    \end{gather*}
    \begin{gather*}
        b_1 + \ldots + b_{n_1} - c_1 - \ldots -  c_{m_1} + b_{n_1 + 1} + \ldots + b_{n_2} - c_{m_1+1} - \ldots - c_{m_2-1} \geqslant s > \\
         > b_1 + \ldots + b_{n_1} - c_1 - \ldots -  c_{m_1} + b_{n_1 + 1} + \ldots + b_{n_2} - c_{m_1+1} - \ldots - c_{m_2} =: S_4
    \end{gather*}
    И так далее.

    При этом мы можем делать любое количество шагов, так как $\sum b$ и $\sum c$ расходятся, поэтому мы можем
    выкинуть какое-то количество из начала и остаток будет все равно расходящимся, тогда его частичная сумма тоже будет сколь угодно большой,
    значит мы точно сможем перевалить через $s$ в любую из сторон.

    На каждом таком шаге мы берем хотя бы одну $b$-шку и одну $c$-шку, значит до каждой $a$-шки мы когда-нибудь доберемся, потому
    например $100$-ю положительную $a$-шку мы точно возьмем на 200-м шаге.
    При этом каждую $a$-шку мы берем ровно один раз, значит получаем перестановку.
    
    При этом надо проверить, что это правильная перестановка, то есть, что у нее сумма $s$.

    Знаем, что можем группировать рядом стоящие члены ряда с одним знаком. Это мы и будем делать. 
    Преобразуем уже написанные неравенства:
    \begin{align*}
        S_1 - b_{n_1} \leqslant s < S_1 &\Longrightarrow |s - S_1| \leqslant b_{n_1} \\
        S_2 < s \leqslant S_2 + c_{m_1} &\Longrightarrow |s - S_2| \leqslant c_{m_1}
    \end{align*}
    Тогда в общем виде:
    \begin{gather*}
        |s - S_{2k-1}| \leq b_{n_k} \to 0, \quad |s - S_{2k}| \leq c_{m_k} \to 0
    \end{gather*}
    То есть наша последовательность частичных сумм стремится к $s$. То есть такая группировка допустима, так как имеет ту же сумму, что и исходный ряд. 
    Значит исходный ряд сходится и имеет сумму $s$. С конечным $s$ разобрались

    Для случая $s = +\infty$ модифицируем идею так: сначала сделаем сумму больше 1, потом возьмем одну $c$-шку. Потом сделаем
    сумму больше 2, потом возьмем еще одну $c$-шку, и так далее. 
    
    Чтобы предела не было вообще - берем $b$-шки чтобы стало больше $1$,
    потом $c$-шки, чтобы стало меньше $-1$, бесконечно повторяем.
\end{proof}

\textbf{Замечание.} В $\C$ можно получить не любую сумму (без доказательства).
