\section{Признаки Дирихле и Лейбница. Пример ряда, который сходится равномерно и абсолютно, но не равномерно абсолютно}
Признаки сравнения и Вейерштрасса помогают, когда нам надо понять равномерную сходимость ряда с модулями (иначе говоря, знакопостоянного).
Для знакопеременных рядов хорошо работают признаки Дирихле и Абеля.

\textbf{Признак Дирихле.} 
Пусть \begin{enumerate}
    \item $\left| \sum\limits_{k = 1}^n a_k(x) \right| \leqslant K \;\; \forall n \in N \; \forall x \in E$ -- частичные суммы ряда $\sum\limits_{n = 1}^\infty a_n(x)$ равномерно ограничены.
    \item $b_n \doublerightarrow 0$ -- последовательность $b_n$ равномерно стремится к 0.
    \item $b_n(x)$ монотонны по $n$ при фиксированном $x$.
\end{enumerate}
Тогда ряд $\sum\limits_{n = 1}^\infty a_n(x)b_n(x)$ будет равномерно сходящимся.

\begin{proof}
    Как и в доказательстве обычного признака Дирихле напишем преобразование Абеля: \[ \sum_{k=1}^n a_k(x)b_k(x) = A_n(x)b_n(x) + \sum_{k=1}^{n-1} A_k(x)(b_k(x) - b_{k+1}(x)), \]
    где $A_n(x) = a_1(x) + \dots + a_n(x)$ -- частичная сумма.

    \quad Последовательность $A_n(x)b_n(x) \doublerightarrow 0$, так как просто по условию $A_n(x)$ равномерно ограничены и $b_n \doublerightarrow 0$ (такая теорема была в параграфе про функциональные пос-ти).

    \quad Осталось доказать равномерную сходимость ряда 
    \begin{gather*}
        \sum\limits_{n = 1}^\infty A_n(x)(b_n(x) - b_{n+1}(x))
    \end{gather*}
    Докажем, что он будет абсолютно равномерно сходится.
    Вследствие равномерной ограниченности $A_n(x)$ можем написать такое неравенство: 
    \begin{gather*}
        \sum\limits_{n = 1}^\infty \left|A_n(x)(b_n(x) - b_{n+1}(x))\right| \leqslant K\sum\limits_{n = 1}^\infty \left|b_n(x) - b_{n+1}(x)\right|
    \end{gather*}
    Посмотрим на частичные суммы следующего ряда:
    \begin{gather*}
        \sum\limits_{n = 1}^\infty \left|b_n(x) - b_{n+1}(x)\right|
    \end{gather*}
    Надо доказать, что они равномерно сходятся.
    Воспользуемся монотонностью $b_n(x)$: \[ \sum\limits_{k = 1}^n \left|b_k(x) - b_{k+1}(x)\right| = \left|\sum\limits_{k = 1}^n b_k(x) - b_{k+1}(x)\right| = |b_1(x) - b_{n+1}(x)| \doublerightarrow |b_1(x)| \]
    \quad Таким образом, ряд $K\sum\limits_{n = 1}^\infty \left|b_n(x) - b_{n+1}(x)\right|$ равномерно сходится, и по признаку сравнения сходится ряд $\sum\limits_{n = 1}^\infty \left|A_n(x)(b_n(x) - b_{n+1}(x))\right|$.
\end{proof}

\textbf{Признак Лейбница.} Пусть \begin{enumerate}
    \item $b_n(x) \geqslant 0 \;\; \forall n \in N \; \forall x \in E$.
    \item $b_n(x) \doublerightarrow 0$.
    \item $b_n(x)$ монотонно убывают по $n$ при фиксированном $x$.
\end{enumerate}
Тогда ряд $\sum\limits_{n = 1}^\infty (-1)^nb_n(x)$ будет равномерно сходящимся.

\begin{proof}
    Тут и доказывать нечего, это прямое следствие признака Дирихле.
    Достаточно взять $a_n(x) = (-1)^n$, тогда частичные суммы будут равномерно ограничены, а все условия на $b_n(x)$ уже есть.
\end{proof}

\begin{example}
    Ряд $\sum\limits_{n = 1}^\infty \frac{(-1)^nx^n}{n}$ на $(0, 1)$ сходится асболютно, равномерно, но не равномерно абсолютно.
    Разберем каждую сходимость: \begin{itemize}
        \item Сходится абсолютно, так как при фиксированном $x$ ряд $\sum\limits_{n=1}^\infty x^n$ сходится как геометрическая прогрессия, а мы еще и деление на $n$ добавляем, что только уменьшает члены.
        \item Сходится равномерно по признаку Лейбница.
        \item Но ряд $\sum\limits_{n = 1}^\infty \left|\frac{(-1)^nx^n}{n}\right| = \sum\limits_{n = 1}^\infty \frac{x^n}{n}$ сходится не равномерно.
        Это доказывается по критерию Коши. 
        Рассмотрим отрезок ряда $\sum\limits_{k = N}^{2N} \frac{x^k}{k}$.
        Тогда при $x \to 1$ получаем:
        \begin{gather*}
            \sum\limits_{k = N}^{2N} \frac{x^k}{k} \to \sum\limits_{k = N}^{2N} \frac{1}{k} > \frac{1}{2}
        \end{gather*}
    \end{itemize}
\end{example}

