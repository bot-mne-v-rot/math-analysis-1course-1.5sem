\section{Теорема об интегрировании равномерно сходящейся последовательности (ряда). Существенность равномерности}
Теперь разберемся с интегрированием.

\begin{theorem}
    Пусть $f_n \in C[a, b]$, $f_n \rightrightarrows f$ на $[a, b]$ и $c \in [a, b]$. 
    Тогда $\int_c^x f_n(t)dt \rightrightarrows \int_c^x f(t)dt$.

    В частности $\lim\limits_{n \to \infty} \int_c^x f_n(t)dt = \int_c^x \lim\limits_{n \to \infty} f_n(t)dt$.
\end{theorem}

\begin{proof}
    Введем $F_n(x) := \int_c^x f_n(t)dt$ и $F(x) := \int_c^x f(t)dt $. 
    
    \quad Будем оценивать разность:
    \begin{gather*}
        \begin{split}
            |F_n(x) - F(x)| &= \left|\int_c^x f_n(t)dt - \int_c^x f(t)dt \right| \\
            &\leqslant \int_c^x |f_n(t) - f(t)|dt \text{ (занесли под один интеграл и внесли модуль) } \\
            &\leqslant (x - c) * \max_{t \in [c, x]} |f_n(t) - f(t)| \text{ (самая простая оценка интеграла) } \\
            &\leqslant (b - a) * \max_{t \in [c, x]} |f_n(t) - f(t)| \text{ (еще увеличили длину) }\\
            &= (b - a) * \underbrace{\sup_{t \in [c, x]} |f_n(t) - f(t)|}_{\to 0 \text{ т.к. равн. сх-ть}} \text{ (на отрезке $\max = \sup$) }
        \end{split}
    \end{gather*}
    Наша разность оценилась как что-то стремящееся к 0 и не зависящее от $x$. Это и есть равномерная сходимость.
\end{proof}

\vspace*{7mm}

Можно определить аналогичную вещь для рядов.

\begin{follow}
    Если $u_n \in C[a, b]$ и $\sum\limits_{n=1}^\infty u_n(x)$ равномерно сходится, то $\int_c^x \sum\limits_{n=1}^\infty u_n(t) dt = \sum\limits_{n=1}^\infty \int_c^x u_n(t) dt$.
\end{follow}
\begin{proof}
    В предыдущей теореме $f_n(t) = \sum\limits_{n=1}^\infty u_n(t)$.
\end{proof}

\begin{notice}
    Поточечной сходимости не хватает.
    
    \quad Пример: $f_n(x) = nxe^{-nx^2}$ на $[0, 1]$.

    \quad Очевидно, что $\forall x \;\; f_n(x) \to 0$ при $n \to \infty$. Но \[ \int_0^1 f_n(x)dx = \frac{-e^{-nx^2}}{2}\Big|_0^1 = \frac{1 - e^{-n}}{2} \nrightarrow 0 = \int_0^1 f(x)dx \]
\end{notice}
