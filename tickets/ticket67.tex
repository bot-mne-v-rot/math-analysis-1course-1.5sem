\section{Линейность диференциала. Дифференциал композиции}


\begin{theorem}[линейность дифференциала] $ $

    Пусть 
    \begin{itemize}
        \item $f, g : E \to \R^m$, $E \subset \R^n$;
        \item $a \in \Int E$, $\lambda \in \R$;
        \item $f$ и $g$ дифф. в точке $a$.
    \end{itemize}
    Тогда
    \begin{itemize}
        \item $f + g$ дифф. в точке $a$ и $d_a(f + g) = d_a f + d_a g$;
        \item $\lambda f$ дифф. в точке $a$ и $d_a(\lambda f) = 
        \lambda d_a f$.
    \end{itemize}
\end{theorem}
\begin{proof} $ $
    
    По определению дифференцирумости:

    $f(a + h) = f(a) + d_a f(h) + \alpha(h) \norm{h}$, где
    $\alpha(h) \to 0$ при $h \to 0$. \\
    $g(a + h) = g(a) + d_a g(h) + \beta(h) \norm{h}$, где
    $\beta(h) \to 0$ при $h \to 0$.

    Тогда $f(a + h) + g(a + h) = f(a) + g(a) + 
    \underbrace{d_a f(h) + d_a g(h)}_{\text{лин. оператор}} +
    \overbrace{\underbrace{(\alpha(h) + \beta(h))}_{\to 0} \norm{h}}
    ^{= o(\norm{h})}$.

    Равенство по определению выполняется, значит, 
    $f + g$ дифф. в точке $a$ и $d_a(f+g) = d_a f + d_a g$.

    $\lambda f(a + h) = \lambda f(a) + 
    \underbrace{\lambda d_a f(h)}_{\text{лин. оп-р}} + 
    \overbrace{\lambda \alpha(h) \norm{h}}^{=o(\norm{h})}$.

    Равенство по определению выполняется, значит, 
    $\lambda f$ дифф. в точке $a$ и $d_a(\lambda f) = \lambda d_a f$.
\end{proof}
\textit{В понимании дифференциала как матриц сложение дифференциалов
и умножение дифференциала на скаляр просто соответствуют сложению
матриц и умножению матрицы на скаляр. Хотя искушённому жуковским линалом
читателю, это очевидно.}

\begin{theorem}[дифференцируемость композиции]
    Пусть
    \begin{itemize}
        \item $D \subset \R^n$, $f : D \to \R^m$;
        \item $E \subset \R^m$, $g : E \to \R^l$;
        \item $a \in \Int D$, $f(D) \subset E$, $f(a) \in \Int E$;
        \item $f$ дифф. в точке $a$; $g$ дифф. в точке $f(a)$.
    \end{itemize}
    Тогда $g \circ f$ дифф. в точке $a$ и 
    $d_a (g \circ f) = d_{f(a)} g \circ d_a f$.
\end{theorem}
\begin{proof} $ $

    По определению дифференцирумости:
    \begin{align*}
        f(a + h) &= \overbrace{f(a)}^{=: b} + 
        \overbrace{d_a f(h) + \alpha(h) \norm{h}}^{=: k}, &&
        \text{ где $\alpha(h) \to 0$ при $h \to 0$} \\
        g(b + k) &= g(b) + d_b g(k) + \beta(k) \norm{k}, &&
        \text{ где $\beta(k) \to 0$ при $k \to 0$}
    \end{align*}
    
    Тогда: $$g(f(a + h)) = g(b + k) = 
    g(b) + d_b g(k) + \beta(k) \norm{k}$$

    По линейности линейного отображения:
    $$d_b g(k) = d_b g(d_a f(h) + \alpha(h) \norm{h}) = 
    d_b g (d_a f(h)) + d_b g(\alpha(h)) \norm{h}$$

    Теперь нужно понять, что $d_b g(\alpha(h)) \norm{h} = o(\norm{h})$,
    т.е. $d_b g(\alpha(h)) \to 0$ при $h \to 0$:
    $$
    \norm{d_b g(\alpha(h))} \leqslant 
    \underbrace{\norm{d_b g}}_{\text{const}} \cdot 
    \underbrace{\norm{\alpha(h)}}_{\to 0 \text{ по опр.}} \to 0
    $$

    Теперь нужно доказать, что $\beta(k) \norm{k} = o(\norm{h})$,
    т.е. $\beta(k) \frac{\norm{k}}{\norm{h}} \to 0$ при $h \to 0$:
    $$\norm{k} = \norm{d_a f(h) + \alpha(h) \norm{h}}
    \leqslant
    \norm{d_a f(h)} + \norm{\alpha(h)} \cdot \norm{h}
    \leqslant \norm{d_a f} \cdot \norm{h} + 
    \norm{\alpha(h)} \cdot \norm{h}$$

    По опр. $\alpha(h) \to 0$ при $h \to 0$, значит,
    $\alpha(h)$ ограничена в окрестности 0, при этом 
    $\norm{d_a f} = \text{const}$, поэтому 
    $\norm{k} \leqslant M \norm{h}$ в окрестности 0. Поэтому 
    $\norm{k} \to 0$ при $h \to 0$. Из этого следует, что
    $\beta(k) \to 0$ при $h \to 0$, т.к. $k \to 0$. Получаем:
    $$\beta(k) \frac{\norm{k}}{\norm{h}} \leqslant M \cdot 
    \underbrace{\beta(k)}_{\to 0} \to 0$$

    В исходном равенстве получаем:
    $$g(f(a + h)) = g(b) + d_b g (d_a f(h)) + (d_b g(\alpha(h)) \norm{h}
    + \beta(k) \norm{k}) = g(b) + d_b g (d_a f(h)) + o(\norm{h})$$
    Отсюда по опр. дифф. получаем, что $g \circ f$ дифф. в точке $a$ и
    $(d_a (g \circ f)) (h) = d_b g (d_a f(h))$, а значит,
    $d_a (g \circ f) = d_{f(a)} g \circ d_a f$.

\end{proof}

\textbf{Следствие.} $(g \circ f)'(a) = g'(f(a)) \cdot f'(a)$

\textit{Очень знакомая формула, только теперь не с числами, 
а с матрицами}.
\begin{proof}
    $d_a(g \circ f) = d_{f(a)} g \circ d_a f$. 
    Матрица композиции линейных отображений -- произведение 
    матриц линейных отображении в том же порядке.
\end{proof}
