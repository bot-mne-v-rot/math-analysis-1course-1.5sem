\section{Задача Коши для дифференциального уравения. Теорема Пикара}
\begin{conj}
\textbf{Задача Коши для дифференциального уравнения}. Хотим найти функцию $y\colon (a, b) \to \R$ дифференцируемую такую, что
\begin{itemize}
    \item $y'(x) = f(x, y(x))$.
    \item $y(x_0) = y_0$, где $x_0 \in (a, b)$.
\end{itemize}
\end{conj}

\begin{examples}
\begin{enumerate}
    \item
    Хотим решить систему
    \begin{equation*}
    \begin{cases}
        y' = y \\
        y(0) = 1
    \end{cases}
    \end{equation*}
    Тогда:
    \begin{equation*}
    \frac{y'(x)}{y(x)} = 1 \implies (\ln y(x))' = 1 \implies \ln y(x) = x + C \implies y(x) = e^{x + c} \implies y(x) = Ce^x
    \end{equation*}
    Дальше подставляем $y(0) = 1$ и понимаем, что $C = 1$, а значит $y(x) = e^x$.

    \item
    Пример отображения заданного не на всей прямой: система
    \begin{equation*}
    \begin{cases}
        y' = -y^2 \\
        y(1) = 1
    \end{cases}
    \end{equation*}
    имеет решение $y(x) = 1/x$. Легко проверить что такая функция подходит и можно показать, что это решение единственно.
\end{enumerate}
\end{examples}

\begin{theorem}[Пикара]
Пусть:
\begin{itemize}
    \item $f\colon D \to \R$, где $D \subset \R^2$ и $D$ --- открыто.
    \item $f$ --- непрерывна.
    \item $(x_0, y_0) \in D$.
    \item $|f(x, y) - f(x, \widetilde{y})| \leqslant M |y - \widetilde{y} | \;\; \forall y, \widetilde{y}$.
\end{itemize}
Тогда при некотором $\delta > 0$ на отрезке $[x_0 - \delta, x_0 + \delta]$ существует единственная 
\begin{equation*}
    \varphi\colon [x_0 - \delta, x_0 + \delta] \to \R
\end{equation*}
такая, что $\varphi'(x) = f(x, \varphi(x))$ и $\varphi(x_0) = y_0$(то есть $\varphi$ --- решение задачи Коши).
\end{theorem}
\begin{proof}
Найдем функцию, удовлетворяющую равенству
\begin{equation*}
    \varphi(x) = y_0 + \int_{x_0}^{x} f(x, \varphi(x)) \: dx
\end{equation*}
Во-первых $\varphi(x_0) = y_0$(интеграл от отрезка длины ноль равен нулю), во-вторых $\varphi(x)$ --- дифференцируемая, так как равна интегралу от непрерывной функции и в-третьих $\varphi'(x) = f(x, \varphi(x))$ просто потому что мы взяли такой интеграл. Поэтому такое $\varphi$ точно подойдет.

Пусть $r$ такой, что $\Cl B_r(x_0, y_0) \subset D$. Тогда по теореме Вейерштрасса $f$ ограничена на $B_r(x_0, y_0)$, то есть $|f(x, y)| \leqslant K \;\; \forall x, y \in B_r(x_0, y_0)$.

Выберем $\delta > 0$ так, что:
\begin{enumerate}
    \item $M \delta < 1$.
    \item Если $|x - x_0| \leqslant \delta$ и $|y - y_0| \leqslant K \delta$, то $(x, y) \in B_r(x_0, y_0)$(то есть $\delta^2(1 + K^2) \leqslant r^2$).
\end{enumerate}

Пусть:
\begin{equation*}
    C^{*} = \{\varphi \in C[x_0 - \delta, x_0 + \delta]\text{ и } |\varphi(x) - y_0| \leqslant K\delta\} \subset C[x_0 - \delta, x_0 + \delta]
\end{equation*}
Заметим, что $C^{*}$ --- замкнутое подмножество полного пространства $C[x_0 - \delta, x_0 + \delta]$, а значит оно само является полным пространством.

Рассмотрим отображение $T\colon C^{*} \to C^{*}$ такое, что $T(\varphi) = \psi$, где
\begin{equation*}
    \psi(x) = y_0 + \int_{x_0}^{x} f(t, \varphi(t)) \: dt
\end{equation*}

Проверим, что $T$ --- корректно заданное отображение:
\begin{enumerate}
    \item Первое условие $\psi \in C[x_0 - \delta, x_0 + \delta]$ --- непрерывна потому что равна какому-то интегралу от непрерывной функции.
    \item Второе условие $|\psi(x) - y_0| \leqslant K \delta$:
    \begin{equation*}
    |\psi(x) - y_0| = 
    \left | \int_{x_0}^{x} f(t, \varphi(t)) \right | \leqslant
    \int_{x_0}^{x} | f(t, \varphi(t)) | \leqslant \oast
    \end{equation*}
    Заметим, что $|x - x_0| \leqslant \delta$, просто потому что мы рассматриваем такие $x$, а так же $| \varphi(t) - y_0 | \leqslant K \delta$, потому что отображение действует из $C^{*}$. Значит $(t, \varphi(t)) \in B_r(x_0, y_0) \implies f(t, \varphi(t)) \leqslant K$. Таким образом мы можем ограничить наш интеграл констаной $K$, умноженной на длину отрезка:
    \begin{equation*}
    \oast \leqslant
    K |x - x_0|
    \leqslant K \delta
    \end{equation*}
\end{enumerate}
Теперь проверим, что $T$ --- сжатие:
\begin{equation*}
    \begin{gathered}
    |\psi(x) - \widetilde{\psi}(x)| =
    \left | \int_{x_0}^{x} f(t, \varphi(t)) \: dt - \int_{x_0}^{x} f(t, \widetilde{\varphi}(t)) \: dt \right | \leqslant
    \int_{x_0}^{x} | f(t, \varphi(t)) - f(t, \widetilde{\varphi}(t)) | \: dt 
    \leqslant \\ \leqslant
    M \int_{x_0}^{x} | \varphi(t) - \widetilde{\varphi}(t)| \: dt \leqslant
    M |x_0 - x| \| \varphi - \widetilde{\varphi} \| \leqslant
    M \delta \| \varphi - \widetilde{\varphi} \|
    \end{gathered}
\end{equation*}
Таким образом у нас получилось сжатие с коэффициентом $M\delta$(что меньше единицы, потому что мы выбирали так $\delta$), а значит существует ровно одна неподвижная точка для такого отображения. Тогда легко видеть, что такое отображение будет искомым(функцией из начала доказательства) и почему оно будет решением задачи Коши уже было доказано раннее. Единственность же верна из единственности неподвижной точки в теореме Банаха.
\end{proof}
