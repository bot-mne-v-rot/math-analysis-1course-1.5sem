\section{Мультииндексы. Определения, обозначения, лемма о производной композиции гладкой и линейной функций. Многомерный Тейлор с остатком в форме Лагранжа. Частные случаи}

Так как мы поняли, что в случае $r$-гладких функций порядок взятия 
производных не имеет значения, а важно только то, сколько раз мы берём 
производную по 
каждой переменной производную;  введём следующее более компактное
обозначение: 
\begin{conj}
    \textbf{Мультииндекс.}
    $$ k = (k_1, \dots, k_n) \text{, где $k_i$ --- неотрицательные
    целые числа, $n$ --- размерность области опр. $f$} $$
    \textbf{Высота мультииндекса}: 
    $\quad \abs{k} = k_1 + \dots + k_n$.
    
    \textbf{Факториал мультииндекса}: 
    $k! = k_1! \cdot k_2! \cdot ... \cdot k_n!$.

    \textbf{Возведение вектора $h \in \R^n$ в степень мультииндекса}:
    $h^k = h_1^{k_1} \cdot h_2^{k_2} \cdot ... \cdot h_n^{k_n} \in \R$.

    \textbf{Частная производная по мультииндексу:}
    $$ f^{(k)} = \frac{d^{\abs{k}} f}
    {d x_1^{k_1} \cdot \dots \cdot d x_n^{k_n}} $$
\end{conj}

\begin{conj}
    \textbf{Мультиномиальный коэффициент:}
    $$ \binom{\abs{k}}{k_1, k_2, \dots, k_n} = \frac{\abs{k}!}{k!} $$

    Комбинаторный смысл: покрасить $\abs{k}$ шаров в $n$ -- цветов так,
    что $k_1$ шаров 1-го цвета, $k_2$ шаров -- 2-го и т.д.
    \begin{gather*} 
        \binom{\abs{k}}{k_1, k_2, \dots, k_n} =
        C^{k_1}_{\abs{k}} \cdot 
        C^{k_2}_{\abs{k} - k_1} \cdot 
        C^{k_3}_{\abs{k} - k_1 - k_2} \cdot \dots 
        C^{k_n}_{\abs{k} - k_1 - ... - k_{n-1}} = \\
        = \frac{\abs{k}!}{k_1! \cdot \cancel{(\abs{k} - k_1)!}}
        \cdot \frac{\cancel{(\abs{k} - k_1)!}}
        {k_2! \cdot \cancel{(\abs{k} - k_1 - k_2)!}}
        \cdot \dots \cdot
        \frac{\cancel{(\abs{k} - k_1 - \dots - k_{n-1})!}}
        {k_n! \cdot \underbrace{(\abs{k} - k_1 - \dots - k_n)!}
        _{= 0! = 1}} =
        \\ = \frac{\abs{k}!}{k_1! k_2! \dots  k_n!}
        =  \frac{\abs{k}!}{k!}
    \end{gather*}
\end{conj}

%Техническая лемма на последних 20 минутах 11ой лекции
\begin{lemma}
    $f \in C^r(D), \; D \subset \R^n$ -- открыто, отрезок $[x, x+h]$ целиком содержится в $D$. Введем функцию
    \begin{gather*}
        F(t) := f(x + th), \qquad F:[0, 1] \longrightarrow \R 
    \end{gather*}
    На отрезке $[0, 1]$ функция точно корректно определена, так как мы затребовали, чтобы весь отрезок $[x, x+h]$ лежал в $D$. 
    Тогда лемма утверждает, что:
    \begin{gather*}
         F \in C^r[0, 1] \text{ и } F^{(l)}(t) = \sum\limits_{\abs{k} = l} \binom{l}{k_1, k_2, \dots, k_n} f^{(k)} (x+th)h^k
    \end{gather*}
\end{lemma}
\begin{proof}
    Давайте сначала попробуем понять, что происходит для одной производной. Введем
    $G(t) := g(x+th)$, где $g$ -- некоторая другая функция из $D$ в $\R$. 
    Хотим посчитать ее производную в некой точке. Так как функция $G$ -- это по сути композиция двух функций, одна из которых 
    с векторным аргументом, а другая -- принимает скаляр и выплевывает вектор, то по формуле производной композиции:
    \begin{gather*}
        G'(t) = g'(x + th)\cdot(x+th)' = \oast
    \end{gather*} 
    Давайте понимать, что есть что. $g$ у нас дейстовало из $D$ в $\R$, то есть матрица $g'(x+th)$ -- это на самом деле вектор строчка из частных производных. 
    В то же время, $x+th:[0,1] \longrightarrow \R^n$. То есть это просто столбец координат $x_i + th_i$. Значит производная этой штуки -- просто столбец производных координат по $t$. 
    А производные по $t$-шкам -- это соответствующие им $h$--ки. Получаем: 
    \begin{gather*}
        \oast = \left( g'_{x_1}(x+th), \dots, g'_{x_n}(x+th) \right) \cdot \begin{pmatrix*}
            h_1 \\
            h_2 \\
            \vdots \\
            h_n
        \end{pmatrix*} = \sum\limits_{i=1}^n g'_{x_i}(x+th)h_i
    \end{gather*}
    Вот мы посчитали первую производную, получили что-то страшное. Когда мы будем дифференцировать много раз -- будем много раз брать производные и все складывать. Выйдет куча вложенных друг в друга сумм. 
    Давайте посмотрим, что мы получим, взяв $l$-ую производную (перейдем к функции $F$):
    \begin{gather*}
        F^{(l)} = \sum\limits_{i_l} \dots \sum\limits_{i_1} \frac{d^l f}{dx_{i_l} dx_{i_{l-1}} \dots dx_{i_1}}(x+th)h_{i_1}h_{i_2}\dots h_{i_l} = \oast
    \end{gather*}
    Давайте свернем эту сумму. Введем такой вот мультииндекс $k$: 
    \begin{gather*}
        k = (\# \{j: i_j = 1\}, \# \{j: i_j = 2\}, \dots, \# \{j: i_j = n\})
    \end{gather*}
    Тогда в нашей сумме:
    $$ \frac{d^l f}{dx_{i_l} dx_{i_{l-1}} \dots dx_{i_1}}(x+th)
    = f^{(k)}(x + th) \quad \quad h_{i_1}h_{i_2}\dots h_{i_l} = h^k $$
    Теперь нужно понять, сколько раз мы такое слагаемое встретим.
    По сути, мы $x_1$ ``красим'' $k_1$ раз, $x_2$ ``красим'' $k_2$ раз 
    и т.д.
    Количество способов так ``покрасить'' и есть 
    комбинаторный смысл мультиномиального коэффициента.

    Тогда нашу сумму можно переписать с использованием этого мультииндекса как: 
    \begin{gather*}
        \oast = \sum\limits_{\abs{k} = l} f^{(k)} (x+th) \binom{l}{k_1, k_2, \dots, k_n} h^k
    \end{gather*}
\end{proof}

\begin{theorem} (Многомерная формула Тейлора)
    
    $D \subset \R^n$ -- открытое, $f \in C^{r+1}(D)$ и $[a, x] \subset D$.
    Тогда существует такое $\Theta \in (0, 1)$, что:
    \begin{gather*}
        f(x) = \sum\limits_{\abs{k} \leqslant r} \frac{f^{(k)}(a)}{k!}(x-a)^k + \sum\limits_{\abs{k} = r+1} \frac{f^{(k)}(a + \Theta(x-a))}{k!}(x-a)^k
    \end{gather*}
    \underline{Мини-замечание}: Все действия производятся с мультииндексами
\end{theorem}
\begin{proof}
    Доказательство не слишком мудрёное, в основном мы пользуемся леммой, которую мы только-только доказали. 
    Введем $F(t) := f(a + th)$ и $h = x - a$. Напишем адекватного Тейлора для $F$ в единице, с остатком в форме Лагранжа: 
    \begin{gather*}
        F(1) = \sum\limits_{j=0}^r \frac{F^{(j)}(0)}{j!} + \frac{F^{(r+1)}(\Theta)}{(r+1)!} 
    \end{gather*}
    Теперь воспользуемся леммой, чтобы разложить производные $F$ разных порядков и выразим $f(x)$:
    \begin{gather*}
        f(x) = F(1) = \sum\limits_{j=0}^r \left[ \frac{1}{\cancel{j!}} \cdot \sum\limits_{\abs{k} = j} \stackabove{\binom{j}{k_1, k_2, \dots, k_n}}{\cancel{j!}/k!} f^{(k)}(a)(x-a)^k\right] + \\
        + \frac{1}{\cancel{(r+1)!}} \cdot \sum\limits_{\abs{k} = r+1} \stackbelow{\binom{r+1}{k_1, k_2, \dots, k_n}}{\cancel{(r+1)!}/k!} f^{(k)}(a + \Theta(x-a))(x-a)^k
    \end{gather*}
    Сокращаем и получаем в точности ту формулу, которую хотели.
\end{proof}
\notice

\begin{enumerate}
    \item Первая сумма в формуле -- многочлен Тейлора степени $r$:
    \begin{gather*}
        \sum\limits_{\abs{k} \leqslant r} \frac{f^{(k)}(a)}{k!}(x-a)^k
    \end{gather*}
    \item При $r = 0$, получаем, что:
    \begin{gather*}
        f(x) = f(a) + \sum\limits_{i=1}^n f_{x_i}' (a + \Theta(x-a))(x_i - a_i) = \\
        f(a) + \langle \nabla f(a + \Theta(x-a)), x-a \rangle
    \end{gather*}
    Это есть многомерная версия формулы Лагранжа.
    \item При $n = 2$:
    \begin{gather*}
        f(x, y) = f(a, b) + f_x'(a, b)(x - a) + f_y'(a, b)(y - b) + \\
        + \frac{f_{xx}''(a, b)(x-a)^2}{2} + \frac{f_{yy}''(a, b)(y-b)^2}{2} + f_{xy}''(a, b)(x-a)(y-b) + \\
        + \frac{f_{xxx}'''(a, b)(x-a)^3}{6} + \frac{f_{yyy}'''(a, b)(y-b)^3}{6} + \frac{f_{xxy}'''(a, b)(x-a)^2(y-b)}{2} + 
        \frac{f_{xyy}'''(a, b)(x-a)(y-b)^2}{2} \\ 
        + \dots 
    \end{gather*}
\end{enumerate}