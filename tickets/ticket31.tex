\section{Преобразование Абеля. Признак Дирихле}
\textbf{Преобразование Абеля.} (дискретный аналог интегрирования по частям) \[ \sum_{k = 1}^n a_kb_k = A_nb_n + \sum_{k = 1}^{n - 1} A_k(b_k - b_{k+1}), \] где $A_k = a_1 + \dots + a_k$, считая $A_0 = 0$.
\begin{proof}
    \begin{gather*}
        \begin{split}
            \sum_{k = 1}^n a_kb_k &= \sum_{k = 1}^n (A_k - A_{k-1})b_k \\
            &= \sum_{k = 1}^n A_kb_k - \sum_{k = 1}^n A_{k-1}b_k \text{ (разделили на 2 суммы)} \\
            &= A_nb_n + \sum_{k = 1}^{n-1} A_kb_k - \sum_{k = 2}^n A_{k-1}b_k \text{ (вынесли одно слагаемое и убрали $A_0 = 0$)} \\
            &= A_nb_n + \sum_{k = 1}^{n-1} A_kb_k - \sum_{j = 1}^{n-1} A_jb_{j+1} \text{ (сменили переменную)} \\
            &= A_nb_n + \sum_{k = 1}^{n-1} A_k(b_k - b_{k+1}) \text{ ($j$ и $k$ пробегают одни и те же индексы)} \\
        \end{split}
    \end{gather*}
\end{proof} 

\textbf{Признак Дирихле.} 

 Пусть \begin{enumerate}
    \item $A_n := \sum\limits_{k = 1}^n a_k$ ограничены
    \item $b_n$ монотонно стремится к 0
\end{enumerate}
Тогда $\sum a_nb_n$ сходится.
\begin{proof}
    Применим преобразование Абеля: $S_n := \sum\limits_{k=1}^n a_kb_k = A_nb_n + \sum\limits_{k=1}^{n-1} A_k(b_k - b_{k+1})$.
    \quad Надо доказать, что существует конечный $\lim S_n$, т.е. конечные $\lim\limits_{n \to \infty} A_nb_n$ и $\lim\limits_{n \to \infty} \sum\limits_{k=1}^{n-1} A_k(b_k - b_{k+1})$.
    С первым пределом все понятно, так как $A_n$ ограничены, а $b_n \to 0$.
    Существование второго предела равносильно сходимости ряда $\sum\limits_{k=1}^{\infty} A_k(b_k - b_{k+1})$.

    \quad Докажем, что он сходится абсолютно. 
    По условию $|A_n| \leqslant M$, а $b_n$ монотонно стремятся к 0, причем будем считать, что они убывают (если возрастают, поменяем знаки).
    Тогда \[ |A_k(b_k - b_{k+1})| \leqslant M|b_k - b_{k+1}| = M(b_k - b_{k+1}) \]
    \quad То есть мы можем ограничить таким рядом: $ \sum\limits_{k=1}^{\infty} M(b_k - b_{k+1}) = M\sum\limits_{k=1}^{\infty} (b_k - b_{k+1}) $. 
    Его частичная сумма это телескопическая сумма, после сокращения получаем $\sum\limits_{k=1}^{n} = b_1 - b_{n+1} \to b_1$.
    Следовательно, $M\sum\limits_{k=1}^{\infty} (b_k - b_{k+1}) = Mb_1$, то есть он сходится, а значит сходится и наш ряд по признаку сравнения.
\end{proof}
