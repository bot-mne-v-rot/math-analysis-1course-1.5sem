\section{Частные производные. Элементы матрицы Якоби. Координатная запись формул для производных}


\begin{conj}
    Пусть $f : E \to \R$. $a \in \Int E$. Тогда 
    \textbf{частная производная} -- производная по направлению
    $k$-го базисного вектора (по напр. $k$-й координаты):
    $$\frac{df}{dx_k}(a) := \lim_{t \to 0} 
    \frac{f(a + te_k) - f(a)}{t}$$
    где $e_k$ -- $k$-й базисный вектор (нулевой вектор, 
    кроме 1 на $k$-й позиции).
    $$ 
    e_k = \begin{pmatrix*}
        0 \\ \vdots \\ 0 \\ 1 \\ 0 \\ \vdots \\ 0
    \end{pmatrix*}
    \begin{matrix*}
        $ $ \\ $ $ \\ $ $ \\
        \longleftarrow \text{$k$-е место}
        \\ $ $ \\ $ $ \\ $ $
    \end{matrix*}
    $$
    Обозначений много: $f'_k$, $D_k f$, $d_k f$, $\frac{df}{dx_k}$. 
\end{conj}
 
\textbf{Следствие 2:}
\begin{enumerate}
    \item[] \textbf{Каноническое определение градиента}.
    $$\frac{df}{dx_k}(a) = \langle \nabla f(a), e_k \rangle =
    \text{$k$-я координата } \nabla f(a)$$
    Другими словами:
    $$ \nabla f(a) =
    \begin{pmatrix*}
        \frac{df}{dx_1}(a) \\
        \frac{df}{dx_2}(a) \\
        \vdots \\
        \frac{df}{dx_n}(a) \\
    \end{pmatrix*}$$
    \textit{На лекции Храбров записал вектор-строку, но мы в $\R^n$ 
    используем векторы-столбцы}.
\end{enumerate}

\textbf{Следствие 3:}
\begin{enumerate}
    \item[] 
    Если $f : E \to \R^m$, $E \subset \R^n$, $a \in \Int E$,
    $f$ дифф. в точке $a$, то матрица Якоби в точке $a$
    выглядит следующим образом:
    $$ f'(a) =
    \begin{pmatrix*}
        \frac{df_1}{dx_1}(a) & \frac{df_1}{dx_2}(a) &
        \dots & \frac{df_1}{dx_n}(a) \\
        \frac{df_2}{dx_1}(a) & \frac{df_2}{dx_2}(a) &
        \dots & \frac{df_2}{dx_n}(a) \\
        \vdots & \vdots & \ddots & \vdots \\
        \frac{df_m}{dx_1}(a) & \frac{df_m}{dx_2}(a) &
        \dots & \frac{df_m}{dx_n}(a) \\
    \end{pmatrix*}$$
    где $f_k$ -- $k$-я координатная функция $f$.
\end{enumerate}

\textbf{Пример:}

Пусть $f(x, y) = x^y$. Тогда
$$
\frac{df}{dx}(a, b) = 
\lim_{t \to 0} \frac{f((a, b) + t(1, 0)) - f(a, b)}{t} =
\lim_{t \to 0} \frac{f(a + t, b) - f(a, b)}{t} =
(x^b)' \mid_{x = a} = b a^{b-1}
$$
\textit{У нас меняется только первая координата,
т.е. по сути это определение производной в точке $x = a$ при
фиксированном $y = b$. Т.е. мы просто воспринимаем $y$ как параметр
и дифференцируем по одной переменной.}

$$
\frac{df}{dy}(a, b) = 
\lim_{t \to 0} \frac{f((a, b) + t(0, 1)) - f(a, b)}{t} =
\lim_{t \to 0} \frac{f(a, b + t) - f(a, b)}{t} =
(a^y)' \mid_{y = b} = a^b \cdot \ln a
$$

\textit{Получается, что, когда мы считаем частную производную по $x_k$,
мы фиксируем все остальные переменные, отличные от $x_k$ ---
считаем, что они константы ---
и просто получаем функцию от одной переменной.}
