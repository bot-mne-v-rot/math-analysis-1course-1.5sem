\section{Две теоремы о дифференцируемости произведения функций}


\begin{theorem}[дифф. произведения скалярной и векторной функции] $ $

    Пусть 
    \begin{itemize}
        \item $E \subset \R^n$, $\lambda : E \to \R$, $f : E \to \R^m$;
        \item $a \in \Int E$, $f$ и $\lambda$ дифф. в точке $a$
    \end{itemize}
    Тогда $\lambda f$ дифф. в точке $a$ и $d_a (\lambda f)(h)
    = d_a \lambda (h) \cdot f(a) + \lambda(a) \cdot d_a f(h)$.
\end{theorem}
\begin{proof} $ $

    По определению дифференцирумости:
    \begin{align*}
        f(a + h) &= f(a) + d_a f(h) + \alpha(h) \norm{h}, &&
        \text{ где $\alpha(h) \to 0$ при $h \to 0$} \\
        \lambda(a + h) &= \lambda(a) + d_a \lambda(h) + \beta(h) \norm{h}, &&
        \text{ где $\beta(h) \to 0$ при $h \to 0$}
    \end{align*}

    Получаем:
    \begin{gather*}
        f(a + h) \cdot \lambda(a + h) - f(a) \cdot \lambda(a) = 
        \underbrace{d_a \lambda(h) \cdot f(a) + 
        \lambda(a) \cdot d_a f(h)}_{\text{то что нужно}} + \\ +
        \underbrace{\lambda(a) \cdot \alpha(h) \norm{h}}
        _{\text{очев. $=o(\norm{h})$}} + 
        \underbrace{f(a) \cdot \beta(h) \norm{h}}
        _{\text{очев. $=o(\norm{h})$}} +
        \underbrace{d_a \lambda(h) \cdot d_a f(h)}_{\text{1.}} + \\ +
        \underbrace{d_a \lambda(h) \cdot \alpha(h) \norm{h}}_{\text{2.}} + 
        \underbrace{d_a f(h) \cdot \beta(h) \norm{h}}_{\text{2.}} +
        \underbrace{\alpha(h) \beta(h) \norm{h}^2}_{\text{3.}}
    \end{gather*}
    \begin{enumerate}
        \item Поймём, что $d_a \lambda(h) \cdot d_a f(h) = o(\norm{h})$:
        $$\norm{\underbrace{d_a \lambda(h)}_{\text{число}} \cdot d_a f(h)}
        = \abs{d_a \lambda(h)} \cdot \norm{d_a f(h)} \leqslant
        \norm{d_a \lambda} \cdot \norm{h} \cdot \norm{d_a f(h)}
        \leqslant \underbrace{\norm{d_a \lambda} \cdot \norm{d_a f}}
        _{\text{= const}} \cdot \norm{h}^2 = o(\norm{h})$$

        \item $d_a \lambda(h) \cdot \alpha(h) \norm{h} = o(\norm{h})$,
        т.к. $\alpha(h) \to 0$ при $h \to 0$ по опр., $\norm{d_a \lambda(h)}
        \leqslant \norm{d_a \lambda} \cdot \norm{h}$ $\Rightarrow$
        $d_a \lambda(h)$ ограничена в окрестности 0. Аналогично с 
        $d_a f(h) \cdot \beta(h) \norm{h}$.

        \item $\alpha(h) \beta(h) \norm{h}^2 = o(\norm{h})$, т.к.
        $\alpha(h) \to 0$, $\beta(h) \to 0$, $\norm{h} \to 0$ при $h \to 0$.
    \end{enumerate} 
\end{proof}

\begin{theorem}[дифф. скалярного произведения]
    Пусть $f, g : E \to \R^m$, $E \subset \R^n$, $a \in \Int E$,
    $f$ и $g$ дифф. в точке $a$. Тогда $\langle f, g \rangle$
    дифф. в точке $a$ и $d_a \langle f, g \rangle (h) =
    \langle d_a f(h), g(a) \rangle + \langle f(a), d_a g(h) \rangle$
\end{theorem}
\begin{proof}
    \begin{gather*}
        \langle f, g \rangle = \sum_{k = 1}^m f_k g_k
        \Rightarrow 
        \underbrace{d_a \langle f, g \rangle = 
        \sum_{k = 1}^m d_a (f_k g_k)}
        _\text{линейность дифф.}; \\
        d_a \langle f, g \rangle (h) = 
        \sum_{k = 1}^m d_a (f_k g_k)(h) =
        \sum_{k = 1}^m 
        \underbrace{(d_a f_k (h) \cdot g_k(a) + 
        f_k (a) \cdot d_a g_k(h))}_{\text{пред. теорема}} = \\
        = \sum_{k = 1}^m 
        \underbrace{(d_a f(h))_k}_{\text{$k$-я коорд.}}
        \cdot g_k(a) + 
        \sum_{k = 1}^m f_k (a) \cdot (d_a g(h))_k =
        \langle d_a f(h), g(a) \rangle + \langle f(a), d_a g(h) \rangle
    \end{gather*}

    \textit{Необходимо пояснение, почему $d_a f_k (h) = (d_a f(h))_k$}
\end{proof}
\textbf{Замечание.} Если $n = 1$, то $\langle f(x), g(x) \rangle' =
\langle f'(x), g(x) \rangle + \langle f(x), g'(x) \rangle$
