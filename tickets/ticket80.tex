\section{Дифференцируемость обратного отображения. Образ области при невырожденном отображении}

Оказывается, что данное обратное отображение будет не только непрерывным, но и дифференерцируемым.
\begin{theorem} (о дифференцируемости обратного отображения) \\
    Пусть \begin{itemize}
        \item $f: X \to Y$  -- непрерывное отображение
        \item $f(a) = b$, $U$ -- окр-ть точки $a$, $V$ -- окр-ть точки $b$
        \item $f$ дифференерцируема в точке $a$, $\A = f'(a)$ обратимо
        \item $f^{-1}: V \to U$ существует и непрерывна
    \end{itemize}
    Тогда $g := f^{-1}$ дифференцируема в точке $b$.
\end{theorem}
\begin{proof}
    Распишем дифференцируемость в точке $a$: $f(a + h) = f(a) + \A h + \alpha(h)\| h \|$, где $\alpha(h) \to 0$ при $h \to 0$.
    Введем $k := f(a + h) - f(a) = \A h + \alpha(h)\| h \|$. 
    Заведем следующее неравенство: \[ \| h \| = \| \A^{-1}(\A h) \| \leqslant \| \A^{-1} \| \| \A h \| \Rightarrow \| \A h \| \geqslant \frac{\| h \|}{\| \A^{-1} \|} \]
    \quad Используем его при оценке нормы $k$: \[ \| k \| = \| \A h + \alpha(h)\| h \| \| \geqslant \frac{\| h \|}{\| \A^{-1} \|} - \| h \|\|\alpha(h) \| = \| h \| \underbrace{\left(\frac{1}{\| \A^{-1} \|} - \| \alpha(h) \| \right)}_{=:\, C \, > \, 0} \]
    \quad Если $k \to 0$, то $\| h \| \left(\frac{1}{\| \A^{-1} \|} - \| \alpha(h) \| \right) \to 0$, но скобка не будет стремиться к 0, так как $\frac{1}{\| \A^{-1} \|}$ это какая-то константа, поэтому $h \to 0 \text{ т.к. $\alpha(h) \to 0$ }$.
    Вспомним, что $k = f(a+h) - f(a)$, а $g = f^{-1}$. 
    \quad Тогда: \begin{gather*}
        g(\underbrace{b + k}_{f(a+h)}) - g(\underbrace{b}_{f(a)}) = a + h - a = h = \oast 
    \end{gather*}
    \quad Чтобы выразить $h$ через $k$, применим $\A^{-1}$ к равентсву $k = \A h + \alpha(h)\| h \|$ : \begin{gather*}
        \oast = \A^{-1}k - \A^{-1}(\alpha(h) \| h \|) \\
        \Rightarrow g(b + k) = g(b) + \A^{-1}k - \A^{-1}(\alpha(h) \| h \|)
    \end{gather*}
    \quad Осталось понять, что $\| \A^{-1}(\alpha(h) \| h \| \| = o(\|k\|)$:
    \[ \| \A^{-1}(\alpha(h) \| h \| \| \leqslant \| \A^{-1} \| * \|\alpha(h) \| * \| h \| \leqslant \| \A^{-1} \| * \underbrace{\|\alpha(h) \|}_{\to 0} * \frac{\|k\|}{C}\  \]
\end{proof}

\begin{follow}
    В теореме об обратной функции $f^{-1}$ непрерывно дифференцируема в точке $b$.
\end{follow}
\begin{proof}
    Мы поняли, что если обратная функция существует и $f'$ обратима в точке $a$, то обратная функция дифференерцируема в точке $b = f(a)$.
    Это позволяет нам понять, что обратная функция дифференцируема во всех точках, на которых определена (мы обозначали это множество за $V$).
    Действительно, в док-ве теоремы об обратной функции мы выбирали окр-ть -- $B_r(a)$(там мы использовали $B_r(x_0)$) -- именно так, чтобы $f'$ было там обратимо.
    
    \quad Осталось понять про непрерывность. 
    Введем классическое обозначение матрицы Якоби: $J_f(a)$ -- матрица Якоби $f$ в точке $a$, $J_{f^{-1}}(b)$ -- матрица Якоби $f^{-1}$ в точке $b$.
    Тогда можно применить формулу дифференцирования композиции:
    \begin{gather*}
        f^{-1} \circ f = \id \Rightarrow J_{f^{-1} \circ f}(a) = E \\
        J_{f^{-1} \circ f}(a) = J_{f^{-1}}(b) \cdot J_f(a) = E \Rightarrow J_{f^{-1}}(b) = (J_f(a))^{-1} 
    \end{gather*}
    \quad Таким образом, матрица Якоби $f^{-1}$ в точке $b$ -- это обратная к матрице Якоби функции $f$ в точке $a$.
    Мы можем посчитать обратную матрицу с помощью формулы с минорами.
    Тогда мы будем производить разные арифметические операции с частными производными (ведь именно они составляют матрицу Якоби для $f$), а они непрерывны, ведь $f$ непрерывно дифференерцируема по условию.
    В итоге, $J_{f^{-1}}(b)$ будет состоять из различных комбинаций непрерывных функций, то есть частные производные будут непрерывны, а значит  $f^{-1}$ будет непрерывно дифференерцируема в точке $b$.
\end{proof}

\begin{notice}
    Зачастую именно этот вывод называют теоремой об обратной функции.
    Мы же для удобства разбили ее на 3 части.
\end{notice}

\vspace*{7mm}

\begin{follow}
    Пусть $f: \underbrace{D}_{\subset \R^n} \to \R^n$ непрерывно дифференерцируема в $D$ и $f'(x)$ обратимо $\forall x \in D$.
    Тогда для любого открытого $G \subset D$ множество $f(G)$ тоже будет открытым.
\end{follow}

\begin{proof}
    \textit{Fun fact: У нас когда-то было теорема о том, что прообраз открытого мн-ва это всегда открытое мн-во,
    а вот образ открытого это открытое -- явление довольно редкое.} 
    Перейдем к доказательству.

    \quad Зафиксируем произвольное $b \in f(G)$. 
    Надо док-ть, что $b$ -- внутренняя точка. 
    Она лежит в образе $\Rightarrow \exists a : b = f(a)$.
    Применим теорему об обратной функции: $\exists U$ -- окр-ть $a$ и $\exists V$ -- окр-ть $b$, т.ч. $f: U \to V$ -- биекция.
    Заметим, что мы можем выбрать такое $U$, что $U \subset G$. 
    Действительно, надо просто уменьшить нашу окр-ть, чтобы она попала в $G$. 
    Тогда очевидно $f(U) = V \subset f(G)$.
    Получается, что $b \in V \subset f(G)$. 
    Итого, $b$ лежит в $f(G)$ с какой-то окрестностью, следовательно, является внутренней.
\end{proof}