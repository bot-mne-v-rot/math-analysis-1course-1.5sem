\section{Произведение $\prod \frac{p_n}{p_n - 1}$ и ряд $\sum \frac{1}{p_n}$. Оценка частичных сумм ряда $\sum \frac{1}{p_n}$}
\textbf{Пример к предыдущей теореме:}

Пусть $p_n$ -- $n$-e простое число. Тогда $\prodn \frac{p_n}{p_n - 1}$ расходится.
Более того, $\prodk \frac{p_k}{p_k - 1} \geq H_n$.

\begin{proof}
    Посчитав сумму геометрической прогрессии получаем:
    \begin{gather*}
        \frac{p}{p-1} = \frac{1}{1-\frac{1}{p}} = \sum \limits_{j=0}^\infty \frac{1}{p^j}
    \end{gather*}
    То есть искомое произведение - это:
    \begin{gather*}
        \prodn \sum \limits_{j=0}^\infty \frac{1}{p^j}
    \end{gather*}
    Предположим, что мы умеем раскрывать скобки, тогда после раскрытия скобок получится сумма обратных ко всем числам:
    \begin{gather*}
        \prodn \sum \limits_{j=0}^\infty \frac{1}{p^j} = \sum \frac{1}{n}
    \end{gather*}
    Действительно, каждое число единственным образом раскладывается на простые множители, и каждый набор простых множителей дает какое-то уникальное число.

    Это было интуитивное рассуждение. Теперь формализуем его для конечных сумм:

    \[\sum \limits_{j=0}^\infty \frac{1}{p^j} \geq \sum \limits_{j=0}^n \frac{1}{p^j}\]
    \[P_n = \prodk \frac{p_k}{p_k - 1} \geq \prodk \sum \limits_{j=0}^n \frac{1}{p^j} = 
    \sum_{0 \leq \alpha_j \leq n} \frac{1}{p_1^{\alpha_1}p_2^{\alpha_2}\ldots p_n^{\alpha_n}} \geq \sumk \frac{1}{k} \]

    Тогда в знаменателе встретятся все числа от $1$ до $n$. Действительно, любое число до $n$ раскладывается на простые,
    не большие $n$, а также со степенями не больше $n$ (даже $2^n$ уже больше $n$). Видно, что это очень грубая оценка, например можно
    показать, что встретятся также и все числа до $p_n$.
\end{proof}

\begin{theorem}
    Ряд $\sumn \frac{1}{p_n}$ расходится. $p_n$ -- $n$-е простое число.
\end{theorem}

\begin{proof}
    $\prodn \frac{p_n}{p_n - 1}$ расходится $\Longrightarrow \sumn \ln(\frac{p_n}{p_n-1})$ расходится
    \begin{gather*}
        \ln(\frac{p}{p-1}) = \ln(\frac{1}{1-\frac{1}{p}}) = -\ln(1 - \frac{1}{p}) = \frac{1}{p} + O(\frac{1}{p^2})
    \end{gather*}
    То есть $\sumn \frac{1}{p_n} + O(\frac{1}{p_n^2})$ -- расходится. При этом знаем, что $\sumn O(\frac{1}{p_n^2})$ -- сходится.
    Значит их разность расходится, а значит $\sumn \frac{1}{p_n}$ -- расходится
\end{proof}

\textbf{Замечания.} 
\begin{enumerate}
    \item \begin{gather*}
        \sumk \frac{1}{p_k} \geq \ln H_n + O(1) \geq \ln \ln n + C
    \end{gather*}
    \begin{proof}
        \begin{align*}
            \sumk \frac{1}{p_k} &= \sumk (-\ln(1-\frac{1}{p_k}) + O(\frac{1}{p_k^2})) \\
            &= \sumk (-\ln(1-\frac{1}{p_k})) + O(1) = \\
            &= \ln \prodk \frac{p_k}{p_k-1} + O(1) \geqslant \ln H_n + O(1) \geqslant \ln \ln n + C
        \end{align*}
    \end{proof}
    \item На самом деле $\sumn \frac{1}{p_k} = \ln \ln n + O(1)$
    \item Докажем чуть менее точную оценку сверху:
    \begin{gather*}
        \sumk \frac{1}{p_k} \leq 2\ln \ln n
    \end{gather*}
    \begin{proof}
        Покажем, что:
        \begin{gather*}
            \sum \limits_{a \leq p \leq a^2} \frac{1}{p} < \frac{4}{3}
        \end{gather*}
        Для этого выпишем числа $1, 2, 3, \ldots, a^2$. Зачеркнём делящиеся на $p$ для $a \leq p \leq a^2$.
        Заметим, что каждое число мы вычеркнем не более одного раза, так как произведение любых двух простых
        такого вида больше $a^2$. При этом для каждого $p$ мы зачеркиваем $\lfloor \frac{a^2}{p} \rfloor$ чисел, делящихся на $p$.
        Но мы зачеркнули чисел всего не больше, чем у нас есть. Оценивая снизу округление вниз, получаем

        \begin{gather*}
            \sum_{a \leq p \leq a^2}(\frac{a^2}{p} - 1) \leqslant \sum_{a \leq p \leq a^2} \lfloor \frac{a^2}{p} \rfloor < a^2
            \Longrightarrow \sum_{a \leq p \leq a^2} \frac{a^2}{p} < a^2 + \text{кол-во простых} < \frac{4}{3}a^2 \
        \end{gather*}

        Делим на $a^2$, получаем искомое неравенство. Использовали, что количество простых не больше чем $\frac{1}{3}a^2$ потому например
        все простые кроме 2, 3 нечетны и не делятся на $3$, а таких уже треть.

        Рассмотрим
        \begin{gather*}
            \sum_{p \leq 2^{2^n}} \frac{1}{p} = \sumk \sum_{2^{2^{k-1}} \leq p \leq 2^{2^k}} \frac{1}{p}
            \leq \sumk \frac{4}{3} = \frac{4}{3}n
        \end{gather*}
        Тогда считая сумму обратных простых до некоторого $n$, возьмем $m = \lceil \log_2 \log_2 n \rceil$. Тогда

        \begin{align*}
            \sum_{j=1}^n \frac{1}{p_j} \leqslant \sum_{p \leqslant 2^{2^m}} \frac{1}{p} < \frac{4}{3}\lceil \log_2 \log_2 n \rceil
            &\leqslant \frac{4}{3} (\log_2 \log_2 n + 1) \\
            &= \frac{4}{3}(\log_2 \ln n + C) \\
            &= \frac{4}{3} \cdot \frac{\ln \ln n}{\ln 2} + C \leq 2 \ln \ln n + C
        \end{align*}

        Легко видеть, что константа контролируется: она не больше $\frac{4}{3} (1 - \log_2 \ln 2) \approx 2.03$
    \end{proof}   


\end{enumerate}