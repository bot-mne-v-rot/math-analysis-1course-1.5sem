\section{Линейные операторы. Свойства. Операции с линейными операторами. Матричное задание линейных операторов из $\R^n$ в $\R^m$}
\begin{conj}
    $X$ и $Y$ -- векторные пространства, $\A: X \longrightarrow Y$ -- \textbf{линейный оператор}, если: 
    \begin{gather*}
        \A(\lambda x + \mu y) = \lambda \A (x) + \mu \A (y) \qquad \forall x, y \in X \qquad \forall \lambda, \mu \in \R \text{, ну или } \C
    \end{gather*}
\end{conj}
\textit{\textbf{Свойства: }}
\begin{enumerate}
    \item $\A(0_X) = 0_Y$
    \item $\A(\sum\limits_{k=1}^n \lambda_k x_k) = \sum\limits_{k=1}^n \lambda_k \A(x_k)$
\end{enumerate}
\begin{proof} \quad

    \begin{enumerate}
        \item $\lambda = \mu = 0 \qquad \A(0_x) = 0 + \dots + 0 = 0_Y$
        \item Индукция
    \end{enumerate}
\end{proof}
\begin{conj}
    $\A, \B$ -- линейные операторы, $\lambda \in \R$. 
    Тогда введем для них сумму и умножение на скаляр, которые кстати тоже линейные операторы:
    \begin{align*}
        \A + \B &: X \longrightarrow Y \quad (\A + \B)x := \A x + \B x \\
        \lambda \A&: X \longrightarrow Y \quad (\lambda \A)(x) := \lambda \cdot \A x
    \end{align*}
\end{conj}
\notice \; Пространство линейных операторов из $X$ в $Y$ -- векторное пространство над тем же полем.
\begin{conj}
    Пусть есть $\A: X \longrightarrow Y$ и $\B: Y \longrightarrow Z$ -- линейные операторы. 
    Тогда мы можем ввести композицию, которая тоже будет линейным оператором:
    \begin{gather*}
        \B \circ \A : X \longrightarrow Z \quad (\B \circ \A)(x) := \B(\A x) 
    \end{gather*}
\end{conj}
\begin{conj}
    Для $\A : X \longrightarrow Y$, оператор $\A^{-1} : Y \longrightarrow X$ -- обратный оператор, если выполняются два уловия:
    \begin{enumerate}
        \item $\A \circ \A^{-1} = \operatorname{Id}_{Y}$
        \item $\A^{-1} \circ \A = \operatorname{Id}_{X}$
    \end{enumerate}
\end{conj}
\textit{\textbf{Свойства: }}
\begin{enumerate}
    \item Если обратимый оператор существует, то он единственнен. 
    \item $(\lambda \A)^{-1} = \frac{1}{\lambda} \cdot \A^{-1}$
    \item Если $X = Y$, то множество обратимых операторов -- группа по отношению композиции. 
\end{enumerate}
\begin{proof} \quad

    \begin{enumerate}
        \item От обратного. Пусть $\B_1 \circ \A = \operatorname{Id}_X$ и $\B_2 \circ \A = \operatorname{Id}_X$. Возьмем $x \in X$. Тогда:
        \begin{gather*}
            \B_1(\A x) = \B_2(\A x) = x
        \end{gather*}
        Значит на любом векторе $y$, который является образом какого-либо вектора $\B_1 y = \B_2 y$. Значит осталось понять, что каждый $y$ -- чей-то образ. А это легко:
        \begin{gather*}
            \A \circ \B_1 = \operatorname{Id}_Y \Longrightarrow \A(\B_1 y) = y
        \end{gather*}
        \item \begin{gather*}
            \left( \frac{1}{\lambda} \A^{-1}\right) \circ (\lambda \A)(x) = \frac{1}{\lambda} \A^{-1} (\lambda \A(x)) = \A^{-1} (\A x) = x
        \end{gather*}
        Аналогично проверяется в другом порядке. 
        \item Аксиомы группы тривиально проверяются. 
    \end{enumerate}
\end{proof}

\vspace*{5mm}

\textbf{Матричная запись оператора}. Пусть $X = \R^n, Y = \R^m$. Также $e_1, \dots, e_n$ -- стандартный базис в $\R^n$, то есть:
\begin{gather*}
    \begin{pmatrix*}
        0 \\ 
        \vdots \\
        0 \\
        1 \\
        0 \\
        \vdots \\
        0
    \end{pmatrix*} \longleftarrow i\text{-ое место}
\end{gather*}
Тогда $x \in X$ можно представить как сумму $i$-ой координаты, умноженной на $i$-ый базисный вектор. 
Также введем обозначение $\A_i$, для применения $\A$ к базисному вектору. 
\begin{align*}
    x &= \sum\limits_{i=1}^n x_i e_i = \begin{pmatrix*}
        x_1 \\
        x_2 \\
        \vdots \\
        x_n
    \end{pmatrix*} & \A_i &= \A e_i = \begin{pmatrix*}
        a_{1i} \\
        a_{2i} \\
        \vdots \\
        a_{mi}
    \end{pmatrix*}
\end{align*}
Тогда линейный оператор $\A$, примененный к иксу -- это:
\begin{gather*}
    \A x = \A \left( \sum\limits_{i=1}^n x_i e_i \right) = \sum\limits_{i=1}^n x_i \A e_i = \sum\limits_{i=1}^n x_i \A_i = \sum\limits_{i=1}^n x_i \cdot \begin{pmatrix*} 
        a_{1i} \\
        a_{2i} \\
        \vdots \\
        a_{mi}
    \end{pmatrix*} =
    \begin{pmatrix*} 
        a_{11} & a_{12} & \cdots & a_{1n} \\
        a_{21} & a_{22} & \cdots & a_{2n} \\
        \vdots & \vdots & \ddots & \vdots \\
        a_{m1} & a_{m2} & \cdots & a_{mn} 
    \end{pmatrix*} \cdot \begin{pmatrix*}
        x_1 \\
        x_2 \\
        \vdots \\
        x_n
    \end{pmatrix*}
\end{gather*}