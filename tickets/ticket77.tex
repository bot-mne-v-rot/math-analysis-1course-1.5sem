\section{Метод касательных для решения уравнения}

\underline{\textit{Наглядный пример использования теоремы Банаха о сжатии:}}

\textbf{Метод касательных (метод Ньютона).}

Пусть $f \in C^2[a, x_0]$. Хотим предъявить хороший способ искать решения уравнения $f(x) = 0$.
От функции требуем выполнения следующих условий: $f(a) = 0$, $f$ строго монотонна и строго выпукла. А также: $f'(a) = \mu > 0$.
Хотим найти значение $a$.

Сам метод заключается в том, что мы стартуем из точки $x_0$ и итерируемся следующим образом, пока не выполнится необходимое условие:
\begin{gather*}
    x_{n+1} = x_n - \frac{f(x_n)}{f'(x_n)}
\end{gather*}
В качестве необходимого условия можно взять $\abs{x_{n+1} - x_n} < \varepsilon$ или $\abs{f(x_{n+1})} < \varepsilon$. 
\begin{proof}
    Зададим отображение: 
    \begin{gather*}
        g(x) = x - \frac{f(x)}{f'(x)}
    \end{gather*}
    Мы зафиксировали, что значение первой производной $f$ в точке $a$ положительно, также мы знаем, что $f$ строго выпукла, значит первая производная 
    растет, а значит в ноль она не обратится. Значит отображение $g$ задано корректно. 

    Чтобы найти корень $f$, давайте найдем неподвижную точку $g$. Это нам поможет, так как $g(x) = x \Longleftrightarrow f(x) = 0$. Чтобы понять, что мы можем быстро это сделать, 
    нужно проверить, что $g: [a, x_0] \longrightarrow [a, x_0]$ и является сжатием.

    Сначала поймем, что у $g$ нужная область значений. 
    Очевидно, что $g(x) \leqslant x$. А $x \leqslant x_0$ при $x \in [a, x_0]$. 
    То есть за правую границу функция не перескочит. Осталось проверить левую границу. То есть хотим проверить, что при $x \in [a, x_0]$ выполняется:
    \begin{gather*}
        x - \frac{f(x)}{f'(x)} \geqslant a
    \end{gather*}
    Про $f(x)$ мы по теореме Лагранжа знаем, что:
    \begin{gather*}
        f(x) - \stackbelow{f(a)}{0} = f'(\xi)\cdot(x - a)
    \end{gather*}
    Теперь мы хотим как то оценить это сверху. 
    Мы могли бы оценить это как $f'(x_0)(x - a)$, но так как мы можем рассматривать 
    нашу функцию только на отрезке $[a, x]$, то можно сказать, что максимальное значение для $\xi$ -- это $x$, и тогда: 
    \begin{gather*}
        f(x) = f'(\xi)\cdot(x - a) \leqslant f'(x)(x-a)
    \end{gather*}
    Вспомним, что мы хотели проверить, что функция не выскакивает налево за границу $a$. 
    Минимизируем ее значение, подставив то, что мы сейчас получили:
    \begin{gather*}
        x - \frac{f(x)}{f'(x)} \geqslant x - \frac{\cancel{f'(x)}(x - a)}{\cancel{f'(x)}} = a
    \end{gather*}
    Получили верное равенство, значит $g$ переводит отрезок в отрезок. 
    Теперь мы хотим, чтобы $g$ была сжатием. А как нам проверить, что это сжатие? 
    Для сжатия мы хотим как-то сверху оценить расстояние между образами функции. 
    Это как раз умеет теорема Лагранжа. Но она нам дает какую-то промежуточную производную, 
    а мы хотим $\lambda$ из интервала $(0, 1)$. Ну так давайте проверим, что производная во 
    всех точках будет сверху ограничена какой-то константой < 1. 
    \begin{align*}
        g'(x) &= 1 - \frac{f'(x) \cdot f'(x) - f''(x)f(x)}{(f'(x))^2} \\
        &= \cancel{1} - \cancel{\frac{(f'(x))^2}{(f'(x))^2}} + \frac{f''(x)f(x)}{(f'(x))^2} = \frac{f''(x)f(x)}{(f'(x))^2}
    \end{align*}
    Пусть $M:= \max\limits_{t \in [a, x_0]} f''(t)$. Оценим $f(x)$ с помощью Лагранжа и произведем несколько несложных оценОчек:
    \begin{gather*}
        \frac{f''(x)f(x)}{(f'(x))^2} \leqslant \frac{M \cancel{f'(x)}(x-a)}{(f'(x))^{\cancel{2}}} \leqslant \frac{M}{\mu} (x_0 - a) =: \lambda 
    \end{gather*}
    Это какая-то константа и мы хотим, чтобы она была меньше единицы. Тут мы уже ничего толком сделать не можем и 
    остается сказать, что при выполнении данного условия $g$ -- сжатие и все хорошо, иначе -- сходимость к корню есть лишь в некоторой его окрестности. Заметим, что 
    $M$ и $\mu$ -- константы, а $(x_0 - a)$ может быть сколь угодно малым, то есть мы можем ручками сделать так, чтобы $\lambda$ была меньше 1. Нужно лишь правильно выбрать 
    стартовое положение, то есть $x_0$. Можно сделать это методом деления пополам, а потом продолжить методом Ньютона искать корень уже с гораздо большей скоростью 
    (почему она будет большой мы пока не понимаем, понимаем только что она будет не хуже).
    
    Не хуже она будет потому, что, когда $\lambda < 1$, то $x_n \longrightarrow a$, причем $\abs{x_n - a} \leqslant \lambda^n (x_0 - a)$, что значит, что у нас есть 
    степенная скорость приближения к нужной точке. 
\end{proof} 
Приведем иллюстрацию к методу, чтобы понять, что происходит с геометрической точки зрения: 
\begin{center}
    \begin{tikzpicture}[thick,yscale=0.8]
        % Axes
        \draw[-latex,name path=xaxis] (-1,0) -- (10,0) node[above]{\large $x$};
        \draw[-latex] (0,-2) -- (0,8)node[right]{\large $y$};;
        
        % Function plot
        \draw[ultra thick, orange,name path=function]  plot[smooth,domain=-1:7] (\x, {-2-10/(\x-8)}) node[left]{$y = f(x)$};
        
        % plot tangent line
        \node[violet,right=0.2cm] at (6.55,4.9) {$(x_0, f(x_0))$};
        \draw[gray,thin,dotted] (6.55,0) -- (6.55,4.9) node[circle,fill,inner sep=2pt]{};
        \draw[dashed, violet,name path=Tfunction]  plot[smooth,domain=5.38:7.25] (\x, {5*\x-28});

        \node[violet,left=0.2cm] at (5.6,2.24) {$(x_1, f(x_1))$};
        \draw[dashed, violet,name path=Rfunction]  plot[smooth,domain=3.5:6] (\x, {1.6*\x-6.83});
        \draw [name intersections={of=Tfunction and xaxis}, gray,thin,dotted] ($(intersection-1)-(0,0.1)$) -- ++(0,2.3) node[circle,fill,inner sep=2pt]{};

        \node[violet,left=0.2cm] at (4.2,1) {$(x_2, f(x_2))$};
        \draw[dashed, violet,name path=Kfunction]  plot[smooth,domain=2.8:4.8] (\x, {0.7*\x-2.35});
        \draw [name intersections={of=Rfunction and xaxis}, gray,thin,dotted] ($(intersection-1)-(0,0.1)$) -- ++(0,0.76) node[circle,fill,inner sep=2pt]{};

        % x-axis labels
        \draw (6.55,0.1) -- (6.55,-0.1) node[below] {$x_0$};
        \draw [name intersections={of=Tfunction and xaxis}] ($(intersection-1)+(0,0.1)$) -- ++(0,-0.2) node[below,fill=white] {$x_1$} ;
        \draw [name intersections={of=Rfunction and xaxis}] ($(intersection-1)+(0,0.1)$) -- ++(0,-0.2) node[below,fill=white] {$x_2$} ;
        \draw [name intersections={of=Kfunction and xaxis}] ($(intersection-1)+(0,0.1)$) -- ++(0,-0.2) node[below] {$x_3$} ;

        %draw "a"
        \node [name intersections={of=function and xaxis}] at ($(intersection-1)+(0,0.35)$) {$a$} ;

        \node[violet] at (6, -1.5) {$y = f(x_0) + f'(x_0)(x-x_0)$} ;
    \end{tikzpicture}
\end{center}
Уравнение самой правой касательной -- это $y = f(x_0) + f'(x_0)(x-x_0)$. Точка $x_1$ задается уравнением:
\begin{gather*}
    f(x_0) + f'(x_0)(x-x_0) = 0 \\
    x_1 = x_0 - \frac{f(x_0)}{f'(x_0)} = g(x_0) 
\end{gather*}
То есть находится рекурсивным соотношением, которое мы ввели, описывая метод. Таким образом, геометрический смысл наших действий следующий: 
проводим касательную в точке, смотрим на точку пересечения касательной и оси $OX$, берем ее за новую точку и повторяем действия. 
Нетрудно убедиться, что:
\begin{gather*}
    x_{n+1} - a \leqslant \underbrace{\frac{M}{2\mu}}_{=:\alpha} (x_n - a)^2
\end{gather*}
Тогда получаем, что: 
\begin{gather*}
    \alpha(x_{n+1} - a) \leqslant \alpha^2 (x_n - a)^2 = (\alpha(x_n - a))^2 \\
    \alpha(x_n - a) \leqslant (\alpha (x_0 - a))^{2^n} \\
    x_n - a \leqslant \frac{1}{\alpha} (\alpha(x_0 - a))^{2^n}
\end{gather*}
Отсюда видно, что, итерируясь методом Ньютона, мы будем крайне быстро приближаться к $a$, 
что как раз и доказывает утверждение, что этот метод быстрее метода деление пополам, которое звучало выше.
\begin{center}
    \underline{Лирическое отступление на тему ``Как компьютер корни считает'':}
\end{center}
\vspace*{0.25cm}
Мы хотим найти решение уравнения $f(x) = x^k - b$. Воспользуемся методом Ньютона: 
\begin{gather*}
    x_{n+1} = x_n - \frac{x_n^k - b}{k x_n^{k-1}} = x_n \left(1 - \frac{1}{k} \right) + \frac{b}{k x_n^{k-1}}
\end{gather*}
Получаем итеративный и быстрый способ посчитать корень $k$-ой степени из числа. При $k=2$ всё вообще песня сказка и получаем: 
\begin{gather*}
    x_{n+1} = \frac{1}{2}\left(x_n + \frac{b}{x_n}\right)
\end{gather*}
