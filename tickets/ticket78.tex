\section{Оценки на норму обратного отображения и на норму разности значений дифференцируемого отображения. Теорема об обратимости отображений, близких к обратимым}

\begin{theorem}
    Если есть линейный оператор $\A: \R^n \longrightarrow \R^n$, такой, что $\norm{\A x} \geqslant m \norm{x} \quad \forall x \in X$, при 
    некотором $m > 0$, то $\A$ обратим и $\norm{\A^{-1}} \leqslant \frac{1}{m}$.
\end{theorem}
\begin{proof}
    Для обратимости нужна биективность. Сюръективность очевидна, так как размерность сохраняется, так что осталось проверить инъективность:
    \begin{align*}
        \A x = \A y 
        &\Longrightarrow \A(x - y) = 0 \\
        &\Longrightarrow \stackbelow{\norm{\A(x-y)}}{0} \geqslant m \norm{x-y} \\
        &\Longrightarrow \norm{x - y} = 0 \\
        &\Longrightarrow x = y
    \end{align*}
    Значит $\A$ -- обратим. Тогда:
    \begin{gather*}
        \norm{A^{-1}} = \sup\limits_{y \neq 0} \frac{\norm{\A^{-1}y}}{\norm{y}} \xlongequal{y = \A x} 
        \sup\limits_{x \neq 0} \frac{\norm{\A^{-1}(\A x)}}{\norm{\A x}} = 
        \sup\limits_{x \neq 0} \frac{\norm{x}}{\norm{\A x}} \leqslant 
        \sup\limits_{x \neq 0} \frac{\cancel{\norm{x}}}{m \cancel{\norm{x}}} = \frac{1}{m}
    \end{gather*}
\end{proof}



Докажем еще пару теорем необходимых для доказательства теоремы об обратной функции.
\begin{theorem}
    Пусть $f: \R^n \to \R^m$ -- функция, дифференцируемая в шаре $B_r(a)$, и $\forall x \in B_r(a)$ норма $\| f'(x) \| \leqslant C$. 
    Тогда $\forall x, y \in B_r(a)$ выполняется $\| f(x) - f(y) \| \leqslant C \| x - y \|$.
\end{theorem}
\begin{proof}
    Введем $\varphi:[0, 1] \to \R \;\; \varphi(t) = \langle f(x + t(y - x)), f(y) - f(x) \rangle$.
    Она дифференцируема, так как $f(x + t(y - x))$ дифференцируема, ведь отрезок $[x, y] \subset B_r(a)$, $f(y) - f(x)$ -- константный вектор, и скалярное произведение дифференцируемо.

    \quad Согласно одномерной теореме Лагранжа: $\varphi(1) - \varphi(0) = \varphi'(\theta)$, где $\theta \in (0, 1)$. 
    Распишем по формуле дифференцирования скалярного произведения: \begin{gather*}
        \begin{split}
            \varphi'(\theta) &= \langle f'(x + \theta(y - x))\cdot(y - x), f(y) - f(x) \rangle + \underbrace{\langle f(x + \theta(y - x)), 0 \rangle}_0  \\
            &\overset{\text{КБШ}}{\leqslant} \| f'(x + \theta(y - x))\cdot(y - x) \| \cdot \| f(y) - f(x) \| \\
            &\leqslant \| f'(x + \theta(y - x))\| \cdot \| y - x \| \cdot \| f(y) - f(x) \| \\
            &\leqslant C\| y - x \| \cdot \| f(y) - f(x) \|
        \end{split}
    \end{gather*}
    \quad Распишем $\varphi(1) - \varphi(0)$ по определению: \begin{gather*}
        \varphi(1) - \varphi(0) = \langle f(y), f(y) - f(x) \rangle - \langle f(x), f(y) - f(x) \rangle = \\
        = \langle f(y), f(y) \rangle - 2 \langle f(y), f(x) \rangle +  \langle f(x), f(x) \rangle = \| f(y) - f(x) \|^2 \\ \\
        \Rightarrow \| f(y) - f(x) \|^2 \leqslant C \| y - x \| \cdot \| f(y) - f(x) \| \\
        \Rightarrow \| f(y) - f(x) \| \leqslant C \| y - x \|
    \end{gather*} 
\end{proof}

\begin{theorem} (об обратимости оператора близкого к обратимому) \\
    Пусть $\A: \R^n \to \R^n$ -- линейный обратимый оператор, $\B: \R^n \to \R^n$ -- просто линейный оператор, и выполняется $\| \B - \A \| < \frac{1}{\| \A^{-1} \|}$ (они достаточно близки).
    Тогда $\B$ обратим, \\ $\| \B^{-1} \| \leqslant \frac{1}{\frac{1}{\| \A^{-1} \|} - \| \B - \A \|}$ и обратные также достаточно близки $\| \B^{-1} - \A^{-1} \| \leqslant \frac{\| \A^{-1} \| \cdot \| \B - \A \|}{\frac{1}{\| \A^{-1} \|} - \| \B - \A \|}$.
\end{theorem}
\begin{proof}
    Напишем неравенство треугольника: \begin{gather*}
        \| \A x \| = \| (\A - \B)x + \B x \| \leqslant \| (\A - \B)x \| + \| \B x \| \\
        \Rightarrow \| \B x \| \geqslant \| \A x \| - \| (\A - \B)x \|
    \end{gather*}
    \quad Заметим, что по стандартному неравенству $\| (\A - \B)x \| \leqslant \| \A - \B \| \| x \|$, а также $\| \A^{-1}(\A x) \| \leqslant \| \A^{-1} \| \| \A x \| \Rightarrow \| \A x \| \geqslant \frac{\| x \|}{\| \A^{-1} \| }$. 
    Подставим все это в неравенство: \begin{gather*}
        \| \B x \| \geqslant  \frac{\| x \|}{\| \A^{-1} \|} - \| \A - \B \| \| x \| = \underbrace{\left(\frac{1}{\| \A^{-1} \|} - \| \A - \B \| \right)}_{=: m} \| x \|
    \end{gather*}
    \quad Тогда по предпредыдущей теореме $\B$ обратим и $\| \B^{-1} \| \leqslant \frac{1}{m} = \frac{1}{\frac{1}{\| \A^{-1} \|} - \| \B - \A \|}$.
    Осталось только неравенство на норму разности: 
    \begin{align*} 
        \| \B^{-1} - \A^{-1} \| &= \| \B^{-1}(\A - \B)\A^{-1} \| \\
        &\leqslant \| \B^{-1}\| \| \A - \B \| \| \A^{-1} \| \\
        &\leqslant \frac{1}{m}\| \A - \B \| \| \A^{-1} \| \\
        &= \frac{\| \A^{-1} \| \cdot \| \B - \A \|}{\frac{1}{\| \A^{-1} \|} - \| \B - \A \|}
    \end{align*}
\end{proof}