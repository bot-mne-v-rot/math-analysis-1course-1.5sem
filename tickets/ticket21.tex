\section{Путь, носитель пути, простой путь, гладкий путь. Эквивалентные пути. Определение кривой}

\begin{conj}
    Пусть $X$ --- метрическое пространство. Тогда
    $\gamma\colon [a, b] \to X$ --- путь, если $\gamma$ непрерывна. При этом $\gamma(a)$ называется началом пути, а $\gamma(b)$ концом пути.
  \end{conj}
  
  \begin{conj}
    Путь называется замкнутым, если $\gamma(a) = \gamma(b)$
  \end{conj}
  
  \begin{conj}
    Путь называется простым(несамопересекающимся), если  $\gamma(t) \neq \gamma(u),\; \forall t, u \in [a, b]$ за исключением, возможно $\gamma(a) = \gamma(b)$.
  \end{conj}
  
  \begin{conj}
    Противоположным к $\gamma$ путем называется путь $\gamma^{-1}\colon [a, b] \to X$ такой, что $\gamma^{-1}(t) = \gamma(a + b - t)$. В частности $\gamma^{-1}(a) = \gamma(b)$ и $\gamma^{-1}(b) = \gamma(a)$.
  \end{conj}
  
  \begin{conj}
    Пути $\gamma\colon[a, b] \to X$ и $\widetilde{\gamma}\colon[c, d] \to X$ называются эквивалентными, если существует $u\colon [a, b] \to [c, d]$ --- непрерывная и строго монотонно возрастающая функция. При этом $u(a) = c,\, u(b) = d$ и $\gamma = \widetilde{\gamma} \circ u$.
    Такая $u$ называется допустимым преобразованием параметра.
  \end{conj}

  То есть, так как пути -- это у нас функция, как бы, переводящая время в координаты, то эквивалентные пути -- это пути, которые на плоскости выглядят
  одинаково, но, которые, быть может, пройдены за разные промежутки ``времени''.

  \begin{notice}
    Это отношение эквивалентности.
  \end{notice}
  
  \begin{conj}
    Кривая --- класс эквивалентных путей. Параметризация кривой --- конкретный представитель класса.
  \end{conj}
  
  \begin{conj}
    Носитель пути $\gamma$ --- множество $\gamma([a, b])$.
  \end{conj}

  Множество точек 
  
  \begin{notice}
    У эквивалентных путей одинаковые носители.
  \end{notice}
  
  \begin{conj}
    Пусть $\gamma\colon [a, b] \to \R^d$. Тогда $\gamma$ --- r-гладкий путь, если $\gamma_j \in C^r[a, b]$ при $j = 1, 2, \dotsc, d$.
    Гладкая кривая --- кривая, у которой есть гладкая параметризация.
  \end{conj}
  