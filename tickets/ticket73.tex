\section{Теорема о равенстве частных производных для непрерывно дифференцируемых функций. Пример, показывающий необходимость непрерывности производных }

\begin{conj}
    $f : D \to \R$, $D \subset \R^n$, $D$ -- открытое множество. \\
    $f$ \textbf{$\mathbf{r}$ раз непрерывно дифференцируема} 
    ($f$ --- \textbf{$\mathbf{r}$-гладкая} функция), если 
    все частные производные до $r$-го порядка существуют и
    непрерывны.

    Обозначение: $f \in C^r(D)$.
\end{conj}

\begin{theorem}
    Пусть $f \in C^r(D)$. $(i_1, \dots, i_n)$ --- перестановка 
    $(j_1, \dots, j_n)$. Тогда:
    $$ \frac{d^r f}{d x_{i_1} d x_{i_2} \dots d x_{i_r}} = 
    \frac{d^r f}{d x_{j_1} d x_{j_2} \dots d x_{j_r}} $$
\end{theorem}
\begin{proof} $ $

    Рассмотрим случай, когда $(i_1, \dots, i_r)$ --- транспозиция
    соседних $(j_1, \dots, j_r)$, т.е. 
    $$(i_1, \dots, i_r) = (j_1, \dots, j_{k - 1}, 
    \underbrace{j_{k + 1}, j_k}_{\text{переставили}}, 
    j_{k + 2}, \dots, j_r)$$

    Пусть:
    $$ g(x) := \frac{d^{r - k - 1} f}{d x_{j_{k + 2}} \dots d x_{j_r}}
    \quad \quad
    A := \frac{d^r f}{d x_{i_1} d x_{i_2} \dots d x_{i_r}} \quad
    \quad B := \frac{d^r f}{d x_{j_1} d x_{j_2} \dots d x_{j_r}} $$
    Тогда:
    $$ A = \frac{d^{k-1}}{d x_1 \dots d x_{k-1}} 
    \frac{d^{2}}{d x_{k+1} d x_{k}} g \quad \quad 
    B = \frac{d^{k-1}}{d x_1 \dots d x_{k-1}} 
    \frac{d^{2}}{d x_{k} d x_{k+1}} g$$
    По теореме о перестановке частных производных в $\R^2$:
    $$ \frac{d^{2}}{d x_{k+1} d x_{k}} g = 
    \frac{d^{2}}{d x_{k} d x_{k+1}} g $$
    А дальше мы навешиваем производные в одинаковом порядке.
    Поэтому $A = B$.

    Теперь воспользуемся фактом из линейной алгебры о том, что любая
    перестановка раскладывается в конечное произведение транспозиций
    соседних элементов. Действительно, если смена местами соседних
    элементов в перестановке не меняет производную, 
    то мы можем сделать конечное
    количество таких действий и производная всё равно не поменяется.
\end{proof}