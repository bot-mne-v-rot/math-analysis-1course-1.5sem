\section{Признак Абеля}
\textbf{Признак Абеля.} Пусть \begin{enumerate}
    \item $\sum\limits_{n = 1}^\infty a_n(x)$ равномерно сходится.
    \item $b_n$ равномерно ограничены.
    \item $b_n$ монотонны по $n$ при каждом фиксированном $x$.
\end{enumerate}
Тогда ряд $\sum\limits_{n = 1}^\infty a_n(x)b_n(x)$ будет равномерно сходящимся.

\begin{proof}
    К сожалению, мы не можем вывести признак Абеля из признака Дирихле, как мы это делали в числовых рядах.
    Поэтому будем пользоваться критерием Коши. 
    Мы хотим показать, что при достаточно больших $n$ и $m$ и любом $x$ данная сумма $\left| \sum\limits_{k = n + 1}^m a_k(x)b_k(x) \right|$ будет сколь угодно мала.
    Перепишем ее следующим образом:
    \begin{gather*}
        \left| \sum\limits_{k = 1}^{m-n} a_{n+k}(x)b_{n+k}(x) \right|
    \end{gather*} 
    Теперь применим преобразование Абеля: 
    \[ \left| \sum\limits_{k = 1}^{m-n} a_{n+k}(x)b_{n+k}(x) \right| = (A_m(x) - A_n(x))b_m(x) + \sum_{k = 1}^{m - n - 1} (A_{n+k}(x) - A_n(x))(b_{n+k}(x) - b_{n+k+1}(x)) \leqslant \oast \]
    \quad Частичные суммы тут действительно считаются правильно, ведь $a_{n+1}(x) + \dots + a_{n+k}(x) = A_{n+k}(x) - A_n(x)$.
    Оценим все по модулю: \[ \oast \leqslant |A_m(x) - A_n(x)||b_m(x)| + \sum_{k = 1}^{m - n - 1} |A_{n+k}(x) - A_n(x)||(b_{n+k}(x) - b_{n+k+1}(x)| < \oast \]
    \quad Мы знаем, что $\sum\limits_{n = 1}^\infty a_n(x)$ равномерно сходится, поэтому мы можем применить к ниму критерий Коши: $\forall \varepsilon > 0$ найдется такой номер $N$, что начиная с него, $\forall x \in E$ все отрезки ряда будет иметь сумму $<\varepsilon$.
    Следовательно, если $n$ будет $\geqslant N$, то $|A_m(x) - A_n(x)| < \varepsilon$ и $|A_{n+k}(x) - A_n(x)| < \varepsilon$.
    Также стоит вспомнить, что $b_n$ были равномерно ограничены $\Leftrightarrow \forall n \in \N \; \forall x \in E \;\; |b_n(x)| \leqslant K$.
    Отразим все это в нашем неравенстве: \[ \oast < K\varepsilon + \varepsilon \sum_{k = 1}^{m - n - 1} |(b_{n+k}(x) - b_{n+k+1}(x)| \leqslant \oast \]
    \quad Как и в доказательстве признака Дирихле воспользуемся монотонностью $b_n(x)$:
    \[ \sum_{k = 1}^{m - n - 1} |(b_{n+k}(x) - b_{n+k+1}(x)| = \left|\sum_{k = 1}^{m - n - 1} b_{n+k}(x) - b_{n+k+1}(x) \right| = |b_{n+1}(x) - b_m(x)| \]
    \quad Из того, что $|b_n(x)| \leqslant K$ очевидным образом следует, что $|b_{n+1}(x) - b_m(x)| \leqslant 2K$. 
    Подставляем это в наше неравенство: \[ \oast \leqslant 3K\varepsilon  \]
    \quad Таким образом, мы доказали, что при достаточно больших $n$ и $m$ и любом $x$ отрезки ряда $\sum\limits_{n = 1}^\infty a_n(x)b_n(x)$ будут сколь угодно малы, а значит этот ряд равномерно сходится по критерию Коши.
\end{proof}
