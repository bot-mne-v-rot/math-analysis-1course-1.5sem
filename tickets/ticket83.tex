\section{Локальные экстремумы. Определение и необходимое условие экстремума. Стационарные точки}
\begin{conj}
Пусть $f\colon E \to \R$, где $E \subset \R^n$ и $a \in E$. Тогда:
\begin{itemize}
    \item $a$ --- точка локального минимума, если существует $U$ --- окрестность точки $a$ такая, что $f(a) \leqslant f(x) \;\; \forall x \in U \cap E$.
    \item $a$ --- точка локального максимума, если существует $U$ --- окрестность точки $a$ такая, что $f(a) \geqslant f(x) \;\; \forall x \in U \cap E$.
    \item $a$ --- точка строгого локального минимума, если существует $U$ --- окрестность точки $a$ такая, что $f(a) < f(x) \;\; \forall x \in U \cap E,\, x \neq a$.
    \item $a$ --- точка строгого локального максимума, если существует $U$ --- окрестность точки $a$ такая, что $f(a) > f(x) \;\; \forall x \in U \cap E,\, x \neq a$.
    \item $a$ --- точка экстремума, если $a$ --- точка локального максимума или локального минимума.
    \item $a$ --- точка строгого экстремума, если $a$ --- точка строгого локального максимума или строгого локального минимума.
\end{itemize}
\end{conj}

\begin{theorem}[необходимое условие экстремума]
Пусть $f\colon E \to \R$, $a$ --- точка экстремума. Тогда, если существует $f'_{x_k}(a)$, то $f'_{x_k}(a) = 0$. В частности, если $f$ дифференцируема в точке $a$, то $f'_{x_1} = \dotsb = f'_{x_n}(a) = 0$, то есть $\triangledown f(a) = 0$.
\end{theorem}
\begin{proof}
Пусть существует $f'(x_1)(a)$. Рассмотрим функцию $g(t) = f(t, a_2, a_3, \dotsc, a_n)$. Тогда точка $t = a_1$ экстремум функции $g$, а значит по теореме про экстремумы функции одной переменной $g'(a_1) = 0 \implies f_{x_1}'(a_1, a_2, \dotsc, a_n) = 0$.
\end{proof}

\begin{conj}
Если $f$ --- дифференцируема в точке $a$ и $\triangledown f(a) = 0$, то $a$ --- стационарная точка.
\end{conj}

\textbf{Формула Тейлора в стационарной точке.}
\begin{equation*}
f(a + h) = f(a) + \frac{1}{2} \sum\limits_{i, j = 1}^{n} f''_{x_i x_j}(a) h_i h_j + o(\| h \|^2)
\end{equation*}