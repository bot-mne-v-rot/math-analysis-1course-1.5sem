\section{Вычисление интеграла $\int_{0}^{\pi/2} sin^n x dx$}
\textbf{Утверждение.}
\begin{equation*}
  \int_{0}^{\frac{\pi}{2}} \sin^n x \: dx =
  \int_{0}^{\frac{\pi}{2}} \cos^n x \: dx
\end{equation*}
\begin{proof}
    Пусть $x = \frac{\pi}{2} - t = \varphi(t), \, \varphi'(t) = -1$. Тогда
    \begin{equation*}
      \int_{0}^{\frac{\pi}{2}} \cos^n x \: dx = -\int_{0}^{\frac{\pi}{2}} \cos^n(\varphi(t))\varphi'(t) \: dt = -\int_{\varphi(0)}^{\varphi(\frac{\pi}{2})} \sin^n x\: dx =\int_{0}^{\frac{\pi}{2}} \sin^n x \: dx
    \end{equation*}
\end{proof}

Теперь посчитаем этот предел. Пусть
\begin{equation*}
  W_n \coloneqq \int_{0}^{\frac{\pi}{2}} \sin^n x \: dx, \qquad W_0 = \frac{\pi}{2},\, W_1 = 1
\end{equation*}
Тогда
\begin{equation*}
  \begin{gathered}
    W_n = \int_{0}^{\frac{\pi}{2}} \sin^n x \: dx =
    -\int_{0}^{\frac{\pi}{2}} \sin^{n - 1}x (\cos x)' \: dx
    \overset{\mathclap{n \geqslant 2}}{=}
    \underbrace{-\sin^{n - 1}x \cos x \big|_{0}^{\frac{\pi}{2}}}_{\mathclap{0}} +
    \int_{0}^{\frac{\pi}{2}}(n - 1)\sin^{n - 2}x \cos^2 x \: dx
    = \\ =
    (n - 1)\int_{0}^{\frac{\pi}{2}} \sin^{n - 2} x (1 - \sin^2 x) \: dx =
    (n -1 )(W_{n - 2} - W_n)
    \end{gathered}
\end{equation*}
Отсюда
\begin{equation*}
    \begin{gathered}
        W_n = (n - 1)(W_{n - 2} - W_n) \\
        nW_n  = (n - 1)W_{n - 2} \\
        W_n = \frac{n - 1}{n} W_{n - 2}
    \end{gathered}
\end{equation*}
Таким образом мы можем отдельно написать ответ для четных, отдельно для нечетных $n$.
\begin{equation*}
  \begin{gathered}
    W_{2n} =
    \frac{2n - 1}{2n}W_{2n - 2} =
    \frac{2n - 1}{2n}\cdot\frac{2n - 3}{2n - 2}W_{2n - 4}
    = \dotsb =
    \frac{(2n - 1)(2n - 3) \dotsm 1}{2n \cdot (2n - 2) \dotsm 2} \cdot \frac{\pi}{2} =
    \frac{(2n - 1)!!}{(2n)!!} \cdot \frac{\pi}{2}
    \\
    W_{2n + 1} =
    \frac{2n}{2n + 1} W_{2n - 1} =
    \frac{2n}{2n + 1}\frac{2n - 2}{2n - 1}W_{2n - 3}
    = \dotsb =
    \frac{2n(2n - 2)\dotsm 2}{(2n + 1)(2n - 1)\dotsm 3}\cdot 1 =
    \frac{(2n)!!}{(2n + 1)!!}
  \end{gathered}
\end{equation*}
Заметим, что
\begin{equation*}
    \sin^{0} x \geqslant \sin^{1} x \geqslant \sin^2 x \geqslant \dotsb \geqslant \sin^n x, \quad x \in [0, \frac{\pi}{2}]
\end{equation*}
Теперь применив монотонность интеграла к полученному равенству можно получить, что:
\begin{equation*}
    W_0 \geqslant W_1 \geqslant W_2 \geqslant \dotsb \geqslant W_n
\end{equation*}
Это поможет нам в доказательстве следующей теоремы.