\section{Связь частных производных и дифференцируемости}

\begin{theorem}
    Пусть $f : E \to \R$, $E \subset \R^n$, $a \in \Int f$,
    все частные производные $f$ существуют в окрестности точки $a$
    и непрерывны в точке $a$. Тогда $f$ дифф. в точке $a$.
\end{theorem}
\begin{proof} $ $

    Пусть:
    $$R(h) = f(a + h) - f(a) - 
    \underbrace{\sum_{k = 1}^n \frac{df}{dx_k}(a) \cdot h_k}
    _{= \langle \nabla f(a), h \rangle = d_a f(h)} $$

    где традиционно $h \in \R^n$. Надо док-ть, что $R(h) = o(\norm{h})$.

    Введём вспомогательные точки:
    
    \noindent
    \begin{minipage}[t]{.5\textwidth}
        $$ b_k = \begin{pmatrix*}
            a_1 + h_1 \\
            a_2 + h_2 \\
            \vdots \\
            a_k + h_k \\
            a_{k + 1} \\
            \vdots \\
            a_n
        \end{pmatrix*}
        = a + \begin{pmatrix*}
            h_1 \\
            h_2 \\
            \vdots \\
            h_k \\
            0 \\
            \vdots \\
            0
        \end{pmatrix*} $$
    \end{minipage}% <---------------- Note the use of "%"
    \begin{minipage}[t]{.5\textwidth}
    Пример на
    двумерном случае:
    $$\begin{tikzpicture}
        \draw[thick,->] (0,0) -- (3,0) node[anchor=north west] {x};
        \draw[thick,->] (0,0) -- (0,3) node[anchor=south east] {y};
        \fill[thick, black] (0.5, 1) circle(0.07) 
        node[below right]{$a = b_0$};
        \fill[thick, black] (2.5, 1) circle(0.07) 
        node[below right]{$b_1$};
        \fill[thick, black] (2.5, 2.5) circle(0.07) 
        node[below right]{$h = b_2$};
    \end{tikzpicture}$$
    \end{minipage}

    Т.е. мы к первым $k$ координатам $a$ прибавляем первые $k$
    координат $h$, а остальные $n - k$ координат $a$ оставляем не
    тронутыми. По сути, эти точки -- способ за $n$ шагов прошагать от точки
    $a$ до точки $h$, прибавляя по одной координате. 

    Зададим ещё одну функцию:
    $$ F_k(t) = f(b_{k-1} + t h_k e_k) $$
    где традиционно $t \in \R$, $e_k$ -- $k$-й базисный вектор. Т.е.
    мы сдвинулись по $k-1$ координате в сторону $h$ и теперь двигаемся
    по $k$-й координате в направлении $h_k$. Более того, 
    $F_k(1) = f(b_k)$.

    Это одномерная функция, поэтому можем применить одномерную
    теорему Лагранжа:
    $$ F_k(1) - F_k(0) = F_k'(\Theta_k) \quad\quad \Theta_k \in (0, 1) $$

    Давайте внимательно посмотрим на $F_k(t)$:
    $$ F_k(t) = f(b_{k-1} + t h_k e_k) =
    f\begin{pmatrix*}
        a_1 + h_1 \\
        \vdots \\
        a_{k-1} + h_{k-1} \\
        a_k + t h_k \\
        a_{k + 1} \\
        \vdots \\
        a_n
    \end{pmatrix*}$$
    Т.е. меняется только $k$-я координата, а значит по сути:
    $$ F_k'(t) = 
    \underbrace{
        \overbrace{\frac{df}{dx_k}(b_{k-1} + t h_k e_k)}
        ^{\text{сущ. по условию теор.}}
    \cdot \frac{d(b_{k-1} + t h_k e_k)}{dx_k}(t)}
    _{\text{произв. композиции}} = \frac{df}{dx_k}(b_{k-1} + t h_k e_k)
    \cdot h_k $$
    Подставляем $\Theta_k$:
    $$ F_k'(\Theta_k) = 
    \frac{df}{dx_k}(b_{k-1} + \Theta_k h_k e_k) \cdot h_k $$
    Собираем вместе:
    $$f(a + h) - f(a) = F_n(1) - F_1(0)
    \underbrace{= \sum_{k = 1}^n (F_k(1) - F_k(0))}
    _{\text{т.к. $F_{k-1}(1) = F_k(0)$}} =
    \sum_{k = 1}^n \frac{df}{dx_k}
    (\overbrace{b_{k-1} + \Theta_k h_k e_k}^{=: c_k}) \cdot h_k
    = \sum_{k = 1}^n \frac{df}{dx_k} (c_k) \cdot h_k $$
    Вспомним про $R(h)$:
    $$R(h) = f(a + h) - f(a) - \sum_{k = 1}^n \frac{df}{dx_k}(a) 
    \cdot h_k = \sum_{k = 1}^n \left(\frac{df}{dx_k} (c_k) -
    \frac{df}{dx_k}(a) \right) \cdot h_k$$
    Применим неравенство Коши-Буняковского:
    $$\abs{R(h)} = \abs{\sum_{k = 1}^n \left(\frac{df}{dx_k} (c_k) -
    \frac{df}{dx_k}(a) \right) \cdot h_k}
    \stackrel{\text{КБШ}}{\leqslant} \sqrt{\sum_{k=1}^n 
    \left(\frac{df}{dx_k} (c_k) - \frac{df}{dx_k}(a) \right)^2}
    \cdot \underbrace{\sqrt{\sum_{k=1}^n h_k^2}}_{= \norm{h}}
    $$
    Получаем, что:
    $$ \left(\frac{R(h)}{\norm{h}} \right)^2 \leqslant  \sum_{k=1}^n 
    \left(\frac{df}{dx_k} (c_k) - \frac{df}{dx_k}(a) \right)^2 $$
    Нужно доказать, что эта сумма стремится к 0 при $h \to 0$.
    Вспомним, что сходимость эквивалентна покоординатной сходимости:
    \begin{gather*}
        h \to 0 \Rightarrow h_k \to 0 \Rightarrow a_k + h_k \to a_k
        \Rightarrow b_k \to a \\
        c_k = \underbrace{b_{k-1}}_{\to a} + 
        \underbrace{\Theta_k}_{\in (0, 1)} \cdot
        \underbrace{h_k}_{\to 0} \cdot
        \underbrace{e_k}_{\text{const}} \to a
    \end{gather*}
    $\frac{df}{dx_k}$ непрерывна в точке $a$, а значит:
    $$\lim_{h \to 0} \frac{df}{dx_k}(c_k) = 
    \lim_{c_k \to a} \frac{df}{dx_k}(c_k) = 
    \frac{df}{dx_k}(a)$$
    Собираем вместе:
    $$\left(\frac{R(h)}{\norm{h}} \right)^2 \leqslant  \sum_{k=1}^n 
    \underbrace{\left(\frac{df}{dx_k} (c_k) - 
    \frac{df}{dx_k}(a) \right)^2}_{\to 0} \to 0 \Longrightarrow
    R(h) = o(\norm{h})$$
    По определению дифференцируемости получаем, что $f$ дифф. 
    в точке $a$.
\end{proof}

\textit{Теперь у нас есть более удобный способ проверять дифференц.
функции в точке, чем просто по определению, т.к. если функция
состоит из элементарных функции, то нам легко понять, какие у неё
частные производные, и легко понять, непрерывны ли они в точке.}

\textbf{Замечания:}
\begin{enumerate}
    \item Непрерывность частной производной по первой координате
    не используется:
    $$ F_1(1) - F_1(0) = f(a + h_1 e_1) - f(a) = 
    \underbrace{\frac{df}{dx_1}(a) \cdot h_1 + o(h_1)}_
    {\text{опр. дифф. в точке $a$}} = 
    \frac{df}{dx_1}(a) \cdot h_1 + o(\norm{h})$$
    По второй координате мы уже так сделать не можем:
    $$ F_2(1) - F_2(0) = 
    \underbrace{f(a + h_1 e_1 + h_2 e_2) - f(a + h_1 e_1)}
    _{\text{точка $(a + h_1 e_1) \neq a$}} $$

    \item Дифф. в точке не даёт существования частных производных
    в окр-ти и уж тем более их дифф. 

    \textbf{Пример:}
    $$ f(x, y) = \begin{cases}
        x^2 + y^2, & \text{если ровно одно из них $\in \Q$,} \\
        0, &\text{иначе.}
    \end{cases} $$

    $f$ дифф. в точке $(0, 0)$:
    $$ \underbrace{f(x, y)}_{\leqslant x^2 + y^2} = 
    \underbrace{f(0, 0)}_{=0} + o(\norm{(x, y)}) = 
    f(0, 0) + o(\sqrt{x^2 + y^2}) $$

    При этом, в любой окрестности $(0, 0)$ найдётся точка, 
    отличная от $(0, 0)$, в которой нет даже непрерывности.
    Пусть $x_0, y_0 \in \R \setminus \Q$. $f(x_0, y_0) = 0$.
    Зафиксируем вторую координату и будем двигаться по первой
    координате. Тогда в любой окрестности $x_0$ найдётся рациональное
    число $x$, а значит, там будет $f(x, y_0) \neq 0$. 
    При этом если брать рациональные точки слева от $x_0$, то при 
    приближении к $x_0$, значение $f(x, y_0)$ будет только расти,
    а значит, ни о какой сходимости к 0 говорить нельзя.
\end{enumerate}