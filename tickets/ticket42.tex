\section{Пространство $\ell^\infty (E)$ и его полнота}
\begin{conj} Нормированое пространство ограниченных функций
    \[ l^{\infty}(E) := \{ f: E \to R; f - \text{ ограниченая функция} \} \]
    
    Введем на этом пространстве следующую норму:    
    \[ \norm{f}_{l^{\infty}(E)} := \norm{f}_{\infty} := \sup_{x \in E}{\abs{f(x)}} \text{ (он конечен, т.к. $f$~--- ограничена)} \]

    Аксиомы нормы очевидны (кроме неравенства $\triangle$):
    \begin{gather*}
        \norm{f + g}_{\infty} \leqslant \norm{f}_{\infty} + \norm{g}_{\infty} \\
        \sup_{x \in E}{\abs{f(x) + g(x)}} \leqslant \sup_{x \in E}(\abs{f(x)} + \abs{g(x)}) \leqslant \sup_{x \in E}{\abs{f(x)}} + \sup_{x \in E}{\abs{g(x)}} = \norm{f}_{\infty} + \norm{g}_{\infty}
    \end{gather*}
\end{conj}

\vspace*{5mm}

\notice \;
Равномерная сходимость $=$ сходимость по норме $\norm{\cdot}_{l^{\infty}(E)}$:
\[ f_n \doublerightarrow f \text{ на } E \Longleftrightarrow \norm{f_n - f}_{l^{\infty}(E)} \to 0 \]
Это действительно так, ведь просто по определению $\norm{f_n - f}_{l^{\infty}(E)} = \sup_{x \in E}{\abs{f_n(x) - f(x)}}$)
\begin{theorem}
    $l^{\infty}(E)$~--- полное пространство.
\end{theorem}

\begin{proof}
    Нужно доказать, что любая фундаментальная последовательность имеет предел, лежащий в этом пространстве.

    \quad Пусть $f_n$~--- фундаментальная последовательность:
    \begin{gather*}
        \forall \varepsilon > 0 \; \exists N: \; \forall n,m \geqslant N \;\; \underbrace{\norm{f_n(x) - f_m(x)}}_{= \sup\limits_{x \in E}{\abs{f_n(x) - f_m(x)}}} < \varepsilon \\
        \Longrightarrow \forall \varepsilon > 0 \;\; \exists N: \forall n,m \geqslant N \;\; \forall x \in E \;\;\;\; \abs{f_n(x) - f_m(x)} < \varepsilon - \text{ это критерий Коши} \\
        \Longrightarrow f_n \doublerightarrow f \Longrightarrow \norm{f_n - f} \to 0
    \end{gather*}
    \quad Осталось убедиться, что $f$, к которой сходится наша фундаментальная последовательность, лежит в нашем пространстве, т.е. является ограниченной функцией.

    \quad Возьмём $\varepsilon = 1$ из определения предела: $\; \exists N \; \forall n \geqslant N \;\; \norm{f_n - f} < 1$. 
    В терминах супремума это означает, что $\sup\limits_{x \in E}{\abs{f_n(x) - f(x)}} < 1$.
    Оставим только $N$-тую функцию: $\forall x \in E \;\; \abs{f_{N}(x) - f(x)} < 1$.
    Мы знаем, что она ограничена, так как лежит в $l^{\infty}(E)$, поэтому $\forall x \in E \;\; \abs{f(x)} < 1 + \abs{f_{N}(x)} \leqslant 1 + C$.
\end{proof}

