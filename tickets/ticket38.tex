\section{Бесконечные произведения. Определение. Примеры. Свойства }
\begin{conj}
    $\prodn p_n$

    $P_n = p_1p_2 \ldots p_n$ - частичные произведения.

    Если существует $\lim P_n$, то его называют значением бесконечного произведения (по аналогии с суммой ряда).
    Если он конечен и \textbf{отличен от нуля}, то говорят, что произведение сходится.
\end{conj}

\textbf{Пример}
\begin{enumerate}
    \item $\prod \limits_{n=2}^\infty (1 - \frac{1}{n^2})$:
    \begin{align*}
        P_n &= (1 - \frac{1}{2^2})(1-\frac{}{3^2})\ldots (1-\frac{1}{n^2}) \\
        &= \frac{(2-1)(2+1)}{2^2}
        \cdot \frac{(3-1)(3+1)}{3^2} 
        \cdot \ldots \cdot \frac{(n-1)(n+1)}{n^2} \\
        &= \frac{n+1}{2n} \longrightarrow \frac{1}{2}
    \end{align*}
    \item $\prodn (1 - \frac{1}{4n^2})$:
    \begin{align*}
        P_n &= \prod \limits_{k=1}^n (1 - \frac{1}{(2k)^2}) \\
        &= \prod \limits_{k=1}^n \frac{(2k-1)(2k+1)}{2k^2} \\
        &= \frac{((2n-1)!!)^2(2n+1)}{((2n)!!)^2} \longrightarrow \frac{2}{\pi} \text{ по Формуле Валлиса}
    \end{align*}
\end{enumerate}
\textbf{Упражнения}
\begin{enumerate}
    \item $\prodn (1 - \frac{1}{(2n+1)^2}) = \frac{\pi}{4}$
    \item $\prodn (1+x^{2^{n-1}}) = \frac{1}{1-x}$ при $|x| < 1$
\end{enumerate}

\textbf{Свойства}
\begin{enumerate}
    \item Конечное число начальных множителей не влияет на сходимость.
    \item Если $\prodn p_n$ сходятся, то $\lim p_n = 1$
    \begin{proof}
        $p_n = \frac{P_n}{P_{n-1}} \to \frac{P}{P} = 1$, если $P \neq 0, \pm \infty$
    \end{proof}
    \item У сходящегося произведения все сомножители положительны, начиная с некоторого места (так как стремятся к 1). Тогда можем считать, что все сомножители положительны.
    \item Для сходимости $\prodn p_n$ при $p > 0$ необходима и достаточна сходимость ряда $\sumn \ln p_n$.
    И если $L = \sumn \ln p_n$, то $P = e^L$
    \begin{proof}
        \begin{align*}
            P_n = \prodk p_k \Longrightarrow \ln P_n &= \sumk \ln p_k \\
            &= S_n \to L \longrightarrow e^{\ln P_n} \\
            &= P_n \longrightarrow e^L
        \end{align*}
        $P_n$ сходится $\Longleftrightarrow \lim P_n \neq 0, +\infty \Longleftrightarrow \lim S_n \neq \pm \infty \Longleftrightarrow S_n$ сходится.
    \end{proof}
\end{enumerate}