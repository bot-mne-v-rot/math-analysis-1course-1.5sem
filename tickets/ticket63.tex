\section{Дифференцируемость отображений из $\R^n$ в $\R^m$. Частные случаи. Матрица Якоби. Градиент}


\begin{conj} $ $\\
    Пусть $f : E \to \R^m$, $E \subset \R^n$, $a \in \Int E$. \\
    $f$ \textbf{дифференцируемо} в точке $a$, если существует
    линейное отображение \\ $T : \R^n \to \R^m$, т.ч.
    $f(a + h) = f(a) + Th + o(\norm{h})$ при $h \to 0$.
\end{conj}

\begin{conj}
    $T$ -- \textbf{дифференциал} $f$ в точке $a$. Обозначается $d_a f$.
\end{conj}

$ $

\begin{theorem-non}
    Если $T$ существует, то оно единственно.
\end{theorem-non}
\begin{proof} $ $ \\
    Пусть $h \in \R^n$.
    Возьмём из головы формулу и поймём, что она верная:
    $$\lim_{t \to 0} \frac{f(a + th) - f(a)}{t} =
    \lim_{t \to 0} \frac{f(a) + T(th) + o(\norm{th}) - f(a)}{t} =
    \lim_{t \to 0} \frac{tTh + o(\norm{th})}{t} = Th$$
    Таким образом, если $T$ существует, то для любого $h$
    можно однозначно вычислить $Th$.
\end{proof}

\begin{conj}
    Матрица оператора $T$ -- \textbf{матрица Якоби} функции 
    $f$ в точке $a$. \\
    Обозначается $f'(a)$.
\end{conj}

\begin{theorem-non}
    Если $f$ дифференцируема в точке $a$, то $f$ непрерывна в точке $a$.
\end{theorem-non}
\begin{proof}
    $$\lim_{h \to 0} f(a + h) = \lim_{h \to 0}(f(a) + Th + o(h))
    = f(a) + \lim_{h \to 0} Th = f(a)$$
    Так как $\norm{Th} \leqslant \norm{T} \norm{h}$.
\end{proof}

$ $

\textbf{Важный частный случай:} $m = 1$. \\
Тогда $T : \R^n \to \R$. И матрица $T$ =
$\begin{pmatrix*}
    t_1 & \dots & t_n
\end{pmatrix*}$.
Пусть $h = \begin{pmatrix*}
    h_1 \\ \vdots \\ h_n
\end{pmatrix*}$. \\
Тогда $Th = \sum_{k = 1}^n t_k h_k$. 
Это по сути скалярное произведение.

$ $

\begin{conj}
    Пусть $f : \R^n \to \R$. $f$ дифференцируема в точке $a$. \\
    Тогда найдётся $v \in \R^n$, т.ч. $f(a + h) = f(a) + 
    \langle v, h \rangle + o(\norm{h})$ при $h \to 0$. \\
    $v$ --- \textbf{градиент} функции $f$ в точке $a$.
    Обозначается $\operatorname{grad} f$ или $\nabla f$ (``набла'' $f$).
\end{conj}