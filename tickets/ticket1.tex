\section{Положительная и отрицательная части функции и их свойства. Подграфик функции. Определенный интеграл. Определение и простейшие свойства}
\begin{conj}
    Отображение $\sigma\colon \overbrace{\mathcal{F}}^{\mathclap{\text{все огр. подмн-ва плоскости}}} \to [0, +\infty]$ --- квазиплощадь, если выполняются следующие свойства:
  \begin{enumerate}
     \item $\sigma([a, b] \times [c, d]) = (b - a) (d - c)$
     \item $E \subset \widetilde{E} \implies \sigma(E) \leq \sigma(\widetilde{E})$
     \item $E = E_{-} \cup E_{+} \implies \sigma(E) = \sigma(E_{-}) + \sigma(E_{+})$
  \end{enumerate}
\end{conj}

\textbf{Обозначение.}
$f_{+}(x) = \max\{f(x), 0\}, \qquad f_{-} = \max\{-f(x), 0\}$

\textbf{Свойства:}
\begin{enumerate}
  \item $f_{\pm} \geq 0$
  \item $f = f_{+} - f_{-}, \quad |f| = f_{+} + f_{-}$
  \item $f_{+} = \frac{f + |f|}{2}, \quad f_{-} = \frac{|f| - f}{2}$
  \item Если $f$ непрерывна, то $f_{\pm}$ непрерывна
\end{enumerate}
\begin{conj}
  Подграфиком функции $f\colon E \to \R, f \geq 0$ называется
  \begin{equation*}
    P_f = \{ (x, y)\colon x \in E, \, 0 \leq y \leq f(x) \}
  \end{equation*}
\end{conj}

\begin{notice}
    Подграфик непрерывной на отрезке функции --- ограниченное множество.
\end{notice}

\begin{conj}
  Пусть $f \in C[a, b]$. Тогда интегралом от $a$ до $b$ функции $f$ называют
    \begin{equation*}
      \int_{a}^{b} f =
      \int_{a}^{b} f(x) \: dx \coloneqq \sigma(P_{f_{+}}) - \sigma(P_{f_{-}})
    \end{equation*}
    где $\sigma$ --- некоторая заранее зафиксированная квазиплощадь.
\end{conj}

\textbf{Свойства:}
\begin{enumerate}
\item $\displaystyle \int_{a}^{a} f = 0$
\item $\displaystyle \int_{a}^{b} 0 = 0$
\item Если $f \geq 0$, то $\displaystyle \int_{a}^{b} f = \sigma(P_f)$
\item $\displaystyle \int_{a}^{b}(-f) = -\int_{a}^{b} f$
\begin{proof}
    \begin{equation*}
        \begin{gathered}
            (-f)_{+} = f_{-} \\
            (-f)_{-} = f_{+}
        \end{gathered}
        \implies
        \int_{{a}}^{{b}} {-f} \: d{x} =
        \sigma(P_{f_{-}}) - \sigma(P_{f_{+}}) =
        - \int_{{a}}^{{b}} {f} \: d{x}
    \end{equation*}
\end{proof}
\item $\displaystyle \int_{a}^{b} c = c(b - a)$, где $c$ --- константа
\item Если $a < b,\, f \geq 0, \, f \not \equiv 0$, то $\displaystyle \int_{a}^{b} f > 0$
\begin{proof}
    Пусть $f(x_0) > 0$. Тогда на $[x_0 - \delta, x_0 + \delta]$:
    \begin{equation*}
    f(x) \geq \frac{f(x_0)}{2}
    \implies
    P_f \supset [x_0 - \delta, x_0 + \delta] \times [0, \frac{f(x_0)}{2}] \implies
    \int_{a}^{b} f = \sigma(P_f) \geq 2\delta \cdot \frac{f(x_0)}{2} = f(x_0)\delta > 0
    \end{equation*}
\end{proof}
\end{enumerate}