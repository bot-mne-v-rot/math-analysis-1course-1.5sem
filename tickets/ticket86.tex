\section{Наибольшее и наименьшее значения квадратичной формы на сфере. Формула для нормы матриц}
Обсудим поясняющий пример.

\textbf{Наибольшее и наименьшее значения квадратичной формы на сфере.}

    Квадратичная форма задается симметричной матрицей $A$: $Q(h) = \langle Ah, h \rangle$.

    Сфера задается уравнением $x_1^2 + \dots + x_n^2 = 1$.
    Значит, $\Phi(x) = x_1^2 + \dots + x_n^2 - 1$.

    Пишем функцию Лагранжа: \[ F(x) = \sum_{i, j = 1}^n a_{ij}x_ix_j - \lambda(x_1^2 + \dots + x_n^2 - 1) \]
    Нас интересует такая точка, в которой $\Phi = 0$, и все частные производные $F$ равны 0.

    Распишем, чему равна $k$-тая частная производная $F$ (мы фиксируем $x_k$, остальное воспринимаем как параметры): \[ F_{x_k}' = a_{kk}*2x_k + \sum_{i\neq k} a_{ik}x_i + \sum_{j \neq k} a_{kj}x_j -2\lambda x_k = \circledast \]
    Заметим, что так как матрица $A$ симметричная, выражение можно упростить: \[ \circledast = 2\sum_{i = 1}^n a_{ki}x_i - 2\lambda x_k \]
    Что такое $\sum\limits_{i = 1}^n a_{ki}x_i$? 
    Это ни что иное как $k$-ая координата вектора $Ax$. 
    Таким образом, все координаты вектора $2Ax$ равны координатами вектора $2\lambda x$, или чуть короче: \[ Ax = \lambda x \]
    Это означает, что $x$ -- собственный вектор матрицы $A$ с собственным значением $\lambda$.
    Таким образом, точки, которые подозрительные на экстремум, обязаны быть собственными векторами матрицы.
    
    Также легко понять, какие значения принимает форма на этих векторах: \[ Q(x) = \langle Ax, x \rangle = \langle \lambda x, x \rangle = \lambda \| x \|^2 = \lambda  \]
    Последний переход верен, так как $x$ лежит на единичной сфере.
    Зафиксируем результат в виде теоремы.

\begin{theorem}
    Наибольшее (наименьшее) значение квадратичной формы на сфере -- наибольшее (наименьшее) собственное число ее матрицы.
\end{theorem}

\vspace*{6mm}

\follow $\, \| A \| = \max \{ \sqrt{\lambda} : \lambda - \text{ собственное число матрицы } A^TA \}$, где $A$ -- матрица линейного отображения из $\R^n$ в $\R^m$.
\begin{proof}
    По определению нормы $\|A\| = \max\limits_{\|x\| = 1} \|Ax\|^2 = \max\limits_{\|x\| = 1} \langle Ax, Ax \rangle$.
    
    Воспользуемся алгебраическим фактом: $\langle Ax, Ax \rangle = \langle A^TAx, x \rangle$.
    А ведь $\max\limits_{\|x\| = 1} \langle A^TAx, x \rangle$ это и есть наибольшее значение квадратичной формы для матрицы $A^TA$ на единичной сфере.
\end{proof}
