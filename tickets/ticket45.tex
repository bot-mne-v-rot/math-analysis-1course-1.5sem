\section{Признак сравнения и признак Вейерштрасса. Следствия. Примеры}
Проверять равномерную сходимость по определению или критерию Коши зачастую оказывается не очень удобно.
Поэтому поговорим о признаках равномерной сходимости функциональных рядов.

\textbf{Признак сравнения.}
Пусть $u_n, v_n\colon E \to \R$ и $\forall x \in E \; \forall n \in N \;\; |u_n(x)| \leqslant v_n(x)$.
Тогда если $\sum\limits_{n = 1}^\infty v_n(x)$ сходится равномерно на $E$, то $\sum\limits_{n = 1}^\infty u_n(x)$ сходится равномерно на $E$.

\begin{proof}
    Распишем равномерную сходимость $\sum\limits_{n = 1}^\infty v_n(x)$ по критерию Коши:
    \[ \forall \varepsilon > 0 \; \exists N : \forall n,m \geqslant N \; \forall x \in E \;\; \sum_{k=n+1}^m v_k(x) < \varepsilon \]
    \quad Мы убрали модуль, так как $v_n(x)$ по условию неотрицательны.
    Воспользуемся неравенством: \[ |u_n(x)| \leqslant v_n(x) \Longrightarrow \sum_{k = n + 1}^m |u_k(x)| \leqslant \sum_{k = n+1}^m v_k(x) \Longrightarrow \left| \sum_{k = n + 1}^m u_k(x) \right| \leqslant \sum_{k = n+1}^m v_k(x) < \varepsilon \]
    \quad Таким образом, $\sum\limits_{n = 1}^\infty u_n(x)$ сходится равномерно согласно критерию Коши.
\end{proof}

\follow \, Из абсолютной сходимости следует обычная.
Тут и доказывать нечего, просто подставляем $v_n(x) = |u_n(x)|$ в признак сравнения. 

\vspace*{7mm}

\textbf{Признак Вейерштрасса.} 
Если $\forall x \in E \; \forall n \in \N \;\; |u_n(x)| \leqslant a_n$ и $\sum\limits_{n = 1}^\infty a_n$ сходится, то $\sum\limits_{n = 1}^\infty u_n(x)$ сходится равномерно на $E$.
\begin{proof}
    Это прямое следствие признака сравнения.
    Берем $v_n(x) = a_n$. 
    Тогда $\sum\limits_{n = 1}^\infty v_n(x)$ сходится равномерно, так как никакой зависимости от $x$ вообще нет.
    Значит и ряд $\sum\limits_{n = 1}^\infty u_n(x)$ сходится равномерно.
\end{proof}

\begin{example}
    $\sum\limits_{n = 1}^\infty \frac{\sin(nx)}{n^2}$ равномерно сходится, так как $\left| \frac{\sin(nx)}{n^2} \right| \leqslant \frac{1}{n^2}$ и ряд $\sum\limits_{n = 1}^\infty \frac{1}{n^2}$ сходится.
\end{example}

\notice \; Абсолютная поточечная и равномерная сходимость это разные вещи.
Я без понятия, зачем такое тупое замечание, но все же.
Приведем примеры, иллюстрирующее это: \begin{itemize}
    \item $\sum\limits_{n = 1}^\infty x^n$ на $(-1, 1)$ абсолютно поточечно сходится, так как при фиксированном $x$ это сумма геометрической прогрессии, но равномерной сходимости очевидно нет, например, так как члены не равномерно стремятся к 0: $x^n \not \doublerightarrow 0$.
    \item $\sum\limits_{n = 1}^\infty \frac{(-1)^n}{n}$ сходится равномерно, так как от $x$ не зависит и числовой ряд сходится, но абсолютной сходимости нет, так как получается гармонический ряд.
    \item Дальше будет пример, когда ряд сходится асбсолютно и равномерно, но не равномерно абсолютно.
\end{itemize}

