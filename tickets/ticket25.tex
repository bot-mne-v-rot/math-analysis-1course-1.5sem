\section{Ряды в нормированном пространстве. Критерий Коши. Абсолютная сходимость}

\begin{conj}
  Пусть $X$ --- нормированное пространство, $\|.\|$ --- норма, $x_n \in X$. Тогда
  $\smash{\sum\limits_{k = 1}^{\infty} x_n}$ --- ряд. Частичная сумма ряда
  $S_n = \sum\limits_{k = 1}^{n} x_k$. Если существует $\lim S_n$, то он называется суммой ряда. Ряд сходящийся, если его сумма конечна.
\end{conj}

\begin{theorem}[необходимое условие сходимости]
  Если ряд $\sum\limits_{n = 1}^{\infty}$ сходится, то $\lim x_n = 0$.
\end{theorem}
\begin{proof}
  Пусть $S = \lim S_n$. Тогда
  \begin{equation*}
    x_n = S_n - S_{n - 1} \implies \lim x_n = \lim S_n - \lim S_{n - 1} = S - S = 0
  \end{equation*}
\end{proof}

\textbf{Свойства}.
\begin{enumerate}
  \item Линейность суммы. Если ряды $\sum\limits_{n = 1}^{\infty} x_n$ и $\sum\limits_{n = 1}^{\infty} y_n$ сходятся и $\alpha, \, \beta \in \R$, то ряд $\sum\limits_{n = 1}^{\infty}(\alpha x_n + \beta y_n)$ сходится и
  \begin{equation*}
    \sum\limits_{n = 1}^{\infty} (\alpha x_n + \beta y_n) = \alpha\sum\limits_{n = 1}^{\infty} x_n + \beta\sum\limits_{n = 1}^{\infty} y_n
  \end{equation*}
  \begin{proof}
    Пусть $X_n \coloneqq \sum\limits_{k = 1}^{n} x_k, \, Y_n \coloneqq \sum\limits_{k = 1}^{n} y_k, \, S_n \coloneqq \sum\limits_{k = 1}^{n}(\alpha x_k + \beta y_k) =
    \alpha X_n + \beta Y_n$. А значит свойство выполняется просто по линейности перехода к пределу в нормированном пространстве.
  \end{proof}

  \item Расстановка скобок не меняет суммы.

  \item В $\C$ и $\R^d$ ряды сходятся тогда и только тогда, когда они сходятся покоординатно.
\end{enumerate}

\begin{theorem}[критерий Коши]
  Пусть $X$ --- полное нормированное пространство. Тогда
  \begin{equation*}
    \text{Ряд }\sum\limits_{n = 1}^{\infty} x_n\text{ сходится }\iff \; \forall \varepsilon > 0 \; \exists N \; \forall m,\, n \geq N\colon
    \left \|\sum\limits_{k = n}^{m} x_K \right \| < \varepsilon
  \end{equation*}
\end{theorem}
\begin{proof}
  \begin{align*}
    \text{Ряд сходится } &\iff \\
    \text{существует }\lim S_n &\iff \\
    S_n\text{ --- фундаментальная последовательность } &\iff \\
    \forall \varepsilon > 0 \; \exists N \; \forall m, n \geq N\colon \|S_n - S_m\| < \varepsilon &\iff \\
    \forall \varepsilon > 0 \; \exists N \; \forall m, n \geq N\colon
    \left\|\sum\limits_{k = m + 1}^{n} x_k\right\| < \varepsilon&
  \end{align*}
\end{proof}

\begin{conj}
  Ряд $\sum\limits_{n = 1}^{\infty} x_n$ --- абсолютно сходится, если
  $\sum\limits_{n = 1}^{\infty} \|x_n\|$ --- сходится.
\end{conj}

\begin{theorem}
  В полном нормированном пространстве если ряд абсолютно сходится, то он сходится и
  \begin{equation*}
    \left\|\sum\limits_{n = 1}^{\infty} x_n\right\| \leq
    \sum\limits_{n = 1}^{\infty}\left\|x_n\right\|
  \end{equation*}
\end{theorem}
\begin{proof}
  $\sum\limits_{n = 1}^{\infty}\|x_n\|$ сходится, а значит
  \begin{equation*}
    \forall \varepsilon > 0 \; \exists N \; \forall n, m \geq N\colon \sum\limits_{k = n}^{m} \|x_k\| < \varepsilon
  \end{equation*}
  Но
  \begin{equation*}
    \sum\limits_{k = n}^{m} \| x_k \| \geq \left\| \sum\limits_{k = n}^{m} x_k \right\| \implies
  \sum\limits_{n = 1}^{\infty} x_n\text{ --- сходится}
  \end{equation*}
  Теперь докажем неравенство, мы знаем, что $\Big \| \sum\limits_{k = 1}^{n} x_k \Big \| \leq \sum\limits_{k = 1}^{n} \| x_k \|$. Применив предельный переход и факт о том, что предел нормы равен норме предела получаем нужное неравенство. Теорема доказана.
\end{proof}

\begin{notice}
  Если $\sum x_n$ и $\sum y_n$ абсолютно сходятся, то $\sum(\alpha x_n + \beta y_n)$ сходится.
\end{notice}
\begin{proof}
  По неравенству треугольника $\| \alpha x_n + \beta y_n \| \leq | \alpha | \| x_n \| + | \beta | \| y_n \|$.
\end{proof}