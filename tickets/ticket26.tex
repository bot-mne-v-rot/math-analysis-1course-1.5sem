\section{Группировка членов ряда. Свойства}

\textbf{Достаточные условия сходимости при группировке членов ряда}.
\begin{enumerate}
  \item Каждая группа состоит из $\leq M$ слагаемых, $\lim x_n = 0$ и сгрупированный ряд сходится к~$S$. Тогда $\sum\limits_{n = 1}^{\infty} x_n = S$.
  \begin{proof}
    Заметим, что частичные суммы сгрупированного ряда это просто какая-то подпоследовательность частичных сумм нашего ряда $S_{n_1}, S_{n_2}, S_{n_3}, \dotsc, \quad \lim S_{n_k} = S$. Мы знаем, что в каждой группе ограниченное число слагаемых:
    \begin{equation*}
      \hphantom{, \quad n_k - n_{k - 1} \leq M}
      n_1, n_2 - n_1, n_3 - n_2, \dotsc
      , \quad n_k - n_{k - 1} \leq M
    \end{equation*}
    Возьмем произвольное $n\colon \; n_k \leq n < n_{k + 1}$.
    Тогда $S_n = S_{n_k} + \overbrace{x_{n_{k + 1}} + x_{n_{k + 2}} + \dotsc + x_n}^{\mathclap{\text{
       $\leq M$ слагаемых
    }}}$. Тогда
    \begin{equation*}
      \| S_n - S \| \leq \| S_{n_k} - S \| + \| x_{n_{k + 1}} \| + \| x_{n_{k + 2}} \| + \dotsc + \| x_n \|
    \end{equation*}
    Также заметим, что
    \begin{equation*}
      \lim x_n = 0 \implies \exists N\colon \forall m \geq N\colon \| x_m \| < \varepsilon
    \end{equation*}
    Тогда мы знаем, что при достаточно больших $k\colon \|S_{n_k} - S\| < \varepsilon$, а так же при достаточно больших $k\colon \forall m \geq n_k\colon \| x_m \| < \varepsilon$. Значит при достаточно больших $k\colon$
    \begin{equation*}
      \| S_n - S \| \leq (M + 1)\varepsilon
    \end{equation*}
    где $(M + 1)$ --- константа, а $\varepsilon$ может быть сколь угодно маленьким. Значит
    \begin{equation*}
      \lim \| S_n - S \| = 0 \implies \lim S_n = S
    \end{equation*}
  \end{proof}

  \item Для числовых рядов. Члены ряда в каждой группе одного знака и сгрупированный ряд сходится. Тогда и исходный ряд сходится.
  \begin{proof}
    Пусть $S_{n_1}, S_{n_2}, \dotsc$ --- частичные суммы для сгрупированного ряда. Берем произвольное $n$. Тогда $n_k \leq n < n_{k + 1}$ и
    \begin{equation*}
      S_n = S_{n_k} + x_{n_{k + 1}} + x_{n_{k + 2}} + \dotsc + x_n
    \end{equation*}
    Пусть все $x_i$ при $i \in (n_k, n_{k + 1}]$(группа, в которой мы находимся) неотрицательны. Тогда $S_n \geq S_{n_k}$. Аналогично получаем, что $S_n \leq S_{n_{k + 1}}$. Тогда $S_n$ лежит между $S_{n_k}$ и $S_{n_{k + 1}}$, каждое из которых стремится к $S$. Значит по теореме о двух миллиционерах $S_n$ тоже стремится к~$S$.
  \end{proof}
\end{enumerate}

\begin{example}
  Посмотрим на ряд $\sum\limits_{n = 1}^{\infty} \frac{(-1)^{[\sqrt[3]{n}]}}{\sqrt{n}}$. Поделим его на блоки следующего вида:
  \begin{equation*}
    (-1)^k \left(
      \frac{1}{\sqrt{k^3}} + \frac{1}{\sqrt{k^3 + 1}} + \dotsb + \frac{1}{\sqrt{(k + 1)^3 - 1}}
    \right)
  \end{equation*}
  Тогда заметим, что $(k + 1)^3 - k^3 \geq 3k^2$, а значит
  \begin{equation*}
    |\text{сумма в блоке}|
    \geq
    \frac{3k^2}{\sqrt{(k + 1)^3 - 1}} > 1
  \end{equation*}
  Таким образом сгрупированный ряд расходится, потому что не выполняется необходимое условие сходимости. А значит и наш ряд расходится, так как сгрупированный ряд это просто подпоследовательность нашего.
\end{example}
