\section{Производная по направлению.Экстремальное свойство градиента.}


\begin{conj}
    Пусть $\norm{h} = 1$, $f : E \to \R$, где $E \subset \R^n$.
    Тогда \textbf{производная по направлению}
    $$\frac{df}{dh}(a) := \lim_{t \to 0} \frac{f(a + th) - f(a)}{t}$$
\end{conj}

\textbf{Замечания:}
\begin{enumerate}
    \item Если $f$ дифференцируема в точке $a$, то 
    $\frac{df}{dh}(a) = d_a f (h)$.

    \item Пусть $g(t) := f(a + th)$, где $t \in \R$ -- 
    маленькое число настолько, что $(a + th) \in E$.

    Поясняющая картинка:

    \begin{center}
        \begin{tikzpicture}
            \draw[red, thick, ->] (1, 2) -- (1.5, 3);
            \node[red, text width=3cm] at (2.9,2.3) {h};
            \draw[thick, -, rounded corners=2mm] (0,0) \irregularcircle{2cm}{2mm};
            \node[text width=3cm] at (0.5,1) {E};
            \draw[thick, dashed] (0,-1) circle(0.7);
            \fill[thick, black] (0,-1) circle(0.07) node[below right]{a};
            \draw[thick, red] (0,-1) node[above left]{th};
            \draw[red, thick, <->] (-0.3, -1.6) -- (0.3, -0.4) ;
        \end{tikzpicture}
    \end{center}

    Пояснение к картинке: 
    $a \in \Int E \Rightarrow \exists r > 0 : B_r(a) \subset E$.
    $\norm{th} = \abs{t} < r$. \\
    \textit{Мы берём вектор направления $h$ и сдвигаемся на небольшие
    числа $t$ так, чтобы оставаться в окрестности $a$}.

    Получаем в итоге:
    $$\frac{df}{dh}(a) = g'(0) = \lim_{t \to 0} \frac{g(t) - g(0)}{t}$$
    \textit{Получается, что $g' : (-r, r) \to \R$ -- обычная числовая
    функция, значение и свойства её производной мы знаем.}

    \item 
    $f$ дифф. в точке $a$ $\Rightarrow$ $\exists \frac{df}{dh}(a)$,
    \textbf{но}
    $\exists \frac{df}{dh}(a)$ $\not\Rightarrow$ $f$ дифф. в точке $a$.

    Пример:

    Рассмотрим $f : \R^2 \to \R$. Пусть $C$ -- точки некоторой 
    окружности, проходящей через ноль. Пусть:
    $$ f(x) = 
    \begin{cases}
        1 & \text{если } x \in C \setminus \{0\} \\
        0 & \text{иначе}
    \end{cases}
    $$
    Тогда для любого $h \in \R^2$, $\norm{h} = 1$ и $t \in \R$,
    так что $(a + th) \in U(a)$,  
    прямая проходящая
    через $0$ и $th$ имеет с $C \setminus \{0\}$ не более одной точки
    пересечения, причём эта точка $\neq 0$. Значит, на некоторой 
    окрестности вдоль этой прямой $f(x) = 0$. 
    А значит $\frac{df}{dh}(0) = 0$.

    С другой стороны, $f$ не непрерывна в нуле, т.к. на сколько угодно
    маленьком расстоянии найдётся точка $x$, в которой $f(x) = 1$.
    А $f(0) = 0$, значит, $f$ не дифференцируема в $0$.
    
\end{enumerate}

\begin{theorem}
    Пусть $f : E \to \R$, $E \subset \R^n$, $a \in \Int E$, 
    $f$ -- дифф. в точке $a$.
    Тогда
    $$ \frac{df}{dh} (a) = d_a f(h) = \langle \nabla f(a), h \rangle $$
\end{theorem}
\begin{proof}
    Доказано ранее по отдельности.
\end{proof}

\textbf{Следствие 1:}
\begin{enumerate}
    \item[] \textbf{Экстремальное свойство градиента}.
    
    $f : E \to \R$, $a \in \Int E$, $f$ дифф. в точке $a$ и 
    $\nabla f(a) \neq 0$. Тогда 
    $$\forall h : \norm{h} = 1 \quad
    -\norm{\nabla f(a)} \leqslant \frac{df}{dh}(a) 
    \leqslant \norm{\nabla f(a)}$$
    И неравенство обращается в равенство лишь при 
    $h = \frac{\pm \nabla f(a)}{\norm{\nabla f(a)}}$.

    \textit{По смыслу получается, что производная по направлению --
    это скорость изменения функции в данном направлении, а 
    градиент -- это направление, в котором функция меняется 
    быстрее всего. На этом важном факте много что построено,
    в частности некоторые алгоритмы оптимизации функций основываются
    на этой идее: мы считаем градиент в точке и движемся по градиенту,
    т.е. по направлению,
    в котором функция быстрее всего убывает. Не факт, что мы получим
    глобальный минимум, но мы получим какой-то локальный минимум.}

    \begin{proof}
        $$\abs{\frac{df}{dh}(a)} = \abs{\langle \nabla f(a), h \rangle}
        \stackrel{\text{КБШ}}{\leqslant} \norm{\nabla f(a)} 
        \cdot \norm{h} = \norm{\nabla f(a)}$$

        Вспомним, что в неравенстве Коши-Буняковского достигается
        равенство тогда и только тогда, когда вектора сонаправлены.
        Получаем, что для достижения равенства надо взять единичный 
        вектор в направлении $\nabla f(a)$, значит, вектор $\nabla f(a)$
        нужно просто нормировать.
    \end{proof}
\end{enumerate}
