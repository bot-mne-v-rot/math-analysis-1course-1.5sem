\section{Определение $e^z$, $\sin z$ и $\cos z$. Ряды Тейлора для $\ln(1 + x)$ и $\arctg x$}
\underline{Разложение элементарных функций в ряд Тейлора:}

\begin{align*}
    1. \; e^x &= \sumi \frac{x^n}{n!} & 2. \; \cos{x} &= \sumi \frac{(-1)^n x^{2n}}{(2n)!} & 3. \; \sin{x} = \sumi \frac{(-1)^n x^{2n+1}}{(2n+1)!}
\end{align*}
Для вещественных чисел эти формулы были получены еще в первом семестре через формулу Тейлора с остатком в форме Лагранжа.
Теперь мы можем заметить, что раз ряды справа сходятся при всех $x \in \R$, то их радиус сходимости равен $+\infty$, а значит ряды сходятся на всем $\C$ $\Rightarrow$ мы также можем определить $e^z, \cos{z}$ и $\sin{z}$ для $z \in \C$.
И тогда станет понятно, откуда растут ноги у формулы Эйлера:
\begin{gather*}
    e^{iz} = \cos{z} + i\sin{z}
\end{gather*}
(Если расписать через ряды, то можно проверить, что слева и справа получится почленно одно и то же).
Верен и следующий набор формул ($1$ и $2$ доказываются аналогично формуле Эйлера, $3$ и $4$~--- прямое следствие из нее):
\begin{align*}
    1. \; e^{z+w} &= e^z \cdot e^w & 2. \; \sin^2{z} + \cos^2{z} &= 1 & 3. \; \sin{z} &= \frac{e^{iz} - e^{-iz}}{2i} & 4. \; \cos{z} &= \frac{e^{iz} + e^{-iz}}{2}
\end{align*}
Заведем еще одно разложение:
\begin{gather*}
    4. \; \ln(1+x) = \sum\limits_{n=1}^\infty \frac{(-1)^{n-1}x^n}{n} \text{ при } x \in (-1, 1)
\end{gather*}
\begin{proof}
    Начнем с того, что:
    \begin{gather*}
        \frac{1}{1 + t} = \sumi (-1)^n t^n \text{ сходится при } \abs{t} < 1
    \end{gather*}
    Тогда можем проинтегрировать такую штуку в круге сходимости:
    \begin{gather*}
        \ln(1 + x) = \int\limits_0^x \frac{dt}{1+t} = \int\limits_0^x \sumi (-1)^n t^n dt = \sumi \int\limits_0^x (-1)^n t^n dt = 
        \sum\limits_{n=0}^\infty \frac{(-1)^{n}x^{n+1}}{n+1}
    \end{gather*}
\end{proof}
Разложим что нибудь еще, нам же абсолютно нечем заняться:
\begin{gather*}
    5. \; \arctg{x} = \sumi \frac{(-1)^n x^{2n+1}}{2n+1} \text{ при } x \in (-1, 1]
\end{gather*}
В частности:
\begin{gather*}
    \frac{\pi}{4} = 1 - \frac{1}{3} + \frac{1}{5} - \frac{1}{7} + \frac{1}{9} - \dots 
\end{gather*}
\begin{proof}
    \begin{gather*}
        \frac{1}{1+t^2} = \sumi (-1)^n t^{2n} \text{ сходится при } \abs{t} < 1 \\
        \arctg{x} = \int\limits_{0}^x \frac{dt}{1+t^2} = \sumi \int\limits_{0}^x (-1)^n t^{2n} dt = \sumi \frac{(-1)^n x^{2n+1}}{2n+1}
    \end{gather*}
    Теорема Абеля: Если ряд сходится в 1, то его сумма $ = \lim\limits_{x \rightarrow 1-} \sum a_n x^n$. Тогда:
    \begin{gather*}
        \sumi \frac{(-1)^n}{2n+1} = \lim\limits_{x \rightarrow 1-} \sumi \frac{(-1)^n x^{2n+1}}{2n+1} = \lim\limits_{x \rightarrow 1-} \arctg{x} = \arctg{1} = \frac{\pi}{4}
    \end{gather*}
\end{proof}