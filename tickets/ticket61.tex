\section{Эквивалентные определения нормы оператора}
Заведем еще парОчку (пять) способов посчитать норму. 
\begin{theorem}
    \begin{align}
        \sup\limits_{\norm{x}_X \leqslant 1} \norm{\A x}_Y &= \\
        &= \sup\limits_{\norm{x}_X < 1} \norm{\A x}_Y \\
        &= \sup\limits_{\norm{x}_X = 1} \norm{\A x}_Y \\
        &= \sup\limits_{x \neq 0} \frac{\norm{\A x}_Y}{\norm{x}_X} \\
        &= \inf{\{ C \in \R : \forall x \in X \quad \norm{\A x}_Y \leqslant C \norm{x}_X \}}
    \end{align}
\end{theorem}
\newcommand{\tikzmark}[1]{\tikz[overlay,remember picture] \node (#1) {};}
\newcommand{\DrawBox}[1]{%
  \begin{tikzpicture}[overlay,remember picture]
    \draw[->,shorten >=5pt,shorten <=5pt,out=50,in=100,distance=0.7cm,#1, thick] ($(MarkA.north)+(.4em,.4em)$) to (MarkB.north);
    \draw[overlay,red,thick,dashed] ($(MarkA)!.5!(MarkABot)+(.4em,.2em)$) ellipse (0.6cm and 0.8cm);
  \end{tikzpicture}
}
\begin{proof} \quad 

    Очевидно, что $(1) \geqslant (2)$ и $(1) \geqslant (3)$ так как в $N_1$ множество больше.

    $(3) = (4)$:
    \begin{gather*}
        \frac{\norm{\A x}_Y}{\norm{x}_X} = \frac{1\tikzmark{MarkA}}{\tikzmark{MarkABot}\norm{x}_X} \cdot \norm{\A \tikzmark{MarkB} x}_Y\DrawBox{red} =
        \norm{\A \bigg( \underbrace{\colorboxed{red}{\frac{x}{\norm{x}_X}}}_{\text{единичн. век.}}\bigg)} \leqslant (3)
    \end{gather*}
    Дробь под оператор занесли по линейности. Так как под операторов получили единичный вектор, то это подмножество $(3)$. 
    Рассматривая в $(4)$ только иксы подходящие в третьем получаем обратное неравенство. Значит выполняется равенство.

    $(4) = (5)$:

    $(4) = $ наименьшая верхняя граница для $\frac{\norm{\A x}_Y}{\norm{x}_X}$. Переформулируем и распишем:
    \begin{gather*}
        (4) = \min \{ C : \frac{\norm{\A x}_Y}{\norm{x}_X} \leqslant C \} = \min \{ C : \norm{\A x}_Y \leqslant C \norm{x}_X\}
    \end{gather*}
    Получили $(5)$, но по $x \neq 0$, так как это было сказано в $(4)$, но, для $x = 0$ неравенство в принципе всегда выполняется, так что все хорошо.

    $(3) \geqslant (1)$:

    \begin{align*}
        \norm{\A x}_Y &= \norm{\A \left( \frac{x}{\norm{x}_X} \cdot \norm{x}_X \right)} \\
        &= \norm{x}_X \cdot \norm{\A \left( \frac{x}{\norm{x}_X} \right)} \text{ по линейности} \\
        &\leqslant \oast
    \end{align*}
    $\oast \leqslant \norm{x}_X \cdot (3)$, так как в $(3)$ тоже единичный вектор, но там стоит значек супремума. А это в свою очередь 
    не превосходит $(3)$, так как норма $x$ не превосходит единицы.

    $(2) \geqslant (1)$:

    Берем $x$ по норме не превосходящие 1. Тогда:
    \begin{gather*}
        \norm{(1 - \varepsilon)x}_X \leqslant 1 - \varepsilon < 1 \Longrightarrow \stackbelow{\underbrace{\sup\limits_{\norm{x}_X \leqslant 1} \norm{\A ((1-\varepsilon)x)}_Y}}{\oast} \leqslant (2)
    \end{gather*}
    Вынесем $(1 - \varepsilon)$ совсем наружу, потому что это вообще положительная константа и мы так можем. Тогда получим:
    \begin{gather*}
        \oast = (1 - \varepsilon) \sup\limits_{\norm{x}_X \leqslant 1} \norm{\A x}_Y = (1 - \varepsilon) \cdot (1) \\
        \Longrightarrow (1 - \varepsilon) \cdot (1) \leqslant (2) \quad \forall \varepsilon > 0
    \end{gather*}
    Тогда устремляем $\varepsilon$ к нулю и получим нужное неравенство.
\end{proof}
\textbf{Следствия: }
\begin{enumerate}
    \item $\norm{\A x}_Y \leqslant \norm{\A} \cdot \norm{x}_X$
    \item $\norm{\A \B} \leqslant \norm{\A} \cdot \norm{\B}$
\end{enumerate}
\begin{proof} \quad 

    \begin{enumerate}
        \item \begin{gather*}
            \norm{\A} = \sup\limits_{x \neq 0} \frac{\norm{\A x}_Y}{\norm{x}_X} \Longrightarrow \norm{\A x}_Y \leqslant \norm{\A} \cdot \norm{x}_X \text{ при } x \neq 0
        \end{gather*}
        \item Распишем норму $\A \B$:
        \begin{gather*}
            \norm{\A \B} = \sup\limits_{x \neq 0} \frac{\norm{\A \B x}_Z}{\norm{x}_X} \leqslant \oast
        \end{gather*}
        По определению: $\B : X \longrightarrow Y, \A : Y \longrightarrow Z$. Тогда:
        \begin{gather*}
            \norm{\A \B x}_Z \leqslant \norm{\A} \cdot \norm{\B x}_Y \leqslant \norm{\A} \cdot \norm{\B} \cdot \norm{x}_X
        \end{gather*}
        Подставим оценОчку:
        \begin{gather*}
            \oast \leqslant \sup\limits_{x \neq 0} \frac{\norm{\A} \cdot \norm{\B} \cdot \cancel{\norm{x}_X}}{\cancel{\norm{x}_X}} = \norm{\A} \cdot \norm{\B}
        \end{gather*}
    \end{enumerate}
\end{proof}