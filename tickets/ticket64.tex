\section{Примеры дифференцируемых отображений. Дифференцируемость координатных функций}

\textbf{Примеры:}
\begin{enumerate}
    \item $f = \operatorname{const}$. Тогда $f(a + h) = f(a)$.
    А значит, $T \equiv 0$, откуда, $\nabla f = 0$.
    \item $f$ -- линейное отображение. Тогда $f(a + h) = f(a) + f(h)$
    $\Rightarrow Th = f(h)$. Таким образом, дифференциал $f$ во всех
    точках -- это само $f$.
    А матрица Якоби -- это матрица отображения $f$.
\end{enumerate}

\begin{theorem}
    Пусть $f : E \to \R^m$, $E \subset \R^n$, $a \in \Int E$;
    $f = \begin{pmatrix*}
        f_1 \\ \vdots \\ f_m
    \end{pmatrix*}$, где $f_j : \R^n \to \R$ -- коорд. функция.
    Тогда $f$ дифференцируема в точке $a$ $\Longleftrightarrow$
    $\forall j \;\; f_j$ дифференцируема в точке $a$.
\end{theorem}
\begin{proof} $ $
    \begin{itemize}
        \item[``$\Longrightarrow$'':]

        Запишем определение дифференцируемости $f$:
        $$f(a + h) = f(a) + Th + \alpha(h) \norm{h}$$ 
        где $\alpha(h) \to 0$ при $h \to 0$ 
        (это просто по определению $o(\dots)$).

        Запишем равенство покоординатно: 
        $$f_j(a + h) = f_j(a) + T_j h +
        \alpha_j(h) \norm{h}$$ 
        $\Rightarrow f_j$ дифф. в точке $a$.

        \item[``$\Longleftarrow$'':]

        $f_j(a + h) = f_j(a) + T_j h + \alpha_j(h) \norm{h}$
        $$
            f = \begin{pmatrix*}
                f_1(a + h) \\
                \vdots \\
                f_m(a + h)
            \end{pmatrix*}
            = \begin{pmatrix*}
                f_1(a) \\
                \vdots \\
                f_m(a)
            \end{pmatrix*}
            + \begin{pmatrix*}
                T_1 h \\
                \vdots \\
                T_m h
            \end{pmatrix*}
            + \begin{pmatrix*}
                \alpha_1(h) \\
                \vdots \\
                \alpha_m(h)
            \end{pmatrix*} \norm{h}
        $$

        Надо доказать, что $\alpha(h) \to 0$ при $h \to 0$. \\
        Док-во:
        $$ \lim_{h \to 0} \norm{\alpha(h)}
        = \lim_{h \to 0} \norm{\begin{pmatrix*}
            \alpha_1(h) \\
            \vdots \\
            \alpha_m(h)
        \end{pmatrix*}}
        = \lim_{h \to 0} \sqrt{\alpha_1(h)^2 + \dots + \alpha_m(h)^2)}
        = \sqrt{\sum_{i = 1}^m \lim_{h \to 0} \alpha_i(h)^2} = 0
        $$

    \end{itemize}
\end{proof}

\follow $ $ транспонированные строки матрицы Якоби -- 
градиенты координатных функций.


