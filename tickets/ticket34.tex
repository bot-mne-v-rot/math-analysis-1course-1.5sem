\section{Перестановка членов абсолютно сходящегося ряда}
\newcommand{\sumn}{\sum \limits_{n=1}^\infty}
\newcommand{\sumk}{\sum \limits_{k=1}^n}
\newcommand{\prodn}{\prod \limits_{n=1}^\infty}
\newcommand{\prodk}{\prod \limits_{k=1}^n}

\begin{conj}

    $\varphi: \N \to \N$ биекция - перестановка членов ряда

    $\sum \limits_{n=1}^\infty a_{\varphi(n)}$
     - ряд с переставленными членами
\end{conj}

\begin{theorem}
    $a_n \in \C$ и $\sumn a_n$ абсолютно сходится, $S = \sumn a_n$
    Тогда:
    \begin{gather*}
        \sumn a_{\varphi(n)} = S
    \end{gather*}
    Другими словами, если ряд абсолютно сходится, то не важно, в каком порядке мы берем слагамые
\end{theorem}

\begin{proof} \quad 

    \begin{enumerate}
        \item[Шаг 1.] 
        \[a_n \geq 0,\ S_n = \sum \limits_{k=1}^n a_k, \quad \widetilde{S_n} = \sum \limits_{k=1}^n a_{\varphi(n)} \]
        Ряд $\sumn a_n$ сходится $\Rightarrow S_n \leq S$
        \[\widetilde{S_n} = a_{\varphi(1)} + a_{\varphi(2)} + \ldots + a_{\varphi(n)}\]
        Если докинуть к этой сумме еще какие-то $a$-шки, то сумма может только увеличиться, потому что они неотрицательны.

        \[ \widetilde{S_n} = a_{\varphi(1)} + a_{\varphi(2)} + \ldots + a_{\varphi(n)} \leq S_{\max\{\varphi(1), \varphi(2), \ldots, \varphi(n)\}} \leq S\]

        $\widetilde{S_n}$ монотонно возрастает и ограничена сверху $\Rightarrow$ имеет предел $\widetilde{S} \leq S$

        Следовательно, никакая перестановка членов ряда не увеличивает сумму. Но значит она не меняет сумму.
        Действительно, посмотрим на обратную перестановку (это будет просто $S$) - она тоже не увеличивает сумму. Значит сумма сохранялась.
        \item[Шаг 2.]
        При $a_n \in \R$

        Обозначим за $(a_n)_{+} := \max\{a_n, 0\},\ (a_n)_- := \max\{-a_n, 0\}$. Тогда:
        \begin{gather*}
            a_n = (a_n)_+ - (a_n)_-, \qquad \abs{a_n} = (a_n)_+ + (a_n)_-
        \end{gather*}
    
        Из того, что $0 \leq (a_n)_\pm \leq |a_n|$, следует, что ряд $\sum_{n=1}^\infty (a_n)_\pm$, сходятся по признаку сравнения, а из этого, благодаря первому шагу, следует, что:
        \begin{gather*}
            \sumn(a_{\varphi(n)})_\pm = \sumn(a_n)_{\pm}
        \end{gather*}
        Ну а теперь, пользуясь тем, что $a_n = (a_n)_+ - (a_n)_-$, получаем, что сумма переставленного ряда -- то же самое, что и сумма исходного ряда: 
        \begin{align*}
            \sumn a_{\varphi(n)} &= \sumn ((a_{\varphi(n)})_+ - (a_{\varphi(n)})_-) \\
            &= \sumn (a_{\varphi(n)})_+ - \sumn (a_{\varphi(n)})_- \\
            &= \sumn (a_n)_+ - \sumn (a_n)_- = \sumn a_n 
        \end{align*}
        \item[Шаг 3.]
        При $a_n \in \C$
        \begin{gather*}
            \abs{\operatorname{Re} a_n},\ \abs{\operatorname{Im}  a_n} \leqslant \abs{a_n} \Longrightarrow \sum \operatorname{Re} a_n,\ \sum \operatorname{Im}  a_n \text{ абсолютно сходятся}  
        \end{gather*}
        В них можем переставлять члены как хотим, потом соберем их обратно и напишем сумму (мнимую часть надо домножить на $i$).
    \end{enumerate}
\end{proof}

\textbf{Замечания}
\begin{enumerate}
    \item[1.] Теорема верна в любом нормированном пространстве, в частности в $\R^d$.
    Для $\R^d$ можно расписать покоординатно и повторить рассуждение, общий случай оставим без доказательства
    \item[2.] В $\R^d$ верно и обратное - если у ряда любая перестановка сходится (даже не важно, что к той же сумме), то он абсолютно сходится,
    и, значит, все суммы равны. В $\R^d$ мы это не докажем, поймем только для $\R$
    \item[3.] Перестановка расходящегося ряда с $a_n \geq 0$ - расходящийся ряд.
    \begin{proof}
        Если некоторая перестановка сходится, то к ней можно применить предыдущую теорему и получится, что изначальный ряд тоже сходится - противоречие.
    \end{proof} 
    \item[4.] Если $\sum a_n$ сходится, но не абсолютно, то ряды $\sum (a_n)_+$ и $\sum (a_n)_-$ расходятся (имеют бесконечную сумму).  
    \begin{proof}
        Предположим, что $\sum(a_n)_+$ сходится. 
        \begin{gather*}
            (a_n)_- = (a_n)_+ - a_n \Longrightarrow \\
            \sum (a_n)_- = \underbrace{\sum (a_n)_+}_{\text{сх-ся}} - \underbrace{\sum a_n}_{\text{сх-ся}} \Longrightarrow \\
            \sum(a_n)_- \text{ тоже сходится}
        \end{gather*}
        Также:
        \begin{gather*}
            |a_n| = (a_n)_+ + (a_n)_- \Longrightarrow \\ 
            \sum |a_n| = \underbrace{\sum (a_n)_+}_{\text{сх-ся}} + \underbrace{\sum (a_n)_-}_{\text{сх-ся}} \Longrightarrow \\
            \sum |a_n| - \text{ сходится}
        \end{gather*}
        Противоречие.
    \end{proof}
\end{enumerate}

\begin{conj}
    Если $\sum a_n$ сходится, но не абсолютно, то $\sum a_n$ условно сходящийся.
\end{conj}