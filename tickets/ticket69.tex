\section{Теорема Лагранжа для векторнозначных функций}

\begin{theorem}[теорема Лагранжа для векторнозначных функций] $ $\\
    Пусть $f : [a, b] \to \R^m$, $f$ непрерывна и
    дифф. на $[a, b]$. \\ Тогда $\exists c \in (a, b)$,
    т. ч. $\norm{f(b) - f(a)} \leqslant \norm{f'(c)}(b - a)$.
\end{theorem}
\begin{proof}
    Пусть $\varphi : [a, b] \to \R^m$,
    $\varphi(x) := \langle f(x), f(b) - f(a) \rangle$. Она непрерывна
    и дифф. на $(a, b)$, т.к. просто является произведением константы
    на непр. и дифф. на $(a, b)$ функцию.

    По замечанию к пред. теореме:
    $$\varphi'(x) = \langle f(x), f(b) - f(a) \rangle' =
    \langle f'(x), f(b) - f(a) \rangle +
    \langle f(x), \underbrace{(f(b) - f(a))'}_{= 0} \rangle =
    \langle f'(x), f(b) - f(a) \rangle$$

    Применим одномерную теорему Лагранжа:
    $$\exists c \in (a, b) : \varphi(b) - \varphi(a) =
    \varphi'(c)(b - a) = 
    (b - a) \langle f'(c), f(b) - f(a) \rangle$$

    С другой стороны:
    $$\varphi(b) - \varphi(a) = \langle f(b), f(b) - f(a) \rangle
    - \langle f(a), f(b) - f(a) \rangle = \norm{f(b) - f(a)}^2$$

    Собираем вместе:
    $$\norm{f(b) - f(a)}^2 = 
    (b - a) \langle f'(c), f(b) - f(a) \rangle
    \stackrel{\text{КБШ}}{\leqslant} (b - a) \norm{f'(c)}
    \cdot \norm{f(b) - f(a)}$$

    Если $f(b) = f(a)$, то неравенство в условии очевидно выполняется,
    иначе можно сократить на $\norm{f(b) - f(a)}$.
\end{proof}
\textbf{Замечание.} Равенство $\norm{f(b) - f(a)} = \norm{f'(c)}(b - a)$
может не достигаться ни в одной точке.

\textbf{Пример:}

Пусть $f(x) = (\cos x, \sin x)$, $x \in [a, b] = [0, 2\pi]$.
Эта функция дифф., т.к. дифф. эквивалентна дифф. коорд. функций.
\begin{gather*}
    f(b) - f(a) = f(2\pi) - f(0) = (0, 0) \Rightarrow
    \norm{f(b) - f(a)} = 0, \\
    f'(x) = (-\sin x, \cos x) \Rightarrow \norm{f'(x)}^2 =
    (-\sin x)^2 + \cos^2 x = 1 \Rightarrow \norm{f'(x)} \equiv 1 \\
    \norm{f(b) - f(a)} = 0 < \norm{f'(x)}(b - a) = 2\pi
\end{gather*}