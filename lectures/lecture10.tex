\section{Лекция номер 10}

\notice $ $ В $\R^n$ мы подразумеваем обычную евклидову норму,
если не оговорено иное.

\begin{theorem}
    Пусть $A : \R^n \to \R^m$ -- линейный оператор. \\
    Тогда $\norm{A}^2 \leqslant \sum_{i = 1}^m \sum_{k = 1}^n a_{ik}^2$,
    где $a_{ik}$ -- коэффициенты соотв. матрицы линейного оператора
    в стандартных базисах. \\
    В частности, $A$ -- ограниченный оператор.
\end{theorem}
\begin{proof} $ $
    
    Вспомним:
    $$ 
    A = \begin{pmatrix*}
        a_{11} & a_{12} & \dots & a_{1n} \\
        a_{21} & a_{22} & \dots & a_{2n} \\
        \vdots & \vdots & \ddots & \vdots \\
        a_{m1} & a_{m2} & \dots & a_{mn} \\
    \end{pmatrix*}
    \quad
    x = \begin{pmatrix*}
        x_1 \\ \vdots \\ x_n
    \end{pmatrix*}
    $$
    Тогда:
    $$
    \norm{Ax} = 
    \norm{\begin{pmatrix*}
        \sum_{k=1}^n a_{1k} x_k \\ 
        \vdots \\ 
        \sum_{k=1}^n a_{mk} x_k \\
    \end{pmatrix*}}^2 =
    \sum_{i = 1}^m \underbrace{\left(\sum_{k = 1}^n a_{ik} x_k \right)^2}
    _{\stackrel{\text{КБШ}}{\leqslant} 
    \sum \limits_{k = 1}^n a_{ik}^2 \cdot
    \sum \limits_{k = 1}^n x_{k}^2 }
    \leqslant \sum_{i = 1}^m \sum_{k = 1}^n a_{ik}^2 \cdot
    \sum_{k = 1}^n x_{k}^2
    = \sum_{i = 1}^m \sum_{k = 1}^n a_{ik}^2 \cdot \norm{x}
    $$
    По определению: 
    $$\norm{A} = \sup_{\norm{x} \leqslant 1}
    \norm{Ax} \leqslant \sqrt{\sum_{i = 1}^m \sum_{k = 1}^n a_{ik}^2}$$
\end{proof}

\notice $ $ в бесконечномерных пространствах бывают 
неограниченные операторы.

\subsection{Дифференцируемые отображения}

\begin{conj} $ $\\
    Пусть $f : E \to \R^m$, $E \subset \R^n$, $a \in \Int E$. \\
    $f$ \textbf{дифференцируемо} в точке $a$, если существует
    линейное отображение \\ $T : \R^n \to \R^m$, т.ч.
    $f(a + h) = f(a) + Th + o(\norm{h})$ при $h \to 0$.
\end{conj}

\begin{conj}
    $T$ -- \textbf{дифференциал} $f$ в точке $a$. Обозначается $d_a f$.
\end{conj}

$ $

\begin{theorem-non}
    Если $T$ существует, то оно единственно.
\end{theorem-non}
\begin{proof} $ $ \\
    Пусть $h \in \R^n$.
    Возьмём из головы формулу и поймём, что она верная:
    $$\lim_{t \to 0} \frac{f(a + th) - f(a)}{t} =
    \lim_{t \to 0} \frac{f(a) + T(th) + o(\norm{th}) - f(a)}{t} =
    \lim_{t \to 0} \frac{tTh + o(\norm{th})}{t} = Th$$
    Таким образом, если $T$ существует, то для любого $h$
    можно однозначно вычислить $Th$.
\end{proof}

\begin{conj}
    Матрица оператора $T$ -- \textbf{матрица Якоби} функции 
    $f$ в точке $a$. \\
    Обозначается $f'(a)$.
\end{conj}

\begin{theorem-non}
    Если $f$ дифференцируема в точке $a$, то $f$ непрерывна в точке $a$.
\end{theorem-non}
\begin{proof}
    $$\lim_{h \to 0} f(a + h) = \lim_{h \to 0}(f(a) + Th + o(h))
    = f(a) + \lim_{h \to 0} Th = f(a)$$
    Так как $\norm{Th} \leqslant \norm{T} \norm{h}$.
\end{proof}

\textbf{Примеры:}
\begin{enumerate}
    \item $f = \operatorname{const}$. Тогда $f(a + h) = f(a)$.
    А значит, $T \equiv 0$, откуда, $\nabla f = 0$.
    \item $f$ -- линейное отображение. Тогда $f(a + h) = f(a) + f(h)$
    $\Rightarrow Th = f(h)$. Таким образом, дифференциал $f$ во всех
    точках -- это само $f$. 
    А матрица Якоби -- это матрица отображения $f$.
\end{enumerate}

$ $

\textbf{Важный частный случай:} $m = 1$. \\
Тогда $T : \R^n \to \R$. И матрица $T$ =
$\begin{pmatrix*}
    t_1 & \dots & t_n
\end{pmatrix*}$.
Пусть $h = \begin{pmatrix*}
    h_1 \\ \vdots \\ h_n
\end{pmatrix*}$. \\
Тогда $Th = \sum_{k = 1}^n t_k h_k$. 
Это по сути скалярное произведение.

$ $

\begin{conj}
    Пусть $f : \R^n \to \R$. $f$ дифференцируема в точке $a$. \\
    Тогда найдётся $v \in \R^n$, т.ч. $f(a + h) = f(a) + 
    \langle v, h \rangle + o(\norm{h})$ при $h \to 0$. \\
    $v$ --- \textbf{градиент} функции $f$ в точке $a$.
    Обозначается $\operatorname{grad} f$ или $\nabla f$ (``набла'' $f$).
\end{conj}

\begin{theorem}
    Пусть $f : E \to \R^m$, $E \subset \R^n$, $a \in \Int E$;
    $f = \begin{pmatrix*}
        f_1 \\ \vdots \\ f_m
    \end{pmatrix*}$, где $f_j : \R^n \to \R$ -- коорд. функция.
    Тогда $f$ дифференцируема в точке $a$ $\Longleftrightarrow$
    $\forall j \;\; f_j$ дифференцируема в точке $a$.
\end{theorem}
\begin{proof} $ $
    \begin{itemize}
        \item[``$\Longrightarrow$'':]

        Запишем определение дифференцируемости $f$:
        $$f(a + h) = f(a) + Th + \alpha(h) \norm{h}$$ 
        где $\alpha(h) \to 0$ при $h \to 0$ 
        (это просто по определению $o(\dots)$).

        Запишем равенство покоординатно: 
        $$f_j(a + h) = f_j(a) + T_j h +
        \alpha_j(h) \norm{h}$$ 
        $\Rightarrow f_j$ дифф. в точке $a$.

        \item[``$\Longleftarrow$'':]

        $f_j(a + h) = f_j(a) + T_j h + \alpha_j(h) \norm{h}$
        $$
            f = \begin{pmatrix*}
                f_1(a + h) \\
                \vdots \\
                f_m(a + h)
            \end{pmatrix*}
            = \begin{pmatrix*}
                f_1(a) \\
                \vdots \\
                f_m(a)
            \end{pmatrix*}
            + \begin{pmatrix*}
                T_1 h \\
                \vdots \\
                T_m h
            \end{pmatrix*}
            + \begin{pmatrix*}
                \alpha_1(h) \\
                \vdots \\
                \alpha_m(h)
            \end{pmatrix*} \norm{h}
        $$

        Надо доказать, что $\alpha(h) \to 0$ при $h \to 0$. \\
        Док-во:
        $$ \lim_{h \to 0} \norm{\alpha(h)}
        = \lim_{h \to 0} \norm{\begin{pmatrix*}
            \alpha_1(h) \\
            \vdots \\
            \alpha_m(h)
        \end{pmatrix*}}
        = \lim_{h \to 0} \sqrt{\alpha_1(h)^2 + \dots + \alpha_m(h)^2)}
        = \sqrt{\sum_{i = 1}^m \lim_{h \to 0} \alpha_i(h)^2} = 0
        $$

    \end{itemize}
\end{proof}

\follow $ $ транспонированные строки матрицы Якоби -- 
градиенты координатных функций.

\begin{conj}
    Пусть $\norm{h} = 1$, $f : E \to \R$, где $E \subset \R^n$.
    Тогда \textbf{производная по направлению}
    $$\frac{df}{dh}(a) := \lim_{t \to 0} \frac{f(a + th) - f(a)}{t}$$
\end{conj}

\textbf{Замечания:}
\begin{enumerate}
    \item Если $f$ дифференцируема в точке $a$, то 
    $\frac{df}{dh}(a) = d_a f (h)$.

    \item Пусть $g(t) := f(a + th)$, где $t \in \R$ -- 
    маленькое число настолько, что $(a + th) \in E$.

    Поясняющая картинка:

    \begin{center}
        \begin{tikzpicture}
            \draw[red, thick, ->] (1, 2) -- (1.5, 3);
            \node[red, text width=3cm] at (2.9,2.3) {h};
            \draw[thick, -, rounded corners=2mm] (0,0) \irregularcircle{2cm}{2mm};
            \node[text width=3cm] at (0.5,1) {E};
            \draw[thick, dashed] (0,-1) circle(0.7);
            \fill[thick, black] (0,-1) circle(0.07) node[below right]{a};
            \draw[thick, red] (0,-1) node[above left]{th};
            \draw[red, thick, <->] (-0.3, -1.6) -- (0.3, -0.4) ;
        \end{tikzpicture}
    \end{center}

    Пояснение к картинке: 
    $a \in \Int E \Rightarrow \exists r > 0 : B_r(a) \subset E$.
    $\norm{th} = \abs{t} < r$. \\
    \textit{Мы берём вектор направления $h$ и сдвигаемся на небольшие
    числа $t$ так, чтобы оставаться в окрестности $a$}.

    Получаем в итоге:
    $$\frac{df}{dh}(a) = g'(0) = \lim_{t \to 0} \frac{g(t) - g(0)}{t}$$
    \textit{Получается, что $g' : (-r, r) \to \R$ -- обычная числовая
    функция, значение и свойства её производной мы знаем.}

    \item 
    $f$ дифф. в точке $a$ $\Rightarrow$ $\exists \frac{df}{dh}(a)$,
    \textbf{но}
    $\exists \frac{df}{dh}(a)$ $\not\Rightarrow$ $f$ дифф. в точке $a$.

    Пример:

    Рассмотрим $f : \R^2 \to \R$. Пусть $C$ -- точки некоторой 
    окружности, проходящей через ноль. Пусть:
    $$ f(x) = 
    \begin{cases}
        1 & \text{если } x \in C \setminus \{0\} \\
        0 & \text{иначе}
    \end{cases}
    $$
    Тогда для любого $h \in \R^2$, $\norm{h} = 1$ и $t \in \R$,
    так что $(a + th) \in U(a)$,  
    прямая проходящая
    через $0$ и $th$ имеет с $C \setminus \{0\}$ не более одной точки
    пересечения, причём эта точка $\neq 0$. Значит, на некоторой 
    окрестности вдоль этой прямой $f(x) = 0$. 
    А значит $\frac{df}{dh}(0) = 0$.

    С другой стороны, $f$ не непрерывна в нуле, т.к. на сколько угодно
    маленьком расстоянии найдётся точка $x$, в которой $f(x) = 1$.
    А $f(0) = 0$, значит, $f$ не дифференцируема в $0$.
    
\end{enumerate}

\begin{theorem}
    Пусть $f : E \to \R$, $E \subset \R^n$, $a \in \Int E$, 
    $f$ -- дифф. в точке $a$.
    Тогда
    $$ \frac{df}{dh} (a) = d_a f(h) = \langle \nabla f(a), h \rangle $$
\end{theorem}
\begin{proof}
    Доказано ранее по отдельности.
\end{proof}

\textbf{Следствие 1:}
\begin{enumerate}
    \item[] \textbf{Экстремальное свойство градиента}.
    
    $f : E \to \R$, $a \in \Int E$, $f$ дифф. в точке $a$ и 
    $\nabla f(a) \neq 0$. Тогда 
    $$\forall h : \norm{h} = 1 \quad
    -\norm{\nabla f(a)} \leqslant \frac{df}{dh}(a) 
    \leqslant \norm{\nabla f(a)}$$
    И неравенство обращается в равенство лишь при 
    $h = \frac{\pm \nabla f(a)}{\norm{\nabla f(a)}}$.

    \textit{По смыслу получается, что производная по направлению --
    это скорость изменения функции в данном направлении, а 
    градиент -- это направление, в котором функция меняется 
    быстрее всего. На этом важном факте много что построено,
    в частности некоторые алгоритмы оптимизации функций основываются
    на этой идее: мы считаем градиент в точке и движемся по градиенту,
    т.е. по направлению,
    в котором функция быстрее всего убывает. Не факт, что мы получим
    глобальный минимум, но мы получим какой-то локальный минимум.}

    \begin{proof}
        $$\abs{\frac{df}{dh}(a)} = \abs{\langle \nabla f(a), h \rangle}
        \stackrel{\text{КБШ}}{\leqslant} \norm{\nabla f(a)} 
        \cdot \norm{h} = \norm{\nabla f(a)}$$

        Вспомним, что в неравенстве Коши-Буняковского достигается
        равенство тогда и только тогда, когда вектора сонаправлены.
        Получаем, что для достижения равенства надо взять единичный 
        вектор в направлении $\nabla f(a)$, значит, вектор $\nabla f(a)$
        нужно просто нормировать.
    \end{proof}
\end{enumerate}

\begin{conj}
    Пусть $f : E \to \R$. $a \in \Int E$. Тогда 
    \textbf{частная производная} -- производная по направлению
    $k$-го базисного вектора (по напр. $k$-й координаты):
    $$\frac{df}{dx_k}(a) := \lim_{t \to 0} 
    \frac{f(a + te_k) - f(a)}{t}$$
    где $e_k$ -- $k$-й базисный вектор (нулевой вектор, 
    кроме 1 на $k$-й позиции).
    $$ 
    e_k = \begin{pmatrix*}
        0 \\ \vdots \\ 0 \\ 1 \\ 0 \\ \vdots \\ 0
    \end{pmatrix*}
    \begin{matrix*}
        $ $ \\ $ $ \\ $ $ \\
        \longleftarrow \text{$k$-е место}
        \\ $ $ \\ $ $ \\ $ $
    \end{matrix*}
    $$
    Обозначений много: $f'_k$, $D_k f$, $d_k f$, $\frac{df}{dx_k}$. 
\end{conj}
 
\textbf{Следствие 2:}
\begin{enumerate}
    \item[] \textbf{Каноническое определение градиента}.
    $$\frac{df}{dx_k} = \langle \nabla f(a), e_k \rangle =
    \text{$k$-я координата } \nabla f(a)$$
    Другими словами:
    $$ \nabla f(a) =
    \begin{pmatrix*}
        \frac{df}{dx_1}(a) \\
        \frac{df}{dx_2}(a) \\
        \vdots \\
        \frac{df}{dx_n}(a) \\
    \end{pmatrix*}$$
    \textit{На лекции Храбров записал вектор-строку, но мы в $\R^n$ 
    используем векторы-столбцы}.
\end{enumerate}

\textbf{Следствие 3:}
\begin{enumerate}
    \item[] 
    Если $f : E \to \R^m$, $E \subset \R^n$, $a \in \Int E$,
    $f$ дифф. в точке $a$, то матрица Якоби в точке $a$
    выглядит следующим образом:
    $$ f'(a) =
    \begin{pmatrix*}
        \frac{df_1}{dx_1}(a) & \frac{df_1}{dx_2}(a) &
        \dots & \frac{df_1}{dx_n}(a) \\
        \frac{df_2}{dx_1}(a) & \frac{df_2}{dx_2}(a) &
        \dots & \frac{df_2}{dx_n}(a) \\
        \vdots & \vdots & \ddots & \vdots \\
        \frac{df_m}{dx_1}(a) & \frac{df_m}{dx_2}(a) &
        \dots & \frac{df_m}{dx_n}(a) \\
    \end{pmatrix*}$$
    где $f_k$ -- $k$-я координатная функция $f$.
\end{enumerate}

\textbf{Пример:}

Пусть $f(x, y) = x^y$. Тогда
$$
\frac{df}{dx}(a, b) = 
\lim_{t \to 0} \frac{f((a, b) + t(1, 0)) - f(a, b)}{t} =
\lim_{t \to 0} \frac{f(a + t, b) - f(a, b)}{t} =
(x^b)' \mid_{x = a} = b a^{b-1}
$$
\textit{У нас меняется только первая координата,
т.е. по сути это определение производной в точке $x = a$ при
фиксированном $y = b$. Т.е. мы просто воспринимаем $y$ как параметр
и дифференцируем по одной переменной.}

$$
\frac{df}{dy}(a, b) = 
\lim_{t \to 0} \frac{f((a, b) + t(0, 1)) - f(a, b)}{t} =
\lim_{t \to 0} \frac{f(a, b + t) - f(a, b)}{t} =
(a^y)' \mid_{y = b} = a^b \cdot \ln a
$$

\textit{Получается, что, когда мы считаем частную производную по $x_k$,
мы фиксируем все остальные переменные, отличные от $x_k$ ---
считаем, что они константы ---
и просто получаем функцию от одной переменной.}

\begin{theorem}[линейность дифференциала] $ $

    Пусть 
    \begin{itemize}
        \item $f, g : E \to \R^m$, $E \subset \R^n$;
        \item $a \in \Int E$, $\lambda \in \R$;
        \item $f$ и $g$ дифф. в точке $a$.
    \end{itemize}
    Тогда
    \begin{itemize}
        \item $f + g$ дифф. в точке $a$ и $d_a(f + g) = d_a f + d_a g$;
        \item $\lambda f$ дифф. в точке $a$ и $d_a(\lambda f) = 
        \lambda d_a f$.
    \end{itemize}
\end{theorem}
\begin{proof} $ $
    
    По определению дифференцирумости:

    $f(a + h) = f(a) + d_a f(h) + \alpha(h) \norm{h}$, где
    $\alpha(h) \to 0$ при $h \to 0$. \\
    $g(a + h) = g(a) + d_a g(h) + \beta(h) \norm{h}$, где
    $\beta(h) \to 0$ при $h \to 0$.

    Тогда $f(a + h) + g(a + h) = f(a) + g(a) + 
    \underbrace{d_a f(h) + d_a g(h)}_{\text{лин. оператор}} +
    \overbrace{\underbrace{(\alpha(h) + \beta(h))}_{\to 0} \norm{h}}
    ^{= o(\norm{h})}$.

    Равенство по определению выполняется, значит, 
    $f + g$ дифф. в точке $a$ и $d_a(f+g) = d_a f + d_a g$.

    $\lambda f(a + h) = \lambda f(a) + 
    \underbrace{\lambda d_a f(h)}_{\text{лин. оп-р}} + 
    \overbrace{\lambda \alpha(h) \norm{h}}^{=o(\norm{h})}$.

    Равенство по определению выполняется, значит, 
    $\lambda f$ дифф. в точке $a$ и $d_a(\lambda f) = \lambda d_a f$.
\end{proof}
\textit{В понимании дифференциала как матриц сложение дифференциалов
и умножение дифференциала на скаляр просто соответствуют сложению
матриц и умножению матрицы на скаляр. Хотя искушённому жуковским линалом
читателю, это очевидно.}

\begin{theorem}[дифференцируемость композиции]
    Пусть
    \begin{itemize}
        \item $D \subset \R^n$, $f : D \to \R^m$;
        \item $E \subset \R^m$, $g : E \to \R^l$;
        \item $a \in \Int D$, $f(D) \subset E$, $f(a) \in \Int E$;
        \item $f$ дифф. в точке $a$; $g$ дифф. в точке $f(a)$.
    \end{itemize}
    Тогда $g \circ f$ дифф. в точке $a$ и 
    $d_a (g \circ f) = d_{f(a)} g \circ d_a f$.
\end{theorem}
\begin{proof} $ $

    По определению дифференцирумости:
    \begin{align*}
        f(a + h) &= \overbrace{f(a)}^{=: b} + 
        \overbrace{d_a f(h) + \alpha(h) \norm{h}}^{=: k}, &&
        \text{ где $\alpha(h) \to 0$ при $h \to 0$} \\
        g(b + k) &= g(b) + d_b g(k) + \beta(k) \norm{k}, &&
        \text{ где $\beta(k) \to 0$ при $k \to 0$}
    \end{align*}
    
    Тогда: $$g(f(a + h)) = g(b + k) = 
    g(b) + d_b g(k) + \beta(k) \norm{k}$$

    По линейности линейного отображения:
    $$d_b g(k) = d_b g(d_a f(h) + \alpha(h) \norm{h}) = 
    d_b g (d_a f(h)) + d_b g(\alpha(h)) \norm{h}$$

    Теперь нужно понять, что $d_b g(\alpha(h)) \norm{h} = o(\norm{h})$,
    т.е. $d_b g(\alpha(h)) \to 0$ при $h \to 0$:
    $$
    \norm{d_b g(\alpha(h))} \leqslant 
    \underbrace{\norm{d_b g}}_{\text{const}} \cdot 
    \underbrace{\norm{\alpha(h)}}_{\to 0 \text{ по опр.}} \to 0
    $$

    Теперь нужно доказать, что $\beta(k) \norm{k} = o(\norm{h})$,
    т.е. $\beta(k) \frac{\norm{k}}{\norm{h}} \to 0$ при $h \to 0$:
    $$\norm{k} = \norm{d_a f(h) + \alpha(h) \norm{h}}
    \leqslant
    \norm{d_a f(h)} + \norm{\alpha(h)} \cdot \norm{h}
    \leqslant \norm{d_a f} \cdot \norm{h} + 
    \norm{\alpha(h)} \cdot \norm{h}$$

    По опр. $\alpha(h) \to 0$ при $h \to 0$, значит,
    $\alpha(h)$ ограничена в окрестности 0, при этом 
    $\norm{d_a f} = \text{const}$, поэтому 
    $\norm{k} \leqslant M \norm{h}$ в окрестности 0. Поэтому 
    $\norm{k} \to 0$ при $h \to 0$. Из этого следует, что
    $\beta(k) \to 0$ при $h \to 0$, т.к. $k \to 0$. Получаем:
    $$\beta(k) \frac{\norm{k}}{\norm{h}} \leqslant M \cdot 
    \underbrace{\beta(k)}_{\to 0} \to 0$$

    В исходном равенстве получаем:
    $$g(f(a + h)) = g(b) + d_b g (d_a f(h)) + (d_b g(\alpha(h)) \norm{h}
    + \beta(k) \norm{k}) = g(b) + d_b g (d_a f(h)) + o(\norm{h})$$
    Отсюда по опр. дифф. получаем, что $g \circ f$ дифф. в точке $a$ и
    $(d_a (g \circ f)) (h) = d_b g (d_a f(h))$, а значит,
    $d_a (g \circ f) = d_{f(a)} g \circ d_a f$.

\end{proof}

\textbf{Следствие.} $(g \circ f)'(a) = g'(f(a)) \cdot f'(a)$

\textit{Очень знакомая формула, только теперь не с числами, 
а с матрицами}.
\begin{proof}
    $d_a(g \circ f) = d_{f(a)} g \circ d_a f$. 
    Матрица композиции линейных отображений -- произведение 
    матриц линейных отображении в том же порядке.
\end{proof}

\begin{theorem}[дифф. произведения скалярной и векторной функции] $ $

    Пусть 
    \begin{itemize}
        \item $E \subset \R^n$, $\lambda : E \to \R$, $f : E \to \R^m$;
        \item $a \in \Int E$, $f$ и $\lambda$ дифф. в точке $a$
    \end{itemize}
    Тогда $\lambda f$ дифф. в точке $a$ и $d_a (\lambda f)(h)
    = d_a \lambda (h) \cdot f(a) + \lambda(a) \cdot d_a f(h)$.
\end{theorem}
\begin{proof} $ $

    По определению дифференцирумости:
    \begin{align*}
        f(a + h) &= f(a) + d_a f(h) + \alpha(h) \norm{h}, &&
        \text{ где $\alpha(h) \to 0$ при $h \to 0$} \\
        \lambda(a + h) &= \lambda(a) + d_a \lambda(h) + \beta(h) \norm{h}, &&
        \text{ где $\beta(h) \to 0$ при $h \to 0$}
    \end{align*}

    Получаем:
    \begin{gather*}
        f(a + h) \cdot \lambda(a + h) - f(a) \cdot \lambda(a) = 
        \underbrace{d_a \lambda(h) \cdot f(a) + 
        \lambda(a) \cdot d_a f(h)}_{\text{то что нужно}} + \\ +
        \underbrace{\lambda(a) \cdot \alpha(h) \norm{h}}
        _{\text{очев. $=o(\norm{h})$}} + 
        \underbrace{f(a) \cdot \beta(h) \norm{h}}
        _{\text{очев. $=o(\norm{h})$}} +
        \underbrace{d_a \lambda(h) \cdot d_a f(h)}_{\text{1.}} + \\ +
        \underbrace{d_a \lambda(h) \cdot \alpha(h) \norm{h}}_{\text{2.}} + 
        \underbrace{d_a f(h) \cdot \beta(h) \norm{h}}_{\text{2.}} +
        \underbrace{\alpha(h) \beta(h) \norm{h}^2}_{\text{3.}}
    \end{gather*}
    \begin{enumerate}
        \item Поймём, что $d_a \lambda(h) \cdot d_a f(h) = o(\norm{h})$:
        $$\norm{\underbrace{d_a \lambda(h)}_{\text{число}} \cdot d_a f(h)}
        = \abs{d_a \lambda(h)} \cdot \norm{d_a f(h)} \leqslant
        \norm{d_a \lambda} \cdot \norm{h} \cdot \norm{d_a f(h)}
        \leqslant \underbrace{\norm{d_a \lambda} \cdot \norm{d_a f}}
        _{\text{= const}} \cdot \norm{h}^2 = o(\norm{h})$$

        \item $d_a \lambda(h) \cdot \alpha(h) \norm{h} = o(\norm{h})$,
        т.к. $\alpha(h) \to 0$ при $h \to 0$ по опр., $\norm{d_a \lambda(h)}
        \leqslant \norm{d_a \lambda} \cdot \norm{h}$ $\Rightarrow$
        $d_a \lambda(h)$ ограничена в окрестности 0. Аналогично с 
        $d_a f(h) \cdot \beta(h) \norm{h}$.

        \item $\alpha(h) \beta(h) \norm{h}^2 = o(\norm{h})$, т.к.
        $\alpha(h) \to 0$, $\beta(h) \to 0$, $\norm{h} \to 0$ при $h \to 0$.
    \end{enumerate} 
\end{proof}

\begin{theorem}[дифф. скалярного произведения]
    Пусть $f, g : E \to \R^m$, $E \subset \R^n$, $a \in \Int E$,
    $f$ и $g$ дифф. в точке $a$. Тогда $\langle f, g \rangle$
    дифф. в точке $a$ и $d_a \langle f, g \rangle (h) =
    \langle d_a f(h), g(a) \rangle + \langle f(a), d_a g(h) \rangle$
\end{theorem}
\begin{proof}
    \begin{gather*}
        \langle f, g \rangle = \sum_{k = 1}^m f_k g_k
        \Rightarrow 
        \underbrace{d_a \langle f, g \rangle = 
        \sum_{k = 1}^m d_a (f_k g_k)}
        _\text{линейность дифф.}; \\
        d_a \langle f, g \rangle (h) = 
        \sum_{k = 1}^m d_a (f_k g_k)(h) =
        \sum_{k = 1}^m 
        \underbrace{(d_a f_k (h) \cdot g_k(a) + 
        f_k (a) \cdot d_a g_k(h))}_{\text{пред. теорема}} = \\
        = \sum_{k = 1}^m 
        \underbrace{(d_a f(h))_k}_{\text{$k$-я коорд.}}
        \cdot g_k(a) + 
        \sum_{k = 1}^m f_k (a) \cdot (d_a g(h))_k =
        \langle d_a f(h), g(a) \rangle + \langle f(a), d_a g(h) \rangle
    \end{gather*}

    \textit{Необходимо пояснение, почему $d_a f_k (h) = (d_a f(h))_k$}
\end{proof}
\textbf{Замечание.} Если $n = 1$, то $\langle f(x), g(x) \rangle' =
\langle f'(x), g(x) \rangle + \langle f(x), g'(x) \rangle$

\begin{theorem}[теорема Лагранжа для векторнозначных функций] $ $\\
    Пусть $f : [a, b] \to \R^m$, $f$ непрерывна и
    дифф. на $[a, b]$. \\ Тогда $\exists c \in (a, b)$,
    т. ч. $\norm{f(b) - f(a)} \leqslant \norm{f'(c)}(b - a)$.
\end{theorem}
\begin{proof}
    Пусть $\varphi : [a, b] \to \R^m$,
    $\varphi(x) := \langle f(x), f(b) - f(a) \rangle$. Она непрерывна
    и дифф. на $(a, b)$, т.к. просто является произведением константы
    на непр. и дифф. на $(a, b)$ функцию.

    По замечанию к пред. теореме:
    $$\varphi'(x) = \langle f(x), f(b) - f(a) \rangle' =
    \langle f'(x), f(b) - f(a) \rangle +
    \langle f(x), \underbrace{(f(b) - f(a))'}_{= 0} \rangle =
    \langle f'(x), f(b) - f(a) \rangle$$

    Применим одномерную теорему Лагранжа:
    $$\exists c \in (a, b) : \varphi(b) - \varphi(a) =
    \varphi'(c)(b - a) = 
    (b - a) \langle \varphi'(c), f(b) - f(a) \rangle$$

    С другой стороны:
    $$\varphi(b) - \varphi(a) = \langle f(b), f(b) - f(a) \rangle
    - \langle f(a), f(b) - f(a) \rangle = \norm{f(b) - f(a)}^2$$

    Собираем вместе:
    $$\norm{f(b) - f(a)}^2 = 
    (b - a) \langle \varphi'(c), f(b) - f(a) \rangle
    \stackrel{\text{КБШ}}{\leqslant} (b - a) \norm{\varphi'(c)}
    \cdot \norm{f(b) - f(a)}$$

    Если $f(b) - f(a)$, то неравенство в условии очевидно выполняется,
    иначе можно сократить на $\norm{f(b) - f(a)}$.
\end{proof}
\textbf{Замечание.} Равенство $\norm{f(b) - f(a)} = \norm{f'(c)}(b - a)$
может не достигаться ни в одной точке.

\textbf{Пример:}

Пусть $f(x) = (\cos x, \sin x)$, $x \in [a, b] = [0, 2\pi]$.
Эта функция дифф., т.к. дифф. эквивалентна дифф. коорд. функций.
\begin{gather*}
    f(b) - f(a) = f(2\pi) - f(0) = (0, 0) \Rightarrow
    \norm{f(b) - f(a)} = 0, \\
    f'(x) = (-\sin x, \cos x) \Rightarrow \norm{f'(x)}^2 =
    (-\sin x)^2 + \cos^2 x = 1 \Rightarrow \norm{f'(x)} \equiv 1 \\
    \norm{f(b) - f(a)} = 0 < \norm{f'(x)}(b - a) = 2\pi
\end{gather*}