\section{Лекция номер 12}
\subsection{До перерыва}
gg
\subsection{После перерыва}
Докажем еще пару теорем необходимых для доказательства теоремы об обратной функции.
\begin{theorem}
    Пусть $f: \R^n \to \R^m$ -- функция дифференерцируемая в шаре $B_r(a)$ и $\forall x \in B_r(a)$ норма $\| f'(x) \| \leqslant C$. 
    Тогда $\forall x, y \in B_r(a)$ выполняется $\| f(x) - f(y) \| \leqslant C \| x - y \|$.
\end{theorem}
\begin{proof}
    Введем $\varphi:[0, 1] \to \R \;\; \varphi(t) = \langle f(x + t(y - x)), f(y) - f(x) \rangle$.
    Она дифференерцируема, так как $f(x + t(y - x))$ дифференерцируема, ведь отрезок $[x, y] \in B_r(a)$, $f(y) - f(x)$ -- константный вектор, и скалярное произведение дифференерцируемо.

    \quad Согласно одномерной теореме Лагранжа: $\varphi(1) - \varphi(0) = \varphi'(\theta)$, где $\theta \in (0, 1)$. 
    Распишем по формуле дифференцирования скалярного произведения: \begin{gather*}
        \begin{split}
            \varphi'(\theta) &= \langle f'(x + \theta(y - x))*(y - x), f(y) - f(x) \rangle + \underbrace{\langle f(x + \theta(y - x)), 0 \rangle}_0  \\
            &\overset{\text{КБШ}}{\leqslant} \| f'(x + \theta(y - x))*(y - x) \| * \| f(y) - f(x) \| \\
            &\leqslant \| f'(x + \theta(y - x))\| * \| y - x \| * \| f(y) - f(x) \| \\
            &\leqslant C\| y - x \| * \| f(y) - f(x) \|
        \end{split}
    \end{gather*}
    \quad Распишем $\varphi(1) - \varphi(0)$ по определению: \begin{gather*}
        \varphi(1) - \varphi(0) = \langle f(y), f(y) - f(x) \rangle - \langle f(x), f(y) - f(x) \rangle = \\
        = \langle f(y), f(y) \rangle - 2 \langle f(y), f(x) \rangle +  \langle f(x), f(x) \rangle = \| f(y) - f(x) \|^2 \\ \\
        \Rightarrow \| f(y) - f(x) \|^2 \leqslant C \| y - x \| * \| f(y) - f(x) \| \\
        \Rightarrow \| f(y) - f(x) \| \leqslant C \| y - x \|
    \end{gather*} 
\end{proof}

\begin{theorem} (об обратимости оператора близкого к обратимому) \\
    Пусть $\A: \R^n \to \R^n$ -- линейный обратимый оператор, $\B: \R^n \to \R^n$ -- просто линейный оператор, и выполняется $\| \B - \A \| \leqslant \frac{1}{\| \A^{-1} \|}$ (они достаточно близки).
    Тогда $\B$ обратим, \\ $\| \B^{-1} \| \leqslant \frac{1}{\frac{1}{\| \A^{-1} \|} - \| \B - \A \|}$ и обратные также достаточно близки $\| \B^{-1} - \A^{-1} \| \leqslant \frac{\| \A^{-1} \| * \| \B - \A \|}{\frac{1}{\| \A^{-1} \|} - \| \B - \A \|}$.
\end{theorem}
\begin{proof}
    Напишем неравенство треугольника: \begin{gather*}
        \| \A x \| = \| (\A - \B)x + \B x \| \leqslant \| (\A - \B)x \| + \| \B x \| \\
        \Rightarrow \| \B x \| \geqslant \| \A x \| - \| (\A - \B)x \|
    \end{gather*}
    \quad Заметим, что по стандартному неравенству $\| (\A - \B)x \| \leqslant \| \A - \B \| \| x \|$, а также $\| \A^{-1}(\A x) \| \leqslant \| \A^{-1} \| \| \A x \| \Rightarrow \| \A x \| \geqslant \frac{\| x \|}{\| \A^{-1} \| }$. 
    Подставим все это в неравенство: \begin{gather*}
        \| \B x \| \geqslant  \frac{\| x \|}{\| \A^{-1} \|} - \| \A - \B \| \| x \| = \underbrace{\left(\frac{1}{\| \A^{-1} \|} - \| \A - \B \| \right)}_{=: m} \| x \|
    \end{gather*}
    \quad Тогда по предпредыдущей тоереме $\B$ обратим и $\| \B^{-1} \| \leqslant \frac{1}{m} = \frac{1}{\frac{1}{\| \A^{-1} \|} - \| \B - \A \|}$. 
    Осталось только неравенство на норму разности: \[ \| \B^{-1} - \A^{-1} \| = \| \B^{-1}(\A - \B)\A^{-1} \| \leqslant \| \B^{-1}\| \| \A - \B \| \| \A^{-1} \| \leqslant \frac{1}{m}\| \A - \B \| \| \A^{-1} \| = \frac{\| \A^{-1} \| * \| \B - \A \|}{\frac{1}{\| \A^{-1} \|} - \| \B - \A \|}  \] 
\end{proof}

Теперь мы готовы сформулировать и доказать главную теорему данного параграфа -- теорему об обратной функции.
Храбров назвал ее самой сложной теоремой курса, так что пристегните ремни.

\underline{Мотивация}

\quad \textit{На данный момент мы знаем условия на то, чтобы линейное отображние было обратимо.
Например, определитель его матрицы не должен быть равен 0. 
Хочется понять, существуют ли такие условия для не линейного, а просто непрерывно дифференерцируемого отображения.
Оказывается, что глобальной обратимости у нас не будет, а вот локальная вполне будет существовать, если отображение будет достаточно хорошим.
Более формально: если у нас в точке дифференциал обратим, то в небольшой окрестности этой точки наше отображение будет вести себя примерно как линейное, а значит будет обратимо.}

\begin{theorem} (об обратной функции) \\
    Пусть
    \begin{itemize}
        \item $f: D \to \R^n$, где $D \subset \R^n$ -- открытое
        \item $x_0 \in D$, причем $f$ непрерывно дифференерцируемо в окр-ти $x_0$; $y_0 = f(x_0)$
        \item линейное отображение $\A = f'(x_0)$ обратимо (дифференциал в точке обратим)
    \end{itemize} 
    Тогда $\exists \, U$ -- окр-ть точки $x_0$, $\exists \, V$ -- окр-ть точки $y_0$, т.ч. $f: U \to V$ обратимо и  $f^{-1}: V \to U$ непрерывно.
\end{theorem}
\begin{proof}
    Введем отображение $G_y(x) := x + \A^{-1}(y - f(x))$.
    Выберем $B_r(x_0)$ так, что $\forall x \in B_r(x_0)$ выполняется $\| \A^{-1} \| * \| \A - f'(x) \| \leqslant \frac{1}{2}$.
    Мы действительно так можем сделать, потому что $\| \A^{-1} \|$ -- это константа, а $f$ непрерывно дифференерцируемо в окр-ти $x_0$, т.е. при $x$ близком к $x_0$ имеем $f'(x)$ близкое к $f'(x_0)$. 
    Благодаря этому неравенству мы можем применить предыдущую тоерему. Получаем, что $f'(x)$ обратим для $x \in B_r(x_0)$.

    \quad Теперь мы хотим, чтобы $G_y(x)$ было сжатием. 
    Для этого нам надо понять, что норма ее производной небольшая: \[ \| G_y'(x) \| = \| \underbrace{\mathcal{E}}_{\text{ед. оп.}} + \underbrace{(\A^{-1}(y))'}_{= c' = 0} - \underbrace{(d_{f(x)}\A^{-1} \circ d_xf)}_{\text{опр. диф. комп.}} \| = \| \mathcal{E} - \A^{-1}(f'(x)) \| =\circledast \]
    \quad Мы воспользовались тем, что $d_{f(x)}\A^{-1} = \A^{-1}$. 
    Это так, потому что $\A^{-1}$ -- линейное отображение, следовательно, дифференциал, посчитанный в любой точке, равен ему самому.
    Продолжим оценивать норму производной: \[ \circledast = \| \A^{-1}(\A - f'(x)) \| \leqslant \| \A^{-1} \| * \|(\A - f'(x)) \| \leqslant \frac{1}{2}  \]
    \quad Значит, $G_y$ -- сжатие с коэффициентом $\frac{1}{2}$. 
    Мы были немного неаккуратны, так как при больших $y$ при применни отображения $G_y(x)$ мы могли выскочить из шара $B_r(x)$, а это бы сломало наши рассуждения.

    \quad Чтобы этого не произошло, подберем $B_R(y_0)$ так, что $\forall y \in B_R(y_0)$ выполняется $G_y(B_r(x_0)) \subset B_r(x_0)$.
    Оценим, как далеко мы отдаляемся при $y \in B_R(y_0)$:
    \begin{gather*}
        \begin{split}
            \| G_y(x) - x_0 \| &\leqslant  \underbrace{\| G_y(x_0) - x_0 \|}_{= x_0 + \A^{-1}(y-f(x_0)) - x_0} + \| G_y(x) - G_y(x_0) \| \\
            &= \| \A^{-1}(y - y_0)\| + \| G_y(x) - G_y(x_0) \| \\
            &\leqslant \| \A^{-1} \| \| y - y_0 \| + \frac{1}{2}\| x - x_0 \| \leqslant R\| \A^{-1} \| + \frac{r}{2}
        \end{split}
    \end{gather*}
    \quad Мы хотим, чтобы $G_y(x)$ попало в шар $B_r(x_0)$. Tаким образом, должно выполняться \\ $\| G_y(x) - x_0 \| < r$.
    С помощью предыдущего неравенства мы легко подберем нужное $R$ и получим необходимый шар $B_R(y_0)$.

    \quad Теперь воспользуемся теоремой Банаха о сжатии. $G_y$ -- это сжимающее отображение, поэтому обязана существовать неподвижная точка:
    \[ \exists \, x \in B_r(x_0) : x = G_y(x) = x + \A^{-1}(y - f(x)) \Rightarrow \A^{-1}(y - f(x)) = 0 \Rightarrow f(x) = y \]
    \quad Заметим, что такой $x$ будет единственнен, так как если $f(x) = y$, то $x$ -- неподвижная точка, а она всего одна.
    Следовательно, у нас нашлась такая окрестность точки $y_0$, что для каждой точки из нее найдется единтсвенный $x$, т.ч. $f(x) = y$.
    Положим $V := B_R(y_0)$ и $U := f^{-1}(V)$ -- открытая окрестность $x_0$.
    Таким образом: $f: U \to V$ -- биекция $\Rightarrow$ существует $f^{-1}: V \to U$.

    \quad Осталось проверить непрерывность $f^{-1}$.
    Пусть $G_y(x) = x$ и $G_{\widetilde{y}}(\widetilde{x}) = \widetilde{x}$.
    Тогда, как мы поняли, $f(x) = y$ и $f(\widetilde{x}) = \widetilde{y}$.
    Нам нужно оценить норму разности обратного отображения: 
    \begin{gather*}
        \| f^{-1}(y) - f^{-1}(\widetilde{y}) \| = \| x - \widetilde{x} \| \underbrace{\leqslant}_{\text{сл. т. Банаха}} \frac{1}{1 - \frac{1}{2}} \| G_y(x) - G_{\widetilde{y}}(x) \| = \\
        = 2 \| x + \A^{-1}(y - f(x)) - x - \A^{-1}(\widetilde{y} - f(x)) \| = 2 \| \A^{-1}(y - \widetilde{y}) \| \leqslant \\
        \leqslant 2 \| \A^{-1} \| \| y - \widetilde{y} \|
    \end{gather*}
    \quad Это и есть критерий непрерывности.
\end{proof}

\vspace*{5mm}

Оказывается, что данное обратное отображение будет не только непрерывным, но и дифференерцируемым.
\begin{theorem} (о дифференерцируемости обратного отображения) \\
    Пусть \begin{itemize}
        \item $f: X \to Y$  -- непрерывное отображение
        \item $f(a) = b$, $U$ -- окр-ть точки $a$, $V$ -- окр-ть точки $b$
        \item $f$ дифференерцируема в точке $a$, $\A = f'(a)$ обратимо
        \item $f^{-1}: V \to U$ существует и непрерывна
    \end{itemize}
    Тогда $g := f^{-1}$ дифференерцируема в точке $b$.
\end{theorem}
\begin{proof}
    Распишем дифференерцируемость в точке $a$: $f(a + h) = f(a) + \A h + \alpha(h)\| h \|$, где $\alpha(h) \to 0$ при $h \to 0$.
    Введем $k := f(a + h) - f(a) = \A h + \alpha(h)\| h \|$. 
    Заведем следующее неравенство: \[ \| h \| = \| \A^{-1}(\A h) \| \leqslant \| \A^{-1} \| \| \A h \| \Rightarrow \| \A h \| \geqslant \frac{\| h \|}{\| \A^{-1} \|} \]
    \quad Используем его при оценке нормы $k$: \[ \| k \| = \| \A h + \alpha(h)\| h \| \| \geqslant \frac{\| h \|}{\| \A^{-1} \|} + \| h \|\|\alpha(h) \| = \| h \| \underbrace{\left(\frac{1}{\| \A^{-1} \|} + \| \alpha(h) \| \right)}_{=:\, C \, > \, 0} \]
    \quad Если $k \to 0$, то $\| h \| \left(\frac{1}{\| \A^{-1} \|} + \| \alpha(h) \| \right) \to 0$, но скобка не будет стремиться к 0, так как $\frac{1}{\| \A^{-1} \|}$ это какая-то константа, поэтому $h \to 0 \text{ т.к. $\alpha(h) \to 0$ }$.
    Вспомним, что $k = f(a+h) - f(a)$, а $g = f^{-1}$. 
    \quad Тогда: \begin{gather*}
        g(\underbrace{b + k}_{f(a+h)}) - g(\underbrace{b}_{f(a)}) = a + h - a = h = \circledast 
    \end{gather*}
    \quad Чтобы выразить $h$ через $k$, применим $\A^{-1}$ к равентсву $k = \A h + \alpha(h)\| h \|$ : \begin{gather*}
        \circledast = \A^{-1}k - \A^{-1}(\alpha(h) \| h \|) \\
        \Rightarrow g(b + k) = g(b) + \A^{-1}k - \A^{-1}(\alpha(h) \| h \|)
    \end{gather*}
    \quad Осталось понять, что $\| \A^{-1}(\alpha(h) \| h \| \| = o(\|k\|)$:
    \[ \| \A^{-1}(\alpha(h) \| h \| \| \leqslant \| \A^{-1} \| * \|\alpha(h) \| * \| h \| \leqslant \| \A^{-1} \| * \underbrace{\|\alpha(h) \|}_{\to 0} * \frac{\|k\|}{C}\  \]
\end{proof}

\begin{follow}
    В теореме об обратной функции $f^{-1}$ непрерывно дифференерцируемо.
\end{follow}
\begin{proof}
    Мы поняли, что если обратная функция существует и $f'$ обратима в точке $a$, то обратная функция дифференерцируема в точке $f(a)$.
    Это позволяет нам понять, что обратная функция дифференерцируема во всех точках, ведь в теореме об обратной функции мы выбирали окрестность $a$ так, что $f'$ там обратима.
    
    \quad На самом деле мы знаем как выглядит матрица $(f^{-1})'$.
    Это обратная матрица к $f'$.
    Мы можем посчитать обратную матрицу с помощью формулы с минорами.
    Тогда мы будем производить разные арифметические операции с частными производными (ведь именно они составляют матрицу $f'$), а они непрерывны,ведь функция была непрерывно дифференерцируема. 
\end{proof}

\begin{follow}
    
\end{follow}