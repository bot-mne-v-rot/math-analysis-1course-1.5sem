% !TEX root = ../MatanColloc03.tex

\section{Лекция номер 11}

\subsection{Непрерывная дифференцируемость}

\begin{theorem}
    Пусть $f : E \to \R$, $E \subset \R^n$, $a \in \Int f$,
    все частные производные $f$ существуют в окрестности точки $a$
    и непрерывны в точке $a$. Тогда $f$ дифф. в точке $a$.
\end{theorem}
\begin{proof} $ $

    Пусть:
    $$R(h) = f(a + h) - f(a) - 
    \underbrace{\sum_{k = 1}^n \frac{df}{dx_k}(a) \cdot h_k}
    _{= \langle \nabla f(a), h \rangle = d_a f(h)} $$

    где традиционно $h \in \R^n$. Надо док-ть, что $R(h) = o(\norm{h})$.

    Введём вспомогательные точки:
    
    \noindent
    \begin{minipage}[t]{.5\textwidth}
        $$ b_k = \begin{pmatrix*}
            a_1 + h_1 \\
            a_2 + h_2 \\
            \vdots \\
            a_k + h_k \\
            a_{k + 1} \\
            \vdots \\
            a_n
        \end{pmatrix*}
        = a + \begin{pmatrix*}
            h_1 \\
            h_2 \\
            \vdots \\
            h_k \\
            0 \\
            \vdots \\
            0
        \end{pmatrix*} $$
    \end{minipage}% <---------------- Note the use of "%"
    \begin{minipage}[t]{.5\textwidth}
    Пример на
    двумерном случае:
    $$\begin{tikzpicture}
        \draw[thick,->] (0,0) -- (3,0) node[anchor=north west] {x};
        \draw[thick,->] (0,0) -- (0,3) node[anchor=south east] {y};
        \fill[thick, black] (0.5, 1) circle(0.07) 
        node[below right]{$a = b_0$};
        \fill[thick, black] (2.5, 1) circle(0.07) 
        node[below right]{$b_1$};
        \fill[thick, black] (2.5, 2.5) circle(0.07) 
        node[below right]{$h = b_2$};
    \end{tikzpicture}$$
    \end{minipage}

    Т.е. мы к первым $k$ координатам $a$ прибавляем первые $k$
    координат $h$, а остальные $n - k$ координат $a$ оставляем не
    тронутыми. По сути, эти точки -- способ за $n$ шагов прошагать от точки
    $a$ до точки $h$, прибавляя по одной координате. 

    Зададим ещё одну функцию:
    $$ F_K(t) = f(b_{k-1} + t h_k e_k) $$
    где традиционно $t \in \R$, $e_k$ -- $k$-й базисный вектор. Т.е.
    мы сдвинулись по $k-1$ координате в сторону $h$ и теперь двигаемся
    по $k$-й координате в направлении $h_k$. Более того, 
    $F_k(1) = f(b_k)$.

    Это одномерная функция, поэтому можем применить одномерную
    теорему Лагранжа:
    $$ F_k(1) - F_k(0) = F_k'(\Theta_k) \quad\quad \Theta_k \in (0, 1) $$

    Давайте внимательно посмотрим на $F_k(t)$:
    $$ F_K(t) = f(b_{k-1} + t h_k e_k) =
    f\begin{pmatrix*}
        a_1 + h_1 \\
        \vdots \\
        a_{k-1} + h_{k-1} \\
        a_k + t h_k \\
        a_{k + 1} \\
        \vdots \\
        a_n
    \end{pmatrix*}$$
    Т.е. меняется только $k$-я координата, а значит по сути:
    $$ F_k'(t) = 
    \underbrace{
        \overbrace{\frac{df}{dx_k}(b_{k-1} + t h_k e_k)}
        ^{\text{сущ. по условию теор.}}
    \cdot \frac{d(b_{k-1} + t h_k e_k)}{dx_k}(t)}
    _{\text{произв. композиции}} = \frac{df}{dx_k}(b_{k-1} + t h_k e_k)
    \cdot h_k $$
    Подставляем $\Theta_k$:
    $$ F_k'(\Theta_k) = 
    \frac{df}{dx_k}(b_{k-1} + \Theta_k h_k e_k) \cdot h_k $$
    Собираем вместе:
    $$f(a + h) - f(a) = F_n(1) - F_1(0)
    \underbrace{= \sum_{k = 1}^n (F_k(1) - F_k(0))}
    _{\text{т.к. $F_{k-1}(1) = F_k(0)$}} =
    \sum_{k = 1}^n \frac{df}{dx_k}
    (\overbrace{b_{k-1} + \Theta_k h_k e_k}^{=: c_k}) \cdot h_k
    = \sum_{k = 1}^n \frac{df}{dx_k} (c_k) \cdot h_k $$
    Вспомним про $R(h)$:
    $$R(h) = f(a + h) - f(a) - \sum_{k = 1}^n \frac{df}{dx_k}(a) 
    \cdot h_k = \sum_{k = 1}^n \left(\frac{df}{dx_k} (c_k) -
    \frac{df}{dx_k}(a) \right) \cdot h_k$$
    Применим неравенство Коши-Буняковского:
    $$\abs{R(h)} = \abs{\sum_{k = 1}^n \left(\frac{df}{dx_k} (c_k) -
    \frac{df}{dx_k}(a) \right) \cdot h_k}
    \stackrel{\text{КБШ}}{\leqslant} \sqrt{\sum_{k=1}^n 
    \left(\frac{df}{dx_k} (c_k) - \frac{df}{dx_k}(a) \right)^2}
    \cdot \underbrace{\sqrt{\sum_{k=1}^n h_k^2}}_{= \norm{h}}
    $$
    Получаем, что:
    $$ \left(\frac{R(h)}{\norm{h}} \right)^2 \leqslant  \sum_{k=1}^n 
    \left(\frac{df}{dx_k} (c_k) - \frac{df}{dx_k}(a) \right)^2 $$
    Нужно доказать, что эта сумма стремится к 0 при $h \to 0$.
    Вспомним, что сходимость эквивалентна покоординатной сходимости:
    \begin{gather*}
        h \to 0 \Rightarrow h_k \to 0 \Rightarrow a_k + h_k \to a_k
        \Rightarrow b_k \to a \\
        c_k = \underbrace{b_{k-1}}_{\to a} + 
        \underbrace{\Theta_k}_{\in (0, 1)} \cdot
        \underbrace{h_k}_{\to 0} \cdot
        \underbrace{e_k}_{\text{const}} \to a
    \end{gather*}
    $\frac{df}{dx_k}$ непрерывна в точке $a$, а значит:
    $$\lim_{h \to 0} \frac{df}{dx_k}(c_k) = 
    \lim_{c_k \to a} \frac{df}{dx_k}(c_k) = 
    \frac{df}{dx_k}(a)$$
    Собираем вместе:
    $$\left(\frac{R(h)}{\norm{h}} \right)^2 \leqslant  \sum_{k=1}^n 
    \underbrace{\left(\frac{df}{dx_k} (c_k) - 
    \frac{df}{dx_k}(a) \right)^2}_{\to 0} \to 0 \Longrightarrow
    R(h) = o(\norm{h})$$
    По определению дифференцируемости получаем, что $f$ дифф. 
    в точке $a$.
\end{proof}

\textit{Теперь у нас есть более удобный способ проверять дифференц.
функции в точке, чем просто по определению, т.к. если функция
состоит из элементарных функции, то нам легко понять, какие у неё
частные производные, и легко понять, непрерывны ли они в точке.}

\textbf{Замечания:}
\begin{enumerate}
    \item Непрерывность частной производной по первой координате
    не используется:
    $$ F_1(1) - F_1(0) = f(a + h_1 e_1) - f(a) = 
    \underbrace{\frac{df}{dx_1}(a) \cdot h_1 + o(h_1)}_
    {\text{опр. дифф. в точке $a$}} = 
    \frac{df}{dx_1}(a) \cdot h_1 + o(\norm{h})$$
    По второй координате мы уже так сделать не можем:
    $$ F_2(1) - F_2(0) = 
    \underbrace{f(a + h_1 e_1 + h_2 e_2) - f(a + h_1 e_1)}
    _{\text{точка $(a + h_1 e_1) \neq a$}} $$

    \item Дифф. в точке не даёт существования частных производных
    в окр-ти и уж тем более их дифф. 

    \textbf{Пример:}
    $$ f(x, y) = \begin{cases}
        x^2 + y^2, & \text{если ровно одно из них $\in \Q$,} \\
        0, &\text{иначе.}
    \end{cases} $$

    $f$ дифф. в точке $(0, 0)$:
    $$ \underbrace{f(x, y)}_{\leqslant x^2 + y^2} = 
    \underbrace{f(0, 0)}_{=0} + o(\norm{(x, y)}) = 
    f(0, 0) + o(\sqrt{x^2 + y^2}) $$

    При этом, в любой окрестности $(0, 0)$ найдётся точка, 
    отличная от $(0, 0)$, в которой нет даже непрерывности.
    Пусть $x_0, y_0 \in \R \setminus \Q$. $f(x_0, y_0) = 0$.
    Зафиксируем вторую координату и будем двигаться по первой
    координате. Тогда в любой окрестности $x_0$ найдётся рациональное
    число $x$, а значит, там будет $f(x, y_0) \neq 0$. 
    При этом если брать рациональные точки слева от $x_0$, то при 
    приближении к $x_0$, значение $f(x, y_0)$ будет только расти,
    а значит, ни о какой сходимости к 0 говорить нельзя.
\end{enumerate}

\begin{conj}
    Пусть $f : E \to \R^m$, $E \subset \R^n$, $a \in \Int E$.
    Тогда $f$ \textbf{непрерывно дифф.} в точке $a$, 
    если $f$ дифф. в точке 
    $a$ и $d_x f$ непрерывен в точке $a$, т.е. $\norm{d_x f - d_a f}
    \to 0$ при $x \to a$.
\end{conj}

\begin{theorem}
    $f$ непрерывно дифф. $\Longleftrightarrow$ $f$ дифф. в окрестности
    точки $a$ и все её частные производные непрерывны в точке $a$.
\end{theorem}
\begin{proof} $ $

    \begin{itemize}
        \item[``$\Longleftarrow$'':]
        $$ d_x f =
        \begin{pmatrix*}
            \frac{df_1}{dx_1}(x) & \frac{df_1}{dx_2}(x) &
            \dots & \frac{df_1}{dx_n}(x) \\
            \frac{df_2}{dx_1}(x) & \frac{df_2}{dx_2}(x) &
            \dots & \frac{df_2}{dx_n}(x) \\
            \vdots & \vdots & \ddots & \vdots \\
            \frac{df_m}{dx_1}(x) & \frac{df_m}{dx_2}(x) &
            \dots & \frac{df_m}{dx_n}(x) \\
        \end{pmatrix*}$$

        Нужно доказать, что $\norm{d_x f - d_a f} \to 0$ при $x \to a$.
        Оценим квадрат нормы этого линейного оператора:
        $$ \norm{d_x f - d_a f}^2 \leqslant 
        \sum_{j, k} \left( \frac{df_k}{dx_j}(x) -
        \frac{df_k}{dx_j}(a) \right)^2 $$

        $\frac{df_k}{dx_j}(x)$ непрерывна в точке $a$, поэтому:
        $$ \frac{df_k}{dx_j}(x) - \frac{df_k}{dx_j}(a) \to 0
        \text{ при } x \to a $$

        \item[``$\Longrightarrow$'':]
    
        $$ \underbrace{\begin{pmatrix*}
            \frac{df_1}{dx_j}(x) \\
            \frac{df_2}{dx_j}(x) \\
            \vdots \\
            \frac{df_m}{dx_j}(x)
        \end{pmatrix*}}_{\text{$j$-й столбец $d_x f$}} =
        \begin{pmatrix*}
            \frac{df_1}{dx_1}(x) & \frac{df_1}{dx_2}(x) &
            \dots & \frac{df_1}{dx_n}(x) \\
            \frac{df_2}{dx_1}(x) & \frac{df_2}{dx_2}(x) &
            \dots & \frac{df_2}{dx_n}(x) \\
            \vdots & \vdots & \ddots & \vdots \\
            \frac{df_m}{dx_1}(x) & \frac{df_m}{dx_2}(x) &
            \dots & \frac{df_m}{dx_n}(x) \\
        \end{pmatrix*} \cdot
        \begin{pmatrix*}
            0 \\ \vdots \\ 0 \\ 1 \\ 0 \\ \vdots \\ 0
        \end{pmatrix*}
        \begin{matrix*}
            $ $ \\ $ $ \\ $ $ \\
            \longleftarrow \text{$j$-е место}
            \\ $ $ \\ $ $ \\ $ $
        \end{matrix*} = d_x f (e_j)$$

        Таким образом:
        \begin{gather*}
            \abs{\frac{d f_k}{d x_j}(x) - \frac{d f_k}{d x_j}(a)}
            = \norm{\begin{pmatrix*}
                0 \\ \vdots \\ 0 \\ 
                \frac{d f_k}{d x_j}(x) - \frac{d f_k}{d x_j}(a) 
                \\ 0 \\ \vdots \\ 0
            \end{pmatrix*}}
            \leqslant
            \norm{\begin{pmatrix*}
                \frac{df_1}{dx_j}(x) - \frac{df_1}{dx_j}(a) \\
                \frac{df_2}{dx_j}(x) - \frac{df_2}{dx_j}(a) \\
                \vdots \\
                \frac{df_m}{dx_j}(x) - \frac{df_m}{dx_j}(a)
            \end{pmatrix*}} = \\
            = \norm{d_x f(e_j) - d_a f(e_j)} =
            \norm{(d_x f - d_a f)(e_j)} \leqslant
            \norm{d_x f - d_a f} \cdot \underbrace{\norm{e_j}}_{= 1}
            = \norm{d_x f - d_a f} \xrightarrow[x \to a]{} 0
        \end{gather*}
    \end{itemize}
    \textit{Возможно, это доказательство записано излишне подробно.
    Всякие рассуждения с матрицами на экзамене можно проговаривать
    устно.}
\end{proof}

\begin{theorem}
    Непрерывная дифф. сохраняется при взятии лин. комбинации,
    композиции, скал. произведения.
\end{theorem}
\begin{proof}
    Предыдущая теорема говорила о том, что непрерывная 
    дифференцируемость в точке эквивалентна непрерывности всех
    коэффициентов матрицы дифференциала в этой точке.

    Вспомним, что матрицы дифференциалов при взятии лин. комбинации,
    композиции, скал. произведения -- это какие-то линейные комбинации
    или произведения матриц дифференциалов исходных функций. Если
    все коэфф. исходных матриц были непрерывны, то после перемножения
    или сложения все коэфф. итоговой матрицы тоже будут непрерывны.
\end{proof}

\subsection{Производные высших порядков}

\begin{conj}
    Пусть $f : E \to \R$, $E \subset \R^n$, $a \in \Int E$.
    Если существует $\frac{df}{d x_i}$ в окрестности $a$
    и у неё есть частная производная $\frac{d}{d x_j}$,
    то она называется \textbf{частной производной второго порядка}.
    \textit{Т.е. это частная производная по частной производной}

    Обозначение (длинное):
    $$\frac{d^2 f}{d x_j d x_i} = \frac{d \frac{d f}{d x_i}}{d x_j}$$
    \textit{Храбров считает, что такую запись надо читать так:
    сначала дифф. по $i$-й координате, потом по $j$-й. В некоторых
    книжках считают иначе.}

    Альтернативное обозначение:
    $$f''_{x_i x_j} = (f'_{x_i})_{x_j}$$
\end{conj}

\begin{conj}
    Если существует в окр. точки $a$ частная производная порядка $r$:
    $$ \frac{d^r f}{d x_{i_r} d x_{i_{r-1}} \dots d x_{i_1}} $$
    и у этой производной есть частная производная $\frac{d}{d x_{r+1}}$,
    то результат:
    $$ \frac{d^{r+1} f}{d x_{i_{r+1}} d x_{i_{r}} \dots d x_{i_1}} $$
    называют \textbf{частной производной порядка $r+1$}.
\end{conj}

\textbf{Пример:}
$$f(x, y) = x^y \quad\quad \frac{df}{dx} = y x^{y - 1} 
\quad\quad \frac{df}{dy} = x^y \cdot \ln x$$
\begin{align*}
    \frac{d^2 f}{d x^2} &= \frac{d}{dx} \left( \frac{df}{dx} \right)
    = \frac{d}{dx} \left( y x^{y - 1} \right) = y(y-1) x^{y-2} \\
    \frac{d^2 f}{d y^2} &= \frac{d}{dy} \left( \frac{df}{dy} \right)
    = \frac{d}{dy} \left( x^y \cdot \ln x \right) = x^y \cdot \ln^2 x \\
    \frac{d^2 f}{d y d x} &= \frac{d}{dy} \left( \frac{df}{dx} \right)
    = \frac{d}{dy} \left( y \cdot x^{y - 1} \right) = x^{y-1} + 
    y \cdot x^{y-1} \cdot \ln x \\
    \frac{d^2 f}{d x d y} &= \frac{d}{dx} \left( \frac{df}{dy} \right)
    = \frac{d}{dx} \left( x^y \cdot \ln x \right) =
    y \cdot x^{y - 1} \cdot \ln x + x^y \cdot \frac{1}{x} =
    \frac{d^2 f}{d y d x}
\end{align*}
\textit{Но частные производные по $x,y$ и по $y,x$ совпадают не всегда.}

\textbf{Пример:}
$$
f(x, y) = \begin{cases}
    xy \cdot \frac{x^2 - y^2}{x^2 + y^2} \text{ при } x^2 + y^2 \neq 0,\\
    0, \text{ иначе.}
\end{cases}
$$
Воспользуемся тем, что:
$$\frac{x^2 - y^2}{x^2 + y^2} = 1 - \frac{2y^2}{x^2 + y^2}$$

Пусть $x^2 + y^2 \neq 0$. Тогда:
$$\frac{df}{dx} = y \cdot \frac{x^2 - y^2}{x^2 + y^2} + 
xy(-2y^2) \frac{-2x}{(x^2 + y^2)^2} =
y\frac{x^4 - y^4 + 4x^2y^2}{x^2 + y^2}$$

$$\frac{df}{dx}(0, 0) = \lim_{h \to 0} \frac{f(h, 0) - f(0, 0)}{h} = 0$$
$$\frac{d^2f}{dy dx} (0, 0) = \lim_{h \to 0} 
\frac{\frac{df}{dx}(0, h) - \frac{df}{dx}(0, 0)}{h} =
\lim_{h \to 0} \frac{1}{h} \cdot h \cdot (-1) = -1
$$
Легко заметить, взглянув на формулу, что производная по $x, y$ ---
это производная по $y, x$, но с другим знаком, поэтому:
$$\frac{d^2f}{dx dy} (0, 0) = 1 \neq \frac{d^2f}{dy dx} (0, 0)$$

\begin{theorem}
    Пусть $f : E \to \R$, $E \subset \R^2$, $(x_0, y_0) \in \Int E$,
    частные производные $f'_x$, $f'_y$, $f''_{xy}$ существуют в
    окрестности точки $(x_0, y_0)$ и $f''_{xy}$ непрерывна в 
    $(x_0, y_0)$. Тогда существует $f''_{yx}(x_0, y_0)$ и 
    $f''_{xy}(x_0, y_0) = f''_{yx}(x_0, y_0)$.
\end{theorem}
\begin{proof}
    Введём:
    $$\varphi(s) := f(s, y_0 + k) - f(s, y_0)$$
    Эта функция дифферецируема в окрестности $x_0$, 
    т.к. производная $\varphi(s)$
    --- это частная производная $f$ по $x$ в окрестности $(x_0, y_0 + k)$
    минус частная производная $f$ по $y$ в окрестности $(x_0, y_0)$.

    Т.к. эта функция дифференцируема, мы можем написать для неё теорему
    Лагранжа:
    $$ \varphi(x_0 + h) - \varphi(x_0) = 
    \underset{\text{где } \Theta_1 \in (0, 1)}
    { h \varphi'(x_0 + \Theta_1 h)} =
    h (f_x'(x_0 + \Theta_1 h, y_0 + k) - 
    f_x'(x_0 + \Theta_1 h, y_0)) = \oast $$
    Введём ещё:
    $$ \psi(t) := f_x'(x_0 + \Theta_1 h, t)$$
    Эта функция тоже дифференцируема в окрестности $y_0$,
    т.к. производная этой функции -- это просто частная производная
    $f$ сначала по $x$, потом по $y$. А она существует по условию.
    
    Т.к. эта ф-я дифф., то про неё тоже можно написать теорему
    Лагранжа. Тогда:
    $$ \oast = h(\psi(y_0 + k) - \psi(y_0)) =
    \underset{\text{где } \Theta_2 \in (0, 1)}
    {hk \cdot \psi'(y_0 + \Theta_2 k)} =
    hk \cdot f''_{xy}(x_0 + \Theta_1 h, y_0 + \Theta_2 k) $$

    Получили в итоге:
    $$ f(x_0 + h, y_0 + k) - f(x_0 + h, y_0) - f(x_0, y_0 + k)
    + f(x_0, y_0) = hk \cdot 
    f''_{xy}(x_0 + \Theta_1 h, y_0 + \Theta_2 k) = \circledcirc $$

    Воспользуемся теперь непрерывностью $f''_{xy}$ в точке $(x_0, y_0)$:
    $$ \circledcirc = hk \cdot (f''_{xy}(x_0, y_0) + \alpha(h, k)) 
    \text{, где $\alpha(h, k) \to 0$ при $(h, k) \to 0$} $$
    Поделим выражение на $hk$. Тогда:
    $$ \abs{\alpha(h, k)} = \abs{\frac{1}{h} \left( 
        \underbrace{\frac{f(x_0 + h, y_0 + k) - f(x_0 + h, y_0)}{k}}
        _{\longrightarrow f'_y(x_0 + h, y_0) \text{ при $k \to 0$}} - 
        \underbrace{\frac{f(x_0, y_0 + k) - f(x_0, y_0)}{k}}
        _{\longrightarrow f'_y(x_0, y_0) \text{ при $k \to 0$}}
    \right) - f''_{xy}(x_0, y_0)} $$
    При этом для любого $\varepsilon > 0$ верно, что
    $ \abs{\alpha(h, k)} < \varepsilon$ при малых $h, k$.

    Сделаем предельный переход:
    $$ \abs{\frac{f'_y(x_0 + h, y_0) - f'_y(x_0, y_0)}{h}
    - f''_{xy}(x_0, y_0)} 
    \leqslant \varepsilon \text{ при малых $h$} $$

    Получаем:
    $$ f''_{yx}(x_0, y_0) = \lim_{h \to 0} 
    \frac{f'_y(x_0 + h, y_0) - f'_y(x_0, y_0)}{h}
    = f''_{xy}(x_0, y_0) $$

\end{proof}

\begin{theorem}
    Если $f'_x$ и $f'_y$ определены в окрестности точки $(x_0, y_0)$
    и дифф. в этой точке, то $f''_{xy}(x_0, y_0) = f''_{yx}(x_0, y_0)$.
\end{theorem}
\begin{proof}
    \textit{Упражнение.}
\end{proof}

\begin{conj}
    $f : D \to \R$, $D \subset \R^n$, $D$ -- открытое множество. \\
    $f$ \textbf{$\mathbf{r}$ раз непрерывно дифференцируема} 
    ($f$ --- \textbf{$\mathbf{r}$-гладкая} функция), если 
    все частные производные до $r$-го порядка существуют и
    непрерывны.

    Обозначение: $f \in C^r(D)$.
\end{conj}

\begin{theorem}
    Пусть $f \in C^r(D)$. $(i_1, \dots, i_n)$ --- перестановка 
    $(j_1, \dots, j_n)$. Тогда:
    $$ \frac{d^r f}{d x_{i_1} d x_{i_2} \dots d x_{i_r}} = 
    \frac{d^r f}{d x_{j_1} d x_{j_2} \dots d x_{j_r}} $$
\end{theorem}
\begin{proof} $ $

    Рассмотрим случай, когда $(i_1, \dots, i_r)$ --- транспозиция
    соседних $(j_1, \dots, j_r)$, т.е. 
    $$(i_1, \dots, i_r) = (j_1, \dots, j_{k - 1}, 
    \underbrace{j_{k + 1}, j_k}_{\text{переставили}}, 
    j_{k + 2}, \dots, j_r)$$

    Пусть:
    $$ g(x) := \frac{d^{r - k - 1} f}{d x_{j_{k + 2}} \dots d x_{j_r}}
    \quad \quad
    A := \frac{d^r f}{d x_{i_1} d x_{i_2} \dots d x_{i_r}} \quad
    \quad B := \frac{d^r f}{d x_{j_1} d x_{j_2} \dots d x_{j_r}} $$
    Тогда:
    $$ A = \frac{d^{k-1}}{d x_1 \dots d x_{k-1}} 
    \frac{d^{2}}{d x_{k+1} d x_{k}} g \quad \quad 
    B = \frac{d^{k-1}}{d x_1 \dots d x_{k-1}} 
    \frac{d^{2}}{d x_{k} d x_{k+1}} g$$
    По предыдущей теореме (не упражнение):
    $$ \frac{d^{2}}{d x_{k+1} d x_{k}} g = 
    \frac{d^{2}}{d x_{k} d x_{k+1}} g $$
    А дальше мы навешиваем производные в одинаковом порядке.
    Поэтому $A = B$.

    Теперь воспользуемся фактом из линейной алгебры о том, что любая
    перестановка раскладывается в конечное произведение транспозиций
    соседних элементов. Действительно, если смена местами соседних
    элементов в перестановке не меняет производную, 
    то мы можем сделать конечное
    количество таких действий и производная всё равно не поменяется.
\end{proof}

Так как мы поняли, что в случае $r$-гладких функций порядок взятия 
производных не имеет значения, а важно только то, сколько раз мы берём 
производную по 
каждой переменной производную;  введём следующее более компактное
обозначение: 
\begin{conj}
    \textbf{Мультииндекс.}
    $$ k = (k_1, \dots, k_n) \text{, где $k_i$ --- неотрицательные
    целые числа, $n$ --- размерность области опр. $f$} $$
    \textbf{Высота мультииндекса}: 
    $\quad \abs{k} = k_1 + \dots + k_n$.
    
    \textbf{Факториал мультииндекса}: 
    $k! = k_1! \cdot k_2! \cdot ... \cdot k_n!$.

    \textbf{Возведение вектора $h \in \R^n$ в степень мультииндекса}:
    $h^k = h_1^{k_1} \cdot h_2^{k_2} \cdot ... \cdot h_n^{k_n} \in \R$.

    \textbf{Частная производная по мультииндексу:}
    $$ f^{(k)} = \frac{d^{\abs{k}} f}
    {d x_1^{k_1} \cdot \dots \cdot d x_n^{k_n}} $$
\end{conj}

\begin{conj}
    \textbf{Мультиномиальный коэффициент:}
    $$ \binom{\abs{k}}{k_1, k_2, \dots, k_n} = \frac{\abs{k}}{k!} $$

    Комбинаторный смысл: покрасить $\abs{k}$ шаров в $n$ -- цветов так,
    что $k_1$ шаров 1-го цвета, $k_2$ шаров -- 2-го и т.д.
    \begin{gather*} 
        \binom{\abs{k}}{k_1, k_2, \dots, k_n} =
        C^{k_1}_{\abs{k}} \cdot 
        C^{k_2}_{\abs{k} - k_1} \cdot 
        C^{k_3}_{\abs{k} - k_1 - k_2} \cdot \dots 
        C^{k_n}_{\abs{k} - k_1 - ... - k_{n-1}} = \\
        = \frac{\abs{k}!}{k_1! \cdot \cancel{(\abs{k} - k_1)!}}
        \cdot \frac{\cancel{(\abs{k} - k_1)!}}
        {k_2! \cdot \cancel{(\abs{k} - k_1 - k_2)!}}
        \cdot \dots \cdot
        \frac{\cancel{(\abs{k} - k_1 - \dots - k_{n-1})!}}
        {k_n! \cdot \underbrace{(\abs{k} - k_1 - \dots - k_n)!}
        _{= 0! = 1}} =
        \\ = \frac{\abs{k}}{k_1! k_2! \dots  k_n!}
        =  \frac{\abs{k}}{k!}
    \end{gather*}
\end{conj}

%Техническая лемма на последних 20 минутах 11ой лекции
\begin{lemma}
    $f \in C^r(D), \; D \subset \R^n$ -- открыто, отрезок $[x, x+h]$ целиком содержится в $D$. Введем функцию
    \begin{gather*}
        F(t) := f(x + th), \qquad F:[0, 1] \longrightarrow \R 
    \end{gather*}
    На отрезке $[0, 1]$ функция точно корректно определена, так как мы затребовали, чтобы весь отрезок $[x, x+h]$ лежал в $D$. 
    Тогда лемма утверждает, что:
    \begin{gather*}
         F \in C^r[0, 1] \text{ и } F^{(l)}(t) = \sum\limits_{\abs{k} = l} \binom{l}{k_1, k_2, \dots, k_n} f^{(k)} (x+th)h^k
    \end{gather*}
\end{lemma}
\begin{proof}
    Давайте сначала попробуем понять, что происходит для одной производной. Введем
    $G(t) := g(x+th)$, где $g$ -- некоторая другая функция из $D$ в $\R$. 
    Хотим посчитать ее производную в некой точке. Так как функция $G$ -- это по сути композиция двух функций, одна из которых 
    с векторным аргументом, а другая -- принимает скаляр и выплевывает вектор, то по формуле производной композиции:
    \begin{gather*}
        G'(t) = g'(x + th)\cdot(x+th)' = \oast
    \end{gather*} 
    Давайте понимать, что есть что. $g$ у нас дейстовало из $D$ в $\R$, то есть матрица $g'(x+th)$ -- это на самом деле вектор строчка из частных производных. 
    В то же время, $x+th:[0,1] \longrightarrow \R^n$. То есть это просто столбец координат $x_i + th_i$. Значит производная этой штуки -- просто столбец производных координат по $t$. 
    А производные по $t$-шкам -- это соответствующие им $h$--ки. Получаем: 
    \begin{gather*}
        \oast = \left( g'_{x_1}(x+th), \dots, g'_{x_n}(x+th) \right) \cdot \begin{pmatrix*}
            h_1 \\
            h_2 \\
            \vdots \\
            h_n
        \end{pmatrix*} = \sum\limits_{i=1}^n g'_{x_i}(x+th)h_i
    \end{gather*}
    Вот мы посчитали первую производную, получили что-то страшное. Когда мы будем дифференцировать много раз -- будем много раз брать производные и все складывать. Выйдет куча вложенных друг в друга сумм. 
    Давайте посмотрим, что мы получим, взяв $l$-ую производную (перейдем к функции $F$):
    \begin{gather*}
        F^{(l)} = \sum\limits_{i_l} \dots \sum\limits_{i_1} \frac{d^l f}{dx_{i_l} dx_{i_{l-1}} \dots dx_{i_1}}(x+th)h_{i_1}h_{i_2}\dots h_{i_l} = \oast
    \end{gather*}
    Давайте свернем эту сумму. Введем такой вот мультииндекс $k$: 
    \begin{gather*}
        k = (\# \{j: i_j = 1\}, \# \{j: i_j = 2\}, \dots, \# \{j: i_j = n\})
    \end{gather*}
    Тогда в нашей сумме:
    $$ \frac{d^l f}{dx_{i_l} dx_{i_{l-1}} \dots dx_{i_1}}(x+th)
    = f^{(k)}(x + th) \quad \quad h_{i_1}h_{i_2}\dots h_{i_l} = h^k $$
    Теперь нужно понять, сколько раз мы такое слагаемое встретим.
    По сути, мы $x_1$ ``красим'' $k_1$ раз, $x_2$ ``красим'' $k_2$ раз 
    и т.д.
    Количество способов так ``покрасить'' и есть 
    комбинаторный смысл мультиномиального коэффициента.

    Тогда нашу сумму можно переписать с использованием этого мультииндекса как: 
    \begin{gather*}
        \oast = \sum\limits_{\abs{k} = l} f^{(k)} (x+th) \binom{l}{k_1, k_2, \dots, k_n} h^k
    \end{gather*}
\end{proof}
